\documentclass[10pt]{book}   	 
\usepackage{geometry}                		 
\geometry{a4paper}  
 
%\usepackage{draftwatermark}
%\SetWatermarkAngle{45}
%\SetWatermarkLightness{0.8}
%\SetWatermarkFontSize{4cm}
%\SetWatermarkScale{2}
%\SetWatermarkText{DRAFT  - NO DISTRIBUTION }
               		
\usepackage{graphicx}	
\usepackage{hyperref}			
\hypersetup{
    colorlinks,
    citecolor=black,
    filecolor=black,
    linkcolor=black,
    urlcolor=black
}
\usepackage{amssymb}
\usepackage{longtable}
\usepackage{listings} \lstset{numbers=left, numberstyle=\tiny, numbersep=5pt} \lstset{language=C}
\usepackage{marginnote}
\usepackage{color}
\newcommand{\todo}[1]{\marginpar{\color{red}#1}}
\newcommand{\zwave}{Z-Wave \raise0.8ex\hbox{\tiny TM} }
\title{Z-Way Developers Documentation}
\author{(c) Z-Wave.Me Team, based on Version 2.0}
\date{}							 

\begin{document}
\maketitle
\tableofcontents


TODOs: check cap 2,2.1,2.2, describe Restore Function 

\chapter{Introduction and Overview}
\label{intro}

\section{How to use Z-Way}

\begin{figure} 
\includegraphics[scale=0.8]{pics/apis.png}
\caption{Z-Way APIs and their use by GUI demos}
\label{apis} 
\end{figure}

Z-Way offers multiple Application Programmers Interfaces (API) that are partly built on each other.
Figure \ref{apis} shows the general structure of Z-Way with focus on the APIs. The most important part of
Z-Way is the Z-Wave core. The Z-Wave core uses the standard
Sigma Designs Serial API to communicate with a Z-Wave compatible transceiver hardware but 
enhanced with some Z-Way specific functions such as Frequency Change. The standard
interface is not public but available for owners of the Sigma Designs Development Kit (SDK) 
\footnote{The Sigma Designs SDK is available from Digikey (www.digikey.com). Depending on the 
hardware options chosen the price varies between 2000 and 4000 USD} only.  

The Z-Wave core services can be accessed directly using the Z-Wave Device API (zDev API). There are two
Z-Wave Device API versions available:
\begin{itemize}
\item JSON API: All functions are available using a JSON API implemented by an 
embedded webserver. The "Expert" UI is using this interface and serves both as programmers 
and installers User Interface to operate the network and as reference implementation to demonstrate the use of this 
API. The Expert UI is completely written in AJAX Technology.
\item C Library API: All functions of the JSON API are available as C library 
function too. In the folder /z-way-devel there are all header files with the function 
prototypes. All function calls and the whole data model are identical. The URL

\paragraph{http://razberry.zwave.me/fileadmin/z-way-test.zip}
\todo{update this package}

provides a sample application written in standard C that makes use of the C level API to 
demonstrate its application. Makefiles and project files for compilation on Linux and 
OSX are provided together with the sample code.


\end{itemize}

The \textbf{Z-Wave device API only allows the management of the Z-Wave network} and the control
and management of the devices as such. No higher order logic except the so called associations
between two Z-Wave devices can be used here.

For all \textbf{automation and higher order logic a Javascript automation engine} is available.
This engine is also shipped with Z-Way. There is also a possibility to create your own home automation engine
working in addition or instead  of the original Z-Way Home Automation engine. 
The implementation of the Java Script engine is organized in so called modules that implement 
a broad variety of applications using the underlying Z-Wave devices. The automation logic 
can also access and use other third party technology stacks such as the EnOcean stack
or any technologies based on HTTP.

The Z-Way Home Automation (HA) solution offers multiple prefabricated modules.
In addition \textbf{Z-Way HA implements
a Virtual Device API (vDev API)}. The Virtual Devices API uses information
from Z-Wave Device API to create virtual devices based in Z-Wave devices and
represent them in a unified way to upper level clients using Virtual Device API
and provide access to them by JSON based user interfaces.
The Demo GUI z-way-ha uses this API and demonstrates the use of the Virtual Device API.


\section{Quick Start}

The best way to learn about Z-Way is to use it. That is why the solutions comes with two
reference User Interfaces that can be accesses using a web browser. This allows to 
built, manage and use a first wireless smart home network with Z-Wave device (and/or with
devices using a third party wireless technology such as EnOcean if supported by Z-Way) 
without writing a single line of code.
 
Once Z-Way is installed turn your browser to the URL

\paragraph{http://YOURIP:8083/expert}

\paragraph{or}

\paragraph{http://YOURIP:8083/z-way-ha} 

 
\paragraph{to use demonstation UIs available with Z-Way.} The port 8083 is predefined
(in automation/main.js).

In most cases you will want to control Z-Wave devices with Z-Way. Before a new device 
can be used, the device needs to be included in the Z-Wave network managed by Z-Way.
This management function is accessible in the Expert UI under the Tab 'Network'. 
Just hit the button "Include new Device" and then confirm the inclusion of the new device 
by hitting a button on the device or the specific action defined for this device to 
confirm inclusion.

You may need to refer to the manual of the new device for further information on how 
to confirm an inclusion by a controller.

The whole Expert UI is described in the Document 
\textbf{'Z-Way Expert User Interface Manual'} and therefore no scope of this document.


\section{API Overview}

All communication between the User Interfaces and Z-Way is handled using a 
web-technology-based  JSON interface provided by a built in web server.

There are different Application Programmers Interfaces available for Z-Way that serve 
different purposes:
\begin{enumerate}
\item Z-Wave Device API (zDev)
\item Third Party Technology APIs
\item JavaScript API
\item Virtual Device API/Business Logic API (vDev)
\end{enumerate}

\subsection{Z-Wave Device API}

The Z-Wave Device API implements the direct access to the Z-Wave network as such.
All Z-Wave devices are referred to by their unique identification in the wireless network - 
the Node Id. Z-Wave devices may have different instances of the same function, also 
called channels (for example sockets in a power strip). The Z-Wave Device API refers to them as daughter objects of the 
physical device object identified by an instance Id. In case there only one instance the 
instance Id = 0  is used.

All device variables and available commands in a Z-Wave device are grouped in so called
command classes. The Z-Wave API allows direct access to all parameters, values and 
commands of these command class structures.

Beside the devices the Z-Wave Device API also offers access to the management interface
of the network. These functions are implemented as so called function classes within 
the object 'controller' or on the top level 'z-way' object.

The Z-Wave Device API can be accessed on the JSON API using the url path 



\paragraph{http://YOURIP:8083/ZWaveAPI/*}
\paragraph{Device objects or commands of these objects are accessed by} 
\paragraph{http://YOURIP:8083/ZWaveAPI/Run/devices[*].* }
\paragraph{http://YOURIP:8083/ZWaveAPI/Run/devices[x].instances[y].*}
\paragraph{http://YOURIP:8083/ZWaveAPI/Run/devices[x].instances[y].commandClasses[z].*} 
\paragraph{the whole data tree of the Z-Wave network is accessed using}  
\paragraph{http://YOURIP:8083/ZWaveAPI/Data/*}

\paragraph{
\textbf{Attention: The Installer UI is a complete reference of the Z-Wave API. It shows how 
to use all function it reveals the dynamics of the stack backend and visualizes all 
internal variables accessible on the API. }}

The chapter \ref{c2} describes the Z-Wave Device API in detail.

The section 'Function Classes' in chapter \ref{c2:fc} explains the different management 
functions and how they  can be used and will use the Expert UI dialogs as application example. The 
section 'Function Class Reference' in chapter \ref{FunctionClasses} all function 
classes available.

The control of devices  is implemented in Command Classes. The section 'Command Class 
Implementation' is again using certain dialogs of the Expert UI to explain how to access 
these functions. The chapter \ref{ccs} documents all 
command class functions. 

In order to access device and network related data in Z-Way, knowledge of the Z-Way 
data model is essential. The chapter \ref{datamodel} gives the necessary insight 
into the data model. All data of Z-Way are exposed on the Expert UI.

\subsection{Third Party Technology API}

Third party Technology APIs implement the same logic as the Z-Wave device API for other 
wireless technologies such as EnOcean.

For more information  please refer to the technology specific descriptions.

\subsection{JavaScript API (JS API)}

The Z-Wave Device API or any other Third Party technology API do 
not offer any higher order logic support but the pure access to functions and parameters 
only.

Z-Way offers an automation engine to overcome this restriction. A server-side JavaScript
Run time environment allows writing JavaScript modules that are executed within Z-Way 
(means on the server). The same time all functions of the JS API can also be access on 
the client side (the web browser). This offers some cool debug and test capabilities. 
Among others it is possible to write whole JS functions right into the URL or the browser.

The JS API can be accessed from the web browser with the URL

\paragraph{http://YOURIP:8083/JS/Run/*}


\paragraph{Among others the whole Z-Wave Device API is available within the JS API using 
the object 'zway'.} As a result the following three statements refer to the 
very same function:

\begin{enumerate}
\item \textbf{http://YOURIP:8083/ZWaveAPI/Run/devices[3].*} Client Side URL access using 
the Z-Wave Device API.
\item \textbf{http://YOURIP:8083/JS/Run/zway.devices[3].*}: Client Side URL access 
using the JS API
\item \textbf{zway.devices[3].*}: Server Side access using the JS and the public zway object
\end{enumerate}

Due to the scripting nature of JavaScript it is possible to 'inject' code at run time
using the interface. Here a nice example how to use the Java Script 
setInterval function:

\begin{lstlisting}[caption=Polling of device \#2]{Name}
/JS/Run/setInterval(function() { 
	zway.devices[2].Basic.Get();
}, 300*1000);
\end{lstlisting}

This code will, once 'executed' as URL within a web browser, call the Get() command
of the command class Basic of Node Id 2 every 300 seconds.  

A very powerful function of the JS API is the ability to bind functions to certain
values of the device tree. They get then executed when the value changes. Here is an 
example for this binding. The device No. 3 has a command class SensorMultilevel that offers
the variable level. The following call - both available on the client side 
and on the server side - will bind a simple alert function to the change of 
the variable.

\begin{lstlisting}[caption=Bind a function]{Name}
zway.devices[3].SensorMultilevel.data[1].val.bind( function() { 
	debugPrint('CHANGED TO:' + this.value + '\n'); 
});  
\end{lstlisting}\footnote{Please note that the Sensor Multilevel Command class data is an array
index by the scale Id. Other command classes such as Basic do not have this index but allow direct access using 
CommandClassName.data.level}

Chapter \ref{jsapi} and \ref{datamodel} describe the whole JS API in detail. The names 
and Ids of the different command classes as well as their instance variables can be found
in the Annex.

JavaScript modules can and will generate new functions that are accessible using the 
JSON interface. For simplification function calls on the API (means on the client
side) are written in URL style starting with the word 'ZAutomation':

\begin{center}
/ZAutomation/JSfunction/JParameter
== JSfunction(JParameter)
\end{center}

\subsection{Virtual Device API}

One of the (server side) JavaScript modules already available is a mapping of all 
physical devices and functions into virtual devices. The purpose of this mapping is 
to simplify and to unify the implementation of a Graphical User Interface.

All functions and all instances of a physical device - that are represented as daughter
objects in the Z-Wave Device API -  are enrolled into individual virtual devices. 

In case the Z-Wave API shows one single physical device with two channels, while the Virtual Device 
API will show two devices with similar functionality. In case the Z-Wave API shows a physical
device with  several different functions (like a binary switch and a analog sensor in one 
device) the Virtual Device API (vDev API) will show them as several devices with one 
function each.

The vDev is accessed using the HTTP REST API in a slightly different style than zDev API. All 
devices, variables and commands are encoded into a URL style for easier handling in AJAX
code. A typical client side command in the vDev API looks like

\begin{center}

http://YOURIP:8083/ZAutomation/api/v1/devices/ZWayVDev\_6:0:37/command/off

\end{center}

'api' points to the vDev API function, 'v1' is just a constant to allow future extensions. 
The devices are referred by a name that is automatically generated from the Z-Wave 
Device API. The vDev also unifies the commands 'command' and the parameters, here 'off'.

On the server side the very same command would be encoded in a JavaScript style.

\begin{center}
	dev = this.controller.devices.get('ZWayVDev\_6:0:37');
	dev.command('off');
\end{center}

The vDev API also offers support for notifications, locations information, the use of other
modules etc. For details please refer to Chapter \ref{vdev}.

\subsection{Comparison}

Table \ref{c1:comp} summarizes the functions of the different APIs.


\begin{table}
\begin{tabular}{|p{40mm}|p{50mm}|p{20mm}|p{20mm}|}
\hline
API Type &	Core Function & Network Management & Automation\\
\hline
Z-Wave Dev API (JSON)	& Access to physical network and physical devices via JSON	&
Yes	&No\\
\hline
Z-Wave Dev API (C lib) 	& Access to physical network and physical devices via C style calls &
Yes	&No\\
\hline
JavaScript API & Access to physical network and devices plus JS type functions	&
No	&Yes, via zDev\\
\hline
vDev API & Unified Access to functions of devices, optimized for AJAX GUI&
No	&Yes\\
\hline
\hline
\end{tabular}
\caption{Different APIs of the Z-Way system} 
\label{c1:comp}
\end{table}		

For demo and sample code please refer to the following sources:

\begin{enumerate}
\item \textbf{Z-Wave Device API JSON} Expert UI on http://YOURIP:8083/expert
\item \textbf{Z-Wave C Library API} Header files in folder /z-way-devel 
\item \textbf{JavaScript API} Modules in folder /automation
\item \textbf{vDev API} z-way-ha UI on http://YOURIP:8083/z-way-ha
\end{enumerate}




  
\chapter{The Z-Wave Device API}
\label{c2}

This chapter describes the Z-Wave Device API and its use in detail All examples will use
the JSON REST API notation. Please not that the C library notation offers equal 
functionality but in a different notation.

The Z-Wave Device API is the north bound interface of the Z-Wave Core. This Z-Wave core
implement the whole control logic of the the Z-Wave network. The two main functions are
\begin{itemize}
\item Management of the network. This includes including and excluding devices, managing 
the routing and rerouting of the network  an executing some housekeeping functions 
to keep the network clean and stable. In the Z-Wave terminology all these functions 
are called 'function classes' and they are described in section \ref{c2:fc}. The function 
classes can be seen as functions offered by the controller itself. Hence the variables
and status parameters of the networks are offered by an object called 'controller'.

\item Execution of commands offered by the wireless devices as such switching switches 
and dimming dimmers. Z-Wave groups the command and their corresponding variables into
so called command classes. The Z-Wave API offers access to these command classes with
their variables and their commands according to the abilities od the respective device.
\end{itemize}
The next chapters first explain the timings of the communication in a wireless 
Z-Wave network. Then the data model is presented that reflects the real data and status 
information in the network.

The description of function Classes and command Classes and their access 
using the JSON API  complete the description of the Z-Wave Device API. For a full 
reference of function classes and command classes please refer to the Annex.  % Z-Wave Device API korrigiert
\section{Executing a command from the GUI to the device and back}

\textbf{Please note that all status variables accessible on the Z-Wave Device APIs
are only proxy of the real value in the network.}

To transport data between the real wireless device and the GUI multiple communication 
instances are involved. The complexity of this communication chain shall be 
explained in the following example:

\begin{figure} 
\begin{center}
\includegraphics[scale=0.6]{pics/zway2en.png}
\caption{Z-Way Timings}
\label{zwaytimings} 
\end{center} 
\end{figure}

Assuming the GUI shows the status of a remote switch and allows to change the switching 
state of this device. When the user hits the switching button he expects to 
see the result of his action as a changing status of the device in the GUI. The first 
step is to hand over the command (SET)  from the GUI to Z-Way using the JSON interface.
Z-Way receives the command and will confirm the reception to the GUI. Z-Way recognizes 
that the execution of the switching command will likely result in a change 
of the status variable However Z-Way will not immediately change the status variable but 
invalidate the actual value (mark as outdated). This is the correct action because at the moment
when the command was received the status on the remote device has not been changed yet 
but the status of the switch is now unknown.
If the GUI polls the value it will still see the old value but marked as invalid.
Z-Way will now hand over the switching command to the Z-Wave transceiver chip. Since it 
is possible that there are other command waiting for execution (sending) by 
the Z-Wave transceiver chip the job queue is queuing them and will handle certain 
priorities if needed. Z-Way has recognized that the command will likely change the status
of the remote device and is therefore adding another command to call the actual status 
after the switching command was issued.
The transceiver is confirming the reception of the command and this confirmation is noted 
in the job queue. This confirmation however only means that the transceiver (Z-Wave chip)
has accepted the command and does neither indicate that the remote device has receives 
it nor even confirming that the remote device has executed accordingly.
The transceiver will now try to send the command wirelessly to the remote device. A 
successful confirmation of the reception from the remote device is the only valid 
indicator that the remote device has received the command (again, not that it was executed!).
The second command (GET) is now transmitted the very same way and confirmed by the remote 
device. This device will now sent a REPORT command back to Z-Way
reporting the new status of the switching device. Now the transceiver has to 
confirm the reception. The transceiver will then send the new value to the Z-Way 
engine by issuing commands via the serial interface. Z-Way receives the report and will 
update the switching state and validate the value.
From now on the GUI will receive a new state when polling.
 % Timings korrigiert
\chapter{The Z-Wave Device API}
\label{c2}

This chapter describes the Z-Wave Device API and its use in detail All examples will use
the JSON REST API notation. Please not that the C library notation offers equal 
functionality but in a different notation.

The Z-Wave Device API is the north bound interface of the Z-Wave Core. This Z-Wave core
implement the whole control logic of the the Z-Wave network. The two main functions are
\begin{itemize}
\item Management of the network. This includes including and excluding devices, managing 
the routing and rerouting of the network  an executing some housekeeping functions 
to keep the network clean and stable. In the Z-Wave terminology all these functions 
are called 'function classes' and they are described in section \ref{c2:fc}. The function 
classes can be seen as functions offered by the controller itself. Hence the variables
and status parameters of the networks are offered by an object called 'controller'.

\item Execution of commands offered by the wireless devices as such switching switches 
and dimming dimmers. Z-Wave groups the command and their corresponding variables into
so called command classes. The Z-Wave API offers access to these command classes with
their variables and their commands according to the abilities od the respective device.
\end{itemize}
The next chapters first explain the timings of the communication in a wireless 
Z-Wave network. Then the data model is presented that reflects the real data and status 
information in the network.

The description of function Classes and command Classes and their access 
using the JSON API  complete the description of the Z-Wave Device API. For a full 
reference of function classes and command classes please refer to the Annex.  % Data Model korrigiert

\section{Function Class Implementation}
\label{c2:fc}

The commands used to control the controller itself and ot manage the Z-Wave network
are called 'function classes'. Most function classes used in the Sigma Designs
Serial API are used by the Z-Way lower layer function only but some of them 
are exposed to the Z-Wave Device API to allow user interactions and network management.

The Expert UI is an excellent reference for the Function Classes. All relevant functions 
can be monitored 'in action'. Hence the description of the network tab of the Expert 
UI is more or less a complete reference to the function classes needed in a UI implementation.

 
\begin{figure} 
\begin{center}
\includegraphics[scale=0.8]{pics/network1.png}
\caption{Expert UI Dialog for Networking functions}
\label{c1:network1} 
\end{center} \end{figure}

\subsection{Inclusion}

 You can include devices by pressing the 'Include Device' button. This turns the 
controller into an inclusion mode that allows including a device.  A status information 
line indicates this status. The inclusion of a device is typically confirmed with a 
triple press of a button of this particular device. However, please refer to the manual of 
this particular device for details how to include them into a Z-Wave network. The 
inclusion mode will time out after about 20 seconds or is aborted by pressing the 
'Stop Include' button.

If the network has a special controller with SIS function (Z-Way will try to activate 
such as function on default, hence this mode should always be active if the USB 
hardware used by Z-Way supports it) the inclusion of further devices can also be 
accomplished by using the include function of any portable remote control which 
is already included into the network.   A short explanation above the include button 
will inform about the ways devices can be included.

The Inclusion function is implemented using the function class 
{\bf AddNodeToNetwork(flag)} with flag=1 for starting the inclusion mode and 
flag=0 for stopping the  inclusion mode. Please refer to the chapter \ref{FunctionClasses} 
for details on how to use this function.


\subsection{Exclusion}

You can exclude devices by pressing the 'Exclude Device' button. This turns the controller 
into an exclusion mode that allows excluding a device. 
The exclusion of a device is typically confirmed with a triple press of a button of this 
particular device as well. However, please refer to the manual 
of this device for details how to exclude them into a Z-Wave network. The exclusion mode 
will time out after about 20 seconds or is aborted by pressing the 
'Stop Exclude' button.
It is possible to exclude all kind of devices regardless if they were included in the 
particular network of the excluding controller.

If a node is not longer in operation it can’t be excluded from the network since exclusion 
needs some confirmation from the device. Please use the 'Remove Failed Node' function 
in this case. 
Please make sure that only failed nodes are moved this way. Removed but still function 
nodes  - called phantom nodes – will harm the network stability.


The Exclusion function is implemented using the function class 
{\bf RemoveNodeToNetwork(flag)} with flag=1 for starting the exclusion mode and flag=0 
for stopping the 
exclusion mode. Please refer to the chapter \ref{FunctionClasses} for details on how to use this function.

\subsection{Mark Battery powered devices as failed}

This function allows marking battery-powered devices as failed. Only devices marked as 
failed can be excluded from the network without using the exclusion 
function. Typically multiple failed communications with a device result in this marking. 
Battery powered devices are recognized as sleeping in the controller 
and therefore all communication attempts with this device will be queued until a wakeup 
notification from this device is received. A faulty battery operated 
device will never send a wakeup notification and hence there is never a communication, 
which would result in a failed node status. Battery operated devices 
can therefore be manually marked as faulty.  Make sure to only mark  and subsequently 
remove  devices that are faulty or have disappeared. A device, which 
was removed with this operation but is still functioning may create malfunctions in the network.

This function is no Function class but sets the internal 'failed' variable of the device
object.

\subsection{Remove Failed Nodes}

Z-Way allows removing a node, if and only if this node was detected as failed by the 
Z-Wave transceiver. The network will recognize that communication with a device fails 
multiple times and the device can’t be reached using alternating routes either. 
The controller will then mark the device as 'failed' but will keep it in the current 
network configuration.  Any successful communication with the device will remove the 
failed mark. Only devices marked as failed can be removed using the 'Remove Failed Node' 
function.

If you want to remove a node that is in operation use the 'Exclude' Function.


This function is implemented using the function class {\bf RemoveFailedNode(node id)} 
with node id as the node id of the device to be removed. Please refer to the chapter 
\ref{FunctionClasses} for details on how to use this function. It is also possible to 
replace a failed node by a new node using the function class {\bf RemoveFailedNode(node id)}. 
Please refer to the  chapter \ref{FunctionClasses} for details too.

The function {\bf IsFailedNode(node id)} can be used to detect if a certain node is 
failed. The Z-Wave transceiver will try to contact the device wirelessly and will then 
update the failed-status inside the transceiver and also the 'is failed' flag of the 
device object in Z-Way.

\subsection{Include into different network}

Z-Way can join a Z-Wave network as secondary controller. It will change its own Home ID to the Home ID of the new network and it will learn all network information 
from the including controller of the new network. To join a different network, the primary controller of this new network need to be in the inclusion mode.

Z-Way needs to be turned into the so called learn mode using the button 'Start Include in 
others network'. The button “Stop Include in others network” can be used to turn off 
the Learn mode, which will time out otherwise or will stop if the learning was successful.

Please be aware that \textbf{all existing relationships to existing nodes will get lost} 
when the Z-Way controller joins a different network. Hence it is recommended to join a 
different network only after a reset with no other nodes already included. 

The 'Learn' function is implemented using the function class {\bf SetLearnMode(flag)} 
with flag=1 for starting the learn mode and flag =0 for stopping the learn mode. Please 
refer to the chapter \ref{FunctionClasses} for details on how to use this function.


\subsection{Z-Wave chip reboot}

This function will perform a soft restart of the firmware of the Z-Wave controller chip 
without deleting any network information or setting. It may be necessary to recover the 
chip from a freezing state. A typical situation of a required chip reboot is if the 
Z-Wave chip fails to come back from the inclusion or exclusion state.

The reboot function is implemented using the function class {\bf SerialAPISoftReset()}.  
Please refer to the chapter \ref{FunctionClasses} for details on how to use this function.


\subsection{Request NIF from all devices}

This function will call the Node Information Frame from all devices in the network. 
This may be needed in case of a hardware change or when all devices 
where included with a portable USB stick such as e.g. Aeon Labs Z-Stick.  Mains powered 
devices will return their NIF immediately, battery 
operated devices will respond after the next wakeup.


This function controls a Z-Way controller function that will send out a function 
class {\bf RequestNodeInformation(node)} to all nodes in the network. 
The function can also be called for one single node only. Please refer to the 
chapter \ref{FunctionClasses} for details on how to use this function.

\subsection{Send controllers NIF}

In certain network configurations it may be required to send out the Node Information 
Frame of the Z-Way controller. This is particularly useful for some some remote 
controls scene activation function. The manual of the remote control will refer to this 
requirement and give further information when and how to use this function.

This function is implemented with the function class {\bf SerialAPIApplicationNodeInfo} 
with plenty of parameters. These parameters are partly set by Z-Way but particularly the 
Command classes supported (parameter 'NIF') can be changed by editing the file defaults.xml. 
Please refer to the chapter \ref{FunctionClasses} for details on how to use this function 
the chapter about the translation files on how to change defaults.xml

\subsection{Reset Controller}

The network configuration (assigned node Ids and the routing table and some other network 
management specific parameters) is stored in the Z-Wave 
transceiver chip and will therefore even survive a complete reinstallation of the Z-Way software.

The function 'Reset Controller' erases all values stored in the Z-Wave chip and sent the 
chip back to factory defaults. This means that \textbf{all network information will be lost 
without recovery option}.

This function is implemented with the function class {\bf SetDefault()}. Please refer 
to the chapter \ref{FunctionClasses} for details on how to use this function.

\subsection{Change Controller }

\begin{figure} 
\begin{center}
\includegraphics[scale=0.8]{pics/network2.png}
\caption{Demo UI Dialog for Networking functions - experts functions}
\label{c1:network2} 
\end{center} \end{figure}


The controller change function allows to handover the primary function to a different 
controller in the network. The function works like a normal inclusion 
function but will hand over the primary privilege to the new controller after inclusion. 
Z-Way will become a secondary controller of the network. This function may be needed 
during installation of larger networks based on remote controls only where Z-Way is 
solely used to do a convenient network 
setup and the primary function is finally handed over to one of the remote controls.

This function is implemented with the function class {\bf ControllerChange()}. Please 
refer to the chapter \ref{FunctionClasses} for details on how to use this function.

\subsection {SUC/SIS Management}

This interface allows controlling the SUC/SIS function for the Z-Wave network. All 
these functions are almost obsolete and only needed for certain enhanced configurations 
of the Z-Wave network. Unless you really know what you do - don't use these functions!

The following Function Classes are mapped to the demo user interface  functions 
for SUC/SIS manipulation:
\begin{itemize}
\item GetSUCNodeId - get the SUC Node ID from the network
\item EnableSUC - enables the SUC function in Z-Way, this is done by default if the transceiver firmware used supports SUC
\item SetSUCNodeId - assign SUC function to a node in the network that is capable of running there SUC function.
\item SendSUCNodeId - inform a different node about the node ID of a SUC in the system
\end{itemize}


 
\subsection{Routing Table}


\begin{figure} 
\begin{center}
\includegraphics[scale=0.5]{pics/routingtable.png}
\caption{Demo UI dialog for Routing Table}
\label{c2:demorouting} 
\end{center} \end{figure}

The routing table of the Z-Wave network is shown in the tab network as well. It indicates how two devices of the Z-Wave network can communicate with each other. 
If two devices are in direct range (they can communicate without the help of any other node) the cross point of the two devices in the table is marked as dark green. 
The color light green indicates that the two nodes are not in direct range but have more than one alternating routes with one node between. This is still considered 
as a stable connection.
The yellow color indicates that there are less than two “one-hop” routes available between the two nodes. However there may be more routes but with more nodes
 between and therefore considered as less stable.

A red indicator shows that there are no good short connections between the two nodes. This does not mean that they are unable to communicate with each other 
 but any route with more than 2 routers between Z-Way is considering as not reliable, even taking into account that Z-Wave supports routes with up to four devices 
 between. Grey cells indicate the connection to the own Node ID. 

The general rule of thumb is: 'The greener the better'.
 
The table lists all nodes on the y-axis and the neighborhood information on the x-axis. On the right hand side of the table a timestamp shows when the neighborhood
 information for a given node was reported.   

In theory the table should be totally symmetric, however different times of the neighborhood detection may result in different neighborhood information of the 
two devices involved.
 
The neighbor information of the controller works with an exception. The Z-Wave implementation used in current Z-Wave transceiver does not allow requesting an 
update of the neighbor list for the controller itself. The neighborhood information displayed for the controller is therefore simply wrong.

Battery powered devices will report their neighbors when woken up and report their mains powered neighbor correctly. However mains powered devices will report 
battery-powered devices as neighbors only when routes are updated twice. This is less critical because battery powered devices can’t be used as routers and are 
therefore not relevant for calculating route between two nodes anyway. 
 
The context menu command 'Network Reorganization' allows re-detecting all neighborhood information (battery powered devices will report after their next wakeup!) 
Please refer to the manual section 'Network Stability' for further information about the use of this function.

The routing table is stored in the Z-Wave transceiver and can be read using the function class {\bf  GetRoutingTableLine(node id)} for a given node ID. The function 
{\bf RequestNodeNeighbourUpdate(nodeid)} will cause a certain node Id to redirect its
 wireless neighbors. It makes sense to call the GetRoutingTable function right 
after successful callback of the RequestNodeNeighbourUpdate function. 

  % Function Classes
\section{JSON-API}
\label{jsapi}
JSON API allows to execute commands on server side using HTTP POST requests 
(currently GET requests are also allowed, but this might be deprecated in future). 
The command to execute is taken from the URL.

All functions are executed in form 

\paragraph{http://YOURIP:8083/$<$URL$>$}


\subsection{/ZWaveAPI/Run/$<$command$>$}

This executes zway.$<$command$>$ in JavaScript engine of the server. As an example to 
switch ON a device no 2 using the command class BASIC (The ID of the command class BASIC is 0x20, for more
information about the IDs of certain command classes please refer to the Annex) its 
possible to write:

\begin{quote}/ZWaveAPI/Run/devices[2].instances[0].commandClasses[0x20].Set(255)\end{quote}

or

\begin{quote}/ZWaveAPI/Run/devices[2].instances[0].Basic.Set(255)\end{quote}



The Z-Way Expert GUI has a JavaScript command runCmd($<$command$>$) to simplify such 
operations. This function  is accessable in the Javascript console of your web browser 
(in Chrome you find the JavaScript  console unter View-$>$Debug-$>$JS Console). Using 
this feature the command in JS console would look like

\begin{quote}runCmd('devices[2].instances[0].Basic.Set(255)')\end{quote}


The usual way to access a command class is using the format \\'devices[nodeId].instances[instanceId].
commandClasses[commandclassId]'.
There are ways to simplify the syntax:

\begin{itemize}
\item 'devices[nodeId].instances[instanceId].Basic' is equivalent to \\
'devices[nodeId].instances[instanceId].commandClasses[0x20]'
\item the instances[0] can be obmitted: 'devices[nodeId].instances[instanceId].Basic' 
then turns into 'devices[nodeId].Basic'
\end{itemize} 

Each Instance object has a device property that refers to the parent device it belongs to. 
Each Command Class Object has a device and an instance property that refers to the instance 
and the device this command class belongs to.
 

Data holder object have properties value, updateTime, invalidateTime, name, but for 
compatibility with JS and previous versions we have valueOf() method (allows to 
omit .value in JS code, hence write "data.level == 255"), updated (alias to updateTime), 
invalidated (alias toinvalidateTime). These aliases are not enumerated if the dataholder 
is requested (data.level returns {value: 255, name: "level", updatedTime: 12345678, 
invalidatedTime: 12345678}).

\subsection{/ZWaveAPI/InspectQueue}

This function is used to visualize the Z-Way job queue. This is for debugging only but 
very useful to understand the current state of Z-Way engine.

\subsection{/ZWaveAPI/Data/$<$timestamp$>$}
Returns an associative array of changes in Z-Way data tree since $<$timestamp$>$. The 
array consists of ($<$path$>$: $<$JSON object$>$) object pairs. The client is supposed 
to assign the new $<$JSON object$>$ to the subtree with the $<$path$>$ discarding previous 
content of that subtree. Zero (0) can be used instead of $<$timestamp$>$ to obtain the 
full Z-Way data tree.

The tree have same structure as the backend tree (Figure \ref{zwaystructure}) with 
one additional root element "updateTime" which contains the time of latest update. 
This "updateTime" value should be used in the next request for changes. 
All timestamps (including updateTime) corresponds to server local time.

Each node in the tree contains the following elements:
\begin{itemize}
\item value — the value itself
\item updateTime — timestamp of the last update of this particular value
\item invalidateTime — timestamp when the value was invalidated by issuing a Get 
command to a device and expecting a Report command from the device
\end{itemize}

The object looks like:
\begin{lstlisting}[caption=JSON Data Structure]{Name}
{
"[path from the root]": [updated subtree],
"[path from the root]": [updated subtree],
...
updateTime: [current timestamp]
}
\end{lstlisting}

Examples for Commands to update the data tree look like:

\begin{quote}Get all data: /ZWaveAPI/Data/0\end{quote}

\begin{quote} Get updates since 134500000 (Unix timestamp): /ZWaveAPI/Data/134500000\end{quote}

Please note that during data updates some values are updated by big subtrees. For example, 
in Meter Command Class value of a scale is always updated as a scale subtree by 
[scale].val object (containing scale and type descriptions).


\subsection{Handling of updates coming from Z-Way}

A good design of a UI is linking UI objects (label, textbox, slider, ...) to a certain 
path in the tree object. Any update of a subtree linked to UI will then update the UI too. 
This is called bindings.

For web applications Z-Way Web UI contains a library called jQuery.triggerPath 
(extention of jQuery written by Z-Wave.Me), that allows to make such links between 
objects in the tree and HTML DOM objects. Use

\begin{quote}var tree;\end{quote}
\begin{quote}jQuery.triggerPath.init(tree);\end{quote}

during web application initialization to attach the library to a tree object. Then run

\begin{quote}jQuery([objects selector]).bindPath([path with regexp], [updater function], 
[additional arguments]);\end{quote}

to make binding between path changes and updater function. The updater function would be 
called upon changes in the desired object with this pointing to the DOM object itself, 
first argument pointing to the updated object in the tree, second argument is the exact 
path of this object (fulfilling the regexp) and all other arguments 
copies additional arguments. RegExp allows only few control characters: * 
is a wildcard, (1|2|3) - is 1 or 
2 or 3.  

\textbf{Please not that the use of the triggerpath extension is one option to handle the incoming
data. You can also extract all the interesting values right when the data is received and 
bind update functions to them.} % JSON
\chapter{The Z-Wave Device API}
\label{c2}

This chapter describes the Z-Wave Device API and its use in detail All examples will use
the JSON REST API notation. Please not that the C library notation offers equal 
functionality but in a different notation.

The Z-Wave Device API is the north bound interface of the Z-Wave Core. This Z-Wave core
implement the whole control logic of the the Z-Wave network. The two main functions are
\begin{itemize}
\item Management of the network. This includes including and excluding devices, managing 
the routing and rerouting of the network  an executing some housekeeping functions 
to keep the network clean and stable. In the Z-Wave terminology all these functions 
are called 'function classes' and they are described in section \ref{c2:fc}. The function 
classes can be seen as functions offered by the controller itself. Hence the variables
and status parameters of the networks are offered by an object called 'controller'.

\item Execution of commands offered by the wireless devices as such switching switches 
and dimming dimmers. Z-Wave groups the command and their corresponding variables into
so called command classes. The Z-Wave API offers access to these command classes with
their variables and their commands according to the abilities od the respective device.
\end{itemize}
The next chapters first explain the timings of the communication in a wireless 
Z-Wave network. Then the data model is presented that reflects the real data and status 
information in the network.

The description of function Classes and command Classes and their access 
using the JSON API  complete the description of the Z-Wave Device API. For a full 
reference of function classes and command classes please refer to the Annex.  % Command Classes
\chapter{JavaScript API}
\label{jsapi}

The JavaScript API mirrors all functions of the Z-Wave device API and 
combines them with the ability to run JavaScript code on top of the 
Z-Wave Device functions and variables.

There are two ways to run JavaScript functions in Z-Way.
\begin{itemize}
\item They can be executed in the web browser URL string (using /JS/Run/ prefix)
\item They can be implemented as module running in the backend or be stored in a file on the server side.
\end{itemize}
Both options have their pros and cons. Running JS code in the browser is a very nice
and convenient way to test things but the function is not persistent.

Writing a module requires some more knowledge and debugging is more complicated. 
On the other hand the possibilities of JavaScript in the module are almost infinite 
and goes far beyond just accessing Z-Wave device. JavaScript as any other language
can make use of services available in the Internet and combine them in any possible 
way with information from the Z-Wave network and can execute functions within 
the Z-Wave network but the same time on any place accessible via Internet based 
services. In theory there is not even a need to have Z-Wave device in order to make 
use of the powerful JavaScript engine. As an example you can write a JavaScript 
module polling weather data from the internet and depending on certain well defined 
conditions the same module can send you a short message on your mobile.
Both functions are by the way already implemented as open source modules and can be 
accessed and studied for further modification or use.

\section {The JavaScript Engine}

Z-Way uses the JavaScript engine provided by Google referred to as V8. You find more 
information about this JavaScript implementation on https://code.google.com/p/v8/.
V8 implements JavaScript according to the specification ECMA 5
\footnote {http://www.ecma-international.org/publications/standards/Ecma-262.htm}.

Z-Way extends the basic functionality provided by V8 with plenty of application 
specific functions.

\section{Accessing the JS API}

The JS API can be accessed from any web browser with the URL

\paragraph{http://YOURIP:8083/JS/Run/*}

All functions of the Z-Wave Device API can be used by JavaScript. They are encapsulated
in the 'zway' object.  This object has the same structure as defined in chapter 
\ref{c2-data model}. 
The client side access to the device data is done like

\paragraph{http://YOURIP:8083/JS/Run/zway.devices[x].*}

Due to the scripting nature of JavaScript its possible to 'inject' code at run time
using the interface. Here a nice example how to use the Java Script 
setInterval function:

\begin{lstlisting}[caption=Polling device \#2]{Name}
/JS/Run/setInterval(function() { 
	zway.devices[2].Basic.Get();
}, 300*1000);
\end{lstlisting}

This code will, once 'executed' as URL within a web browser, call the Get() command
of the command class Basic of Node ID 2 every 300 seconds.  

A very powerful function of the JS API is the ability to bind functions to certain
values of the device tree. they get then executed when the value changes. Here an 
example for this binding. The device No. 3 has a command class SensorMultilevel that offers
the variable level. The following call - both available on the client side 
and on the server side - will bind a simple alert function to the change of 
the variable.

\begin{lstlisting}[caption=Bind a function]{Name}
zway.devices[3].SensorMultilevel.data[1].val.bind( function() { 
	debugPrint('CHANGED TO: ' + this.value + '\n'); 
});  
\end{lstlisting}

\section{HTTP Access}

The JavaScript implementation of Z-Way allows to directly accessing HTTP objects.

The http request is much like jQuery.ajax(): r = http.request(options);

Here's the list of options:
\begin{itemize}
\item url - required. Url you want to request (might be http, https, or maybe even ftp);
\item method – optional. HTTP method to use (currently one of GET, POST, HEAD). If not 
specified, GET is used;
\item headers – optional. Object containing additional headers to pass to server:

\begin{lstlisting}
headers: {
    "Content-Type": "text/xml",
    "X-Requested-With": "RaZberry/1.5.0"
}
\end{lstlisting}

\item data – used only for POST requests. Data to post to the server. May be either a
string (to post raw data) or an object with keys and values (will be serialized as 
'key1=value1\&key2=value2\&…');
\item auth – optional. Provides credentials for basic authentication. It is an object 
containing login and password:
\begin{lstlisting}
auth: {
    login: 'username',
    password: 'secret'
}
\end{lstlisting}
\item contentType – optional. Allows to override content type returned by server for 
parsing data (see below);
\item async – optional. Specifies whether request should be sent asynchronously. Default 
is false. In case of synchronous request result is returned immediately (as function 
return value), otherwise function exits immediately, and response is delivered later 
thru callbacks.
\item success, error and complete – optional, valid only for async requests. Success 
callback is called after successful request, error is called on failure, complete is 
called nevertheless (even if success/error callback produces exception, so it is like 
'finally' statement);
\end{itemize}

Response (as stated above) is delivered either as function return value, or as callback 
parameter. Is is always an object containing following members:

\begin{itemize}
\item status – HTTP status code (or -1 if some non-HTTP error occurred). Status codes 
from 200 to 299 are considered success;
\item statusText – status string;
\item URL – response URL (might differ from url requested in case of server redirects);
\item headers – object containing all the headers returned by server;
\item contentType – content type returned by server;
\item data – response data.
\end{itemize}


Response data is handled differently depending on content type (if contentType on request is set, it takes priority over server content type):
\begin{itemize}
\item application/json and text/x-json are returned as JSON object;
\item application/xml and text/xml are returned as XML object;
\item application/octet-stream is returned as binary ArrayBuffer;
\item string is returned otherwise.
\end{itemize}
In case data cannot be parsed as valid JSON/XML, it is still returned as string, and additional parseError member is present.


\begin{lstlisting}
http.request({
	url: "http://server.com" (string, required),
	method: "GET" (GET/POST/HEAD, optional, default "GET"),
	
	headers: (object, optional)
	{
		"name": "value",
		...
	},
	
	auth: (object, optional)
	{
		"login": "xxx" (string, required),
		"password": "***" (string, required)
	},
	
	data: (object, optional, for POST only)
	{
		"name": "value",
		...
	}
	-- OR --
	data: "name=value&..." (string, optional, for POST only),

	async: true (boolean, optional, default false),
	
	success: function(rsp) {} (function, optional, for async only),
	error: function(rsp) {} (function, optional, for async only),
	complete: function(rsp) {} (function, optional, for async only)
});


response:
{
	status: 200 (integer, -1 for non-http errors),
	statusText: "OK" (string),
	url: "http://server.com" (string),
	contentType: "text/html" (string),
	headers: (object)
	{
		"name": "value"
	},
	data: result (object or string, depending on content type)
}
\end{lstlisting}

\section{XML parser}

ZXmlDocument object allows to convert any valid XML document into a JSON object and vice versa.

\subsection{var x = new ZXmlDocument()}
Create new empty XML document

\subsection{x = new ZXmlDocument("xml content")}
Create new XML document from a string

\subsection{x.root}
Get/set document root element. Elements are got/set in form of JS objects:

\begin{lstlisting}
{
    name: "node_name", – mandatory
    text: "value", – optional, for text nodes
    attributes: { – optional
    	name: "value",
    	...
    },
    children: [ – optional, should contain a valid object of same type
    	{ ... }
    ]
}
\end{lstlisting}

For example:
\begin{lstlisting}
(new ZXmlDocument('<weather><city id="1"><name>Zwickau</name><temp>2.6</temp></city><city id="2"><name>Moscow</name><temp>-23.4</temp></city></weather>')).root =
{  
   "children":[  
      {  
         "children":[  
            {  
               "text":"Zwickau",
               "name":"name"
            },
            {  
               "text":"2.6",
               "name":"temp"
            }
         ],
         "attributes":{  
            "id":"1"
         },
         "name":"city"
      },
      {  
         "children":[  
            {  
               "text":"Moscow",
               "name":"name"
            },
            {  
               "text":"-23.4",
               "name":"temp"
            }
         ],
         "attributes":{  
            "id":"2"
         },
         "name":"city"
      }
   ],
   "name":"weather"
}
\end{lstlisting}

\subsection{x.isXML}
This hidden readonly property allows to detect if object is XML object or not (it is always true).

\subsection{x.toString()}
Converts XML object into a string with valid XML content.

\subsection{x.findOne(XPathString)}
Returns first matching to XPathString element or null if not found.
\begin{lstlisting}
x.findOne('/weather/city[@id="2"]') // returns only city tag for Moscow
x.findOne('/weather/city[name="Moscow"]/temp/text()') // returns temperature in Moscow
\end{lstlisting}

\subsection{x.findAll(XPathString)}
Returns array of all matching to XPathString elements or empty array if not found.
\begin{lstlisting}
x.findAll('/weather/city') // returns all city tags
x.findAll('/weather/city/name/text()') // returns all city names
\end{lstlisting}

\subsection{XML elements}
Each XML element (tag) in addition to properties described above (text, attributes, children) have hidden readonly property parent pointing to parent object and the following methiods:
\begin{itemize}
\item insertChild(element) Insert new child eleemnt
\item removeChild(element) Remove child element
\item findOne(XPathString) Same as on root object, but relative (no leading / needed in XPathString
\item findAll(XPathString) Same as on root object, but relative (no leading / needed in XPathString
\end{itemize}

ZXmlDocument is returned from http.request() when content type is "application/xml", "text/xml" or any other ending with "+xml". Namespaces are not yet supported.

\section{Cryptographic functions}

crypto object provides access to some popular cryptographic functions such
as SHA1, SHA256, SHA512, MD5, HMAC, and provides good random numbers.

\subsection{var guid = crypto.guid()}
Provides standard GUID in string format.

\subsection{var rnd = crypto.random(n)}
Generates n random bytes.
Returned values is of type ArrayBuffer. To convert it into array use this trick:
\begin{lstlisting}
	rnd = (new Uint8Array(crypto.random(10)));
\end{lstlisting}

\subsection{var dgst = crypto.digest(hash, data, ...)}
Returns digest calculated using selected hash algorithm. It supports virtually all the algorithms available in OpenSSL (md4, md5, mdc2, sha, sha1, sha224, sha256, sha384, sha512, ripemd160).
If no data parameters specified, it returns a digest of an empty value. If more than one data parameters are specified, they're all used to calculate the result. Data parameters may be of different types (strings, arrays, ArrayBuffers).
Return value is of type ArrayBuffer.

There are also a few shortcut functions for popular algorithms: "md5", "sha1", "sha256", "sha512". For example, these calls are equivalent:

\begin{lstlisting}
	dgst = crypto.digest("sha256", data);
	dgst = crypto.sha256(data);
\end{lstlisting}

\subsection{var hmac = crypto.hmac(cipher, key, data, ...)}
Returns hmac calculated using selected hash algorithm. Hash algorithms are the same as for digest() function.
Key parameter is required. 
If no data parameters specified, it returns a HMAC of an empty value. If more than one data parameters are specified, they're all used to calculate the result. Key and data parameters may be of different types (strings, arrays, ArrayBuffers).
Return value is of type ArrayBuffer.

There are also a few shortcut functions for popular algorithms: "hmac256", "hmac512". For example, these calls are equivalent:

\begin{lstlisting}
  dgst = crypto.hmac("sha256", key, data);
  dgst = crypto.hmac256(key, data);
\end{lstlisting}

\section{Sockets functions}

Socket module allows easy access to TCP and UDP sockets from JavaScript.
Both connection to distant ports and listening on local are available. This API fully mirrors into JavaScript POSIX TCP/IP sockets.
This can be used to control third party devices like Global Cache or Sonos
as well as emulating third party services.

To start communications one need to create socket and either
\textbf{connect} it or \textbf{listen} it. \textbf{onrecv} method is called
on data receive from remote, while \textbf{send} is used to send data to remote side.

The example below dumps to log file response to http://ya.ru:80/ (raw HTTP
protocol is used as an example).

\begin{lstlisting}
var sock = new sockets.tcp();

sock.onrecv = function(data) {
    debugPrint(data.byteLength);
};

sock.connect("ya.ru", 80);

sock.send("GET / HTTP/1.0\r\n\r\n");
\end{lstlisting}

Here is an example of TCP echo server on port 8888:

\begin{lstlisting}
var sock = new sockets.tcp();

sock.bind(8888);

sock.onrecv = function(data) {
    this.send(data);
};

sock.listen();
\end{lstlisting}

And echo server for UDP:
\begin{lstlisting}
var sock = new sockets.udp();

sock.bind(8888);

sock.onrecv = function(data, host, port) {
    this.sendto(data, host, port);
};

sock.listen();
\end{lstlisting}

Detailed description of Socket API:
\begin{itemize}
\item bind(ip, port) or bind(port) binds socket to port (integer number). ip should be a string like "192.168.0.1". If omited "0.0.0.0" is used (bind on all IP addresses of all interfaces).
\item connect(ip, port) connects to remote side ip:port. TCP sockets requires this call before sending data. For UDP sockets it is optional, but once used allows to use send call instead of sendto call.
\item listen() starts listening port (this is required not only for TCP, but for UDP too).
\item close() initiate close of socket.
\item send(data) sends data to connected or accepted socket.
\item sendto(data, host, port) sends data to a non-connected UDP socket.
\item onrecv(data, host, port) called on new data receiption from remote side. For UDP sockets and connected TCP sockets "this" object reffers to the socket itself, while for accepted TCP sockets "this" reffers to the client's individual objects.
\item onconnect(host, port) called only for TCP sockets once connection to remote side is established or on new connection accept. For connected TCP sockets "this" object reffers to the socket itself, while for accepted TCP sockets "this" reffers to the client individual object.
\item onclose(host, port) called on socket close by remote or due to close() call. Note that for TCP sockets this callback is called for client sockets on connection close and for binded listening socket if close() is called. "this" object will be defined like in onrecv.
\end{itemize}

\section{Other JavaScript Extensions}

\subsubsection{fs.list(folder)}

This returns list of items in the folder or undefined if not folder is not existing.


\subsubsection{fs.stat(file)}

This returns one of the following values:

\begin{itemize}
\item 1) undefined if object does not exist or not readable
\item 2) object \{ type: 'file', size: \textless{}size\textgreater{}\} if it is a file
\item 3) object \{ type: 'dir' \} if it is a folder
\end{itemize} 


\subsubsection{fs.loadJSON(filename)}

This function reads a file from the file system and loads it into the memory. The file must contain a valid JSON object. The only argument is the name of the file including full pathname of the local file system. The functions returns the full JSON object or null in case of error.

\subsubsection{fs.load(filename)}

This function reads a file from the file system and returns it's content as a string. The only argument is the name of the file including full pathname of the local file system. The functions returns null in case of error.

\subsubsection{executeFile(filename) and executeJS(string)}

Loads and executes a particular JavaScript file from the local filesystem or executes JavaScript code represented in string (like eval in browsers).

The script is executed withig the global namespace.

Remark: If an error occurred during the execution it won't stop from further execution, but erroneous script will not be executed completely. It will stop on the first error.
Exceptions in the callee can be trapped in the caller using standard try-catch mechanism.

\subsubsection{system(command)}

The command system() allows to execute any shell level command available on the operating 
system. It will return the shell output of the command.  On default the execution of 
system commands is forbidden. Each command executed need to be permitted by putting one 
line with the starting commands in the file automation/.syscommands or in an different 
automation folder as specified in config.xml.

\subsubsection{Timers}
Timers are implemented exactly as they are used in browsers. They are very helpfull for periodical and delayed operations. Timeout/period is defined in milliseconds.
\begin{itemize}
\item timerId = setTimeout(function() { }, timeout)
\item timerId = setInterval(function() { }, period)
\item clearTimeout(timerId)
\item clearInterval(timerId)
\end{itemize}

\subsubsection{loadObject(object\_name) and saveObject(object\_name, object)}
Loads and saves JSON object from/to storage. These functions implements flat storage for application with access to the object by it's name. No folders are available.

Data is saved in automation/storage folder. Filenames are made from object names by stripping characters but [a-ZA-Z0-9] and adding checksum from original name (to avoid name conflicts).

\subsubsection{exit()}
Stops JavaScript engine and shuts down Z-Way server


\subsubsection{allowExternalAccess(handlerName) and listExternalAccess()}
allowExternalAccess allows to register HTTP handler. handlerName can contain strings like aaa.bbb.ccc.ddd - in that case any HTTP request starting by /aaa/bbb/ccc/ddd will be handled by a function aaa.bbb.ccc.ddd() if present, otherwise aaa.bbb.ccc(), ... up to aaa().
Handler should return object with at least properties status and body (one can also specify headers like it was in http.request module).

listExternalAccess returns array with names of all registered HTTP handlers.

Here is an example how to attach handlers for /what/timeisit and /what:

\begin{lstlisting}
what = function() {
  return { status: 500, body: 'What do you want to know' };
};

what.timeisit = function() {
  return { status: 200, body: (new Date()).toString() }
};

allowExternalAccess("what");
allowExternalAccess("what.timeisit");
\end{lstlisting}

\subsubsection{debugPrint(object, object, ...)}

Prints arguments converted to string to Z-Way console. Very usefull for debuggin.
For convenience one can map 'console.log()' to debugPrint().

This is how it was done in automation/main.js in Z-Way Home Automation engine:
\begin{lstlisting}
var console = {
    log: debugPrint,
    warn: debugPrint,
    error: debugPrint,
    debug: debugPrint,
    logJS: function() {
        var arr = [];
        for (var key in arguments)
            arr.push(JSON.stringify(arguments[key]));
        debugPrint(arr);
    }
};
\end{lstlisting}

\subsection{Debugging JavaScript code}
Change in config.xml debug-port to 8183 (or some other) turn on V8 debugger capability on Z-Way start.

\begin{lstlisting}
<config>
    ...
    <debug-port>8183</debug-port>
    ....
</config>
\end{lstlisting}

node-inspector debugger tool is required. It provides web-based UI for debugging similar to Google Chrome debug console.

You might want to run debugger tool on another machine (for example if it is not possible to install it on the same box as Z-Way is running on).

Use the following command to forward debugger port defined in config.xml to your local machine:
\begin{lstlisting}
ssh -N USER@IP_OF_Z-WAY_MACHINE -L 8183:127.0.0.1:8183
\end{lstlisting}
(for RaZberry USER is pi)

Install node-inspector debugger tool and run it:
\begin{lstlisting}
npm install -g node-inspector
node-inspector --debug-port 8183
\end{lstlisting}

Then you can connect to http://IP\_OF\_MACHINE\_WITH\_NODE\_INSPECTOR:8080/debug?port=8183

If debugging is turned on, Z-Way gives you 5 seconds during startup to reconnect debugger to Z-Way (refresh the page of debugger Web UI withing these 5 seconds).
This allows you to debug startup code of Z-Way JavaScript engine from the very first line of code.
 % JS
\chapter{The Automation Subsystem}
\label{automation}

The automation subsystem allows writing automation scripts using Javascript. It uses the 
ECMA compatible Javascript Engine described in chapter \ref{jsapi}.
All the code realizing the automation engine is written in Javascript itself and is 
available as open source for further study and modification.

The automation engine performs different actions based on events.  The actions are either 
signal commands or scripts that can add additional logic and conditions.
Events are  either generated 
from the Z-Wave network or from an outside sources such as the Internet or even from a 
user interaction is causing certain actions, either within the Z-Wave network (e.g. 
switching a light) or outside Z-Wave (e.g. sending a email). In Z-Way all automation 
is organized in so called modules. The subsequent manual will explain how these 
modules work and how to create own modules.

{\bf Attention: There is no Graphical User Interface for the automation engine at this 
moment in time. You will need a text editor such as joe, textwrangler or vi to edit 
certain files. }

\section{How to get to the automation engine}

The starting point for automation in Z-Way is called config.xml and is located in the main 
folder of Z-Way. The statement for the automation engine looks like

\begin{quote}
{\tt  
$<automation-dir>pathToAutomationCodeBase</automation-dir>$
}
\end{quote}
Assuming the automation is – like on default – in the subdirectory /automation the 
statement should look like

\begin{quote}
{\tt  
$<automation-dir>automation </automation-dir>$
}
\end{quote}

The automation folder consists of several files and subdirectories. The most important 
file of the automation is called config.json. This file contains the information 
about all automation modules and their instances. This file is automatically generated and 
should be changed without proper knowledge.


\section{The Event Bus}

All communication from and to the automation modules is handled by events. An event 
is a structure containing certain information that is exchanged using a central 
distribution place, {\bf the event bus}. This means that all modules can send events 
to the event bus and can listen to event in order to execute commands on them. All 
modules can 'see' all events but need to filter out their events of relevance.  The 
core objects of the automation are written in JS and they are available as source 
code in the sub folder 'classes':

\begin{itemize}
\item AutomationController.js: This is the main engine of the automation function
\item AutomationModule.js: the basic object for the module
\end{itemize}

The file main.js is the startup file for the automation system and it is loading the three 
classes just mentioned. The subfolder /lib contains the key JS script for the Event 
handling: eventemitter.js.

\subsection{Emitting events}

The 'Event emitter' emits events into the central event bus. The event emitter can 
be called from all modules and scripts of the automation system. The syntax is:

\begin{quote}
{\tt  
$controller.emit(eventName, args1,arg2,...argn)$
}
\end{quote}

The event name 'eventName' has to be noted in the form of 'XXX.YYY' where 'XXX' is the name 
of the event source (e.g. the name of the module issuing the event or the name of the 
module using the event) and 'YYY' is the name of the event itself.  To allow a scalable 
system it makes sense to name the events by the name of the module that is supposed to 
receive and to manage events. This simplifies the filtering of these events by the 
receiver module(s).

Certain event names are forbidden for general use because they are already used in the 
existing modules. One example are events with the name cron.XXXX that are used by the 
cron module handling all timer related events.

Every event can have a list of arguments developers can decide on. For the events used by 
preloaded modules (first and foremost the cron module) this argument structure is 
predefined. For all other modules the developer is free to decide on structure and content. 
It is also possible to have list fields and or any other structure as argument for the event

One example of an issued event can be 

\begin{quote}
{\tt  
$emit(“mymodule.testevent”,”Test”,[“event1”,”event2”])$
}
\end{quote}

\subsection{Catching (binding to) events}

The controller object, part of every module, offers a function called 'on()' to catch events. 
The 'on(name, function())' function subscribes to events of a certain name type. If not 
all  events of a certain name tag shall be processed a further filtering needs to be 
implemented  processing  the further arguments of the event. The function argument contains a reference 
to the implementation using the event to perform certain actions. The argument list of the event is 
handed over to this function in its order but need to be declared in the function call statement.

\begin{quote}
{\tt  
this.controller.on(“mymodule.testevent”, function (name,eventarray) {})
}
\end{quote}

The same way objects can unbind from events:

\begin{quote}
{\tt  
this.controller.off(“mymodule.testevent”, function (name,eventarray));
}
\end{quote}


\section{Module-Syntax}

Each module is located in a sub directory of the module-subfolder defined in the config.json file.
The name of the sub folder equals to the module name (not the instance of the module name!) 
and has at least two files:

\subsection{Module.json}

This file contains the module meta-definition used by the AutomationController. It must 
be a valid JSON object with the following fields (all of them are required):
\begin{itemize}
\item \textbf{autoload} — Boolean, defines will this module automatically instantiated during Home Automation startup.
\item \textbf{singleton} — Boolean, defines this module can be instantiated more than one time or not.
\item \textbf{defaults} — Object, default module instance settings. This object will be patched with the particular 
config object from the controller's configuration and resulting object will be passed to the initializer.
\item \textbf{actions} — Object, defines exported module instance actions. Object keys 
are the names of actions and  values are meta-definitions of exported actions used by 
AutomationController and API webserver.
\item \textbf{metrics} — Object, defines exported module metrics.
\end{itemize}
All configuration fields are required. Types of the object must be equal in every definition in every case. For instance, if module doesn't export any metric corresponding key value should be and empty object “{}”.

\subsection{index.js}
 
This script defines an automation module class which is descendant of AutomationModule base class.
During initialization the module script must define the variable '\_module' containing the particular module class.


Example of a minimal automation module:

\begin{lstlisting}[caption=Minimal Module]{MMIN}

function SampleModule (id, controller) {
SampleModule.super_.call.init(this, id, controller);

this.greeting = "Hello,World!";
}

inherits(SampleModule, AutomationModule);
_module = SampleModule;

SampleModule.prototype.init = function () {
    this.sayHello();
}

SampleModule.prototype.sayHello = function () {
    debugPrint(this.greeting);
} 

SampleModule.prototype.stop = function () {
    this.sayByeBye();
}    

\end{lstlisting} 
 
The first part of the code illustrates how to define a class function named SampleModule that calls the superclass' constructor. Its highly recommended not to do further instantiations in the constructur. Initializations should be implemented within the 'init' function.
 
The second part of the code is almost immutable for any module. It calls prototypal inheritance support routine and it fills in \_module variable.

The third part of the sample code defines module's init() method which is an 
instance initializer. This initializer must call the superclass's initializer prior to all other tasks. In the initializer module can setup it's private environment, subscribe to the events and do any other stuff.
Sometimes, whole module code can be placed withing the initializer without creation of any other class's methods. As the reference of such approach you can examine AutoOff module source code.

After the init function a module may contain other functions. The 'sayHello' function of the Sample Module shows this as example.

\section{Available Core Modules}


The automation engine already contains certain modules essential for the work of the whole system. Do now exclude these modules from the config.json and alter them only if you know exactly what you do.


\subsection{Cron, the timer module}

All time driven actions need a timer. The Z-Way automation engine implement a cron-type timer 
system as a module as well. The basic function of the cron module is

\begin{itemize}
\item It accepts registration of events that are triggered periodically
\item It allows to de-register such events. 
\end{itemize}


The registration and deregistration of events is also handled using the event mechanism. 
The cron module is listening for events with the tags 'cron.addTask' and  'cron.removeTask'. 
The first argument of these events are the name of the event fired by the cron module. The 
second argument of the 'addTask' event is an array desricing the times when this event shall be issued. It has the format: 
\begin{itemize}
\item Minute [start,stop, step] or 0-59 or null
\item Hour [start,stop, step] or 0-23 or null
\item weekDay [start,stop, step] or 0-6 or null
\item dayOfMonth [start,stop, step] or 1-31 or null
\item Month [start,stop, step] or 1-12 or null
\end{itemize}
The argument for the different time parameters has one of three formats
\begin{itemize}
\item null: the event will be fired on every minute or hour etc.
\item single value: the event will be fired when the value reaches the given value
\item array [start, stop, step]: The event will be fired between start and stop in steps.
\end{itemize}
 
 The object  
\begin{quote}
{\tt  $\{minute: null,hour: null,weekDay: null, day: null, month: null\} $   }
\end{quote} 

will fire every minute within every hour within every weekday on every day of the month every month. Another example of an event emitted towards the cron 
module for registering an timer event can be found in the Battery Polling Module:

\begin{lstlisting}[caption=Registering a Battery Polling Command]{SJSON3}
    this.controller.emit("cron.addTask", "batteryPolling.poll", {
        minute: 0,
        hour: 0,
        weekDay: this.config.launchWeekDay,
        day: null,
        month: null        
    });
\end{lstlisting}

This call will cause the cron module to emit an event at night (00:00) on a day 
that is defined in the configuration variable this.config.launchWeekDay, e.g. 0 = Sunday.

The 'cron.removeTask' only needs the name of the registered event to deregister.

\subsection{The Virtual Device Module}

This module generates virtual devices and manages them. For more information about
virtual devices and the use in a Graphical User Interface please refer to
chapter \ref {vdevapi} and \ref{vdev}.

\subsection{DeviceCollection module}

The Device Collection Module manages devices and shall not be changed. % Module

\chapter{The Z-Way HA User Manual}
\label{vdev}

You can access the User Interface 'z-way-ha' using the URL

\paragraph{http://YOURIP:8083/z-way-ha}
\paragraph{}

The following sections describes the 'Z-Way-HA' from the users point of view  

The Z-Way-HA User Interface is a AJAX based user interface available for web browsers. At 
the moment it supports Google Chrome, Firefox and Apple Safari only but no Microsoft Internet
Explorer.

The functions of the Z-Way-HA UI are:

\begin{itemize}
\item show all device functions of the Z-Way based Smart Home systems as widgets
\item allow to activate and manage automation modules that make use of the widgets
and may generate new widgets
\end{itemize}

The User Interface offers four function groups:

\begin{itemize}
\item \textbf{Dashboard}: Important widgets are shown in the dashboard. The section 'widgets'
in the 'Preferences' allow to define what widget is shown in the Dashboard.
\item \textbf{Widgets}: The widget section allows to access all widgets of the Home 
Automation System. They are grouped by 'Rooms', 'Type', and 'Tags' The 'Preferences'
allow to manage rooms, types and widgets and to assign certain widgets to these groups.
\item \textbf{Notifications}: Clocking om the notifications button opens a dialog showing all 
notification generated by the system and the modules. Notifications will stay in this list
until these are individually confirmed.
\item \textbf{Preferences}: The preferences tab opens a dialog with different setup options.
\begin{itemize}
\item \textbf{General}:  This allows to setup and manage profiles
\item \textbf{Rooms}:  This allows to setup and manage rooms
\item \textbf{Widgets}:  This allows managing widgets
\item \textbf{Automation}:  This allow managing the modules of the Javascript based 
automation engine
\end{itemize}
\end{itemize}

\section{Widgets}

The widget section allows managing all widgets that are automatically created from the 
device included into the Z-Wave or Third party Wireless Control System plus the widgets
generated from Automation Modules.\footnote{Technically also te widgets from the wireless 
control systems are generated by modules but this happens automatically when they registered
resp. included in the wireless system.}

A widget does not necessarily represent a physical device but a function of a device.
This means that one single device can create multiple widgets.
For Z-Wave devices every function (switch, battery, sensor value) and every channel in 
a multichannel environment generated a widget. The widget is not technology dependent but 
the initial name and the unique id of the generated widget is referring to the attributes 
of the physical  device. The pattern for the id is

\paragraph{ZWayVDev\_[Node ID]:[Instance ID]:[Command Class ID]:[Scale ID]}

\paragraph{}

The Node ID referred to the node ID of the physical device corresponding to thie widget, 
the instance is the instance or zero in case there are no multiple instances.
The command class ID refers to the command class generated the function of the widget.
Some command classes offer multiple sensor values differentiated by their scale id (e.g. 
Celsius or Fahrenheit). For command classes without multiple scales (e.g. battery value) 
this value is always zero.

The line below the main menu offers three options grouping the widgets:

\begin{itemize}
\item \textbf{by Room}:  The Ui can define rooms and in the room definitions widgets
can be assigned to rooms. Each widget can be assigned only to one room. The line below shows
all rooms currently defined. Clicking on the room shows all widgets assigned to this room. 
To manage rooms please refer to the section Preferences.
\item \textbf{by Type}:  All Widgets belong to one specific type. At the moment the following
types are defined and supported by the Z-Way-HA UI:
\begin{itemize}
\item \textbf{sensorBinary}: A binary sensor, only showing on or off
\item \textbf{sensorMultilevel}: The type, the value and the scale of the sensor are shown
\item \textbf{switchBinary}: The device can be switched on and off
\item \textbf{switchMultilevel}: The device can be switched on and off plus set to any
percentage level between 0 \% and 100 \%.
\item \textbf{switchRGBW}: This device allows setting RGB colors
\item \textbf{switchControl}:
\item \textbf{toggleButton}: The device can only be turned on. This is for scene activation.
\item \textbf{thermostat}: The thermostat shows the setpoint temperature plus a drop 
down list of thermostat modes if available
\item \textbf{battery}: The battery widget just shows the percentage of charging capacity left
\item \textbf{camera}: A camera will show the image and can be operated
\item \textbf{fan}: A fan can be turned on and off
\end{itemize}
The line below shows
all types where devices exist. Device types can not be managed on the UI but will be shown
automatically when new widgets are generated.
\item \textbf{By Tag}:  The system allows to generate user defined tags and assign 
these tags to defines. The only predefined tag is the 'dashboard'. This tag is used to 
select all widgets that are shown in the dashboard.
The line below shows all tags currently defined. Clicking on the tags shows all 
widgets assigned where this tag is assigned to. Tags can be freely defined when managing
a certain widget.
\end{itemize}

\section{Notifications}

Left beside the Notification button the number of notifications are shown. Clicking on 
the String 'Notification' opens a dialog box with the notification string. Notifications
are generated by the system (error message) or by application modules.
'Hide' deletes the message.

\section{Preferences}

Clicking on 'Preferences' opens a dialog with four sub menus as shown in Figure 
\ref{ha_prefs}. One the upper 
side there is a 'X' to close the dialog. Once a sub menu is opened on the 
upper left side the $'<'$ returns to the sub menu overview.

\begin{figure} 
\begin{center}
\includegraphics[width=0.8\textwidth]{pics/ha_preferences.png}
\caption{Preferences Submenu}
\label{ha_prefs}
\end{center} 
\end{figure}


\subsection{General}

The dialog shows different profiles. At the moment there are no further options 
and actions available except naming the profiles and giving a Name.

On the lower left corner there is a '-' and a '+'. Clicking on the '+' adds a new
profile, '-' deletes the profile highlighted on the left hand side. A filter can 
be applied to find certain profiles.

On factory default only the 'default' profile is available.

\subsection{Rooms}

The dialog 'Rooms' offers four submenus. On the lower left corner there is a '-' and 
a '+'. Clicking on the '+' adds a new room, '-' deletes the room highlighted on the left 
hand side. A filter can be applied to find certain rooms.

\paragraph{General}

Clicking on the 'Edit'-Button allows editing the Room setup.

The name can be chosen by the user. Clicking on the icon image opens a file chooser 
dialog to pick a new icon. Clicking on the device button opens a new dialog where 
devices can be assigned to this room. The left hand side shows the device available (not 
assigned ot any room). Picking the device is done by 'drag and drop'.
Devices on the right hand side are assigned to the room.

\paragraph{Temperature Preference}

This dialog is not used at the moment.

\paragraph{Auto Mode}

This dialog is not used at the moment.

\paragraph{Devices}

This is a short cut to the device assignment dialog also accessible in the 'General'
sub menu.

\subsection{Widgets}

The widgets menu allows managing the widgets generated by the 
Home Automation system. All widget are listed with their names on the left hand side.
A filter can be applied to find certain widgets.

The name of the widget is auto-generated by the system but can be changed. The id of the 
widget is unique and shown below the name entry.
The device type is shown.

The tag section allows to add new tags defined by the user. 
Defined tags are listed and can be deleted clicking on the 'x' symbol.

A checkbox defined if the widget is shown on the dash board.


\subsection{Automation}

This section allows managing the automation modules of the Automation system.
On the left hand side there is a list of all defined module instances. Please refer 
to the automation section \ref{automation} for the relationship between module 
definitions and instances of modules.

On the lower left corner there is a '-' and 
a '+'. Clicking on the '+' adds a new module instance, '-' deletes the module instance 
highlighted on the left  hand side. A filter can be applied to find certain module
instances.

Clicking on the module instances name or the '+' button opens a dialog menu on the 
right hand side. When a new module instance is created the first option is to pick
one of the modules available.

The list of modules consist of a set of modules that are scope of delivery of the system.
Additionally the list will show all modules defined by the users.

All setups of modules consist of a module specific part (upper part) and a generic 
part (lower part), that allows to 

\begin{itemize}
\item  define a name
\item  define a description of the instance
\item  enable or disable the module
\end{itemize}

Important predefined modules are:

\paragraph{Bind Devices: } This modules implements the association function 
known from Z-Wave. Compared to the Associations in Z-Wave that are stored
in the devices the bind function settings are stored in the gateway but can
bind devices of different wireless technologies or widgets created.

\paragraph{Load custom JS Code: } This allows to activate an own piece of 
Javascript.

\paragraph{Load custom JS File: } This allows to activate an own piece of 
Javascript loaded from a file.

\paragraph{Group devices: } Groups several devices together and adds a 
new widget.

\paragraph{Sensor Polling} Z-Way itself will not poll devices 
but rely on unsolicited status updates to keep the UI information updated.
Certain sensor values should be updated periodically.  If different
sensors shall be polled in different intervals (e.g. meter less frequently 
than actual power draw) multiple instances of the modules need to be defined.

\paragraph{Sensors Values Logging: } This modules allows to report sensor values 
to a cloud service. The values are either written into a JSON file or  sent over the 
Internet. In this case a receiving URL can be defined.

\paragraph{Trap events from Remote And Sensors: } Generates new widgets on the fly for 
Remote Switch Controls and other devices sending control commands to 
controller.

Other modules are

\begin{itemize}
\item \textbf{Always On}: keeps a device on regardless of switching command
\item \textbf{Auto Off}: turns off a device after defined time
\item \textbf{BatteryPolling}: polls all battery devices for charging level
\item \textbf{Camera}: controls a IP camera
\item \textbf{LightScene}: defines light scenes
\item \textbf{NotificationSMSru}: allows to sen SMS notifications
\item \textbf{OpenRemoteHelpers}: Adapts to the solution from openremote.org
\item \textbf{OpenWeather}: polls weather data from Internet
\item \textbf{RGB}: Creates RGB device based on three different dimmers
\item \textbf{RoundRobinScenes}: Activates scene in Round Robin
\item \textbf{SecurityNotifications}: Notify on changes of sensors and switches state
\item \textbf{SwitchControlGenerator}: Generates new widgets on the fly for Remote Switch 
Controls and other devices sending control commands to controller
\item \textbf{ZWaveGate}: creates virtual devices from Z-Wave Devices
\end{itemize}

\section{Dashboard}

The dashboard shows all widgets selected as 'in dashboardÄ in the widget 
dialog of the Preferences section.
 % Z-Way-Ha
\chapter{Virtual Device API (vDev)}
\label{vdevapi}


The functions of the 'Z-Way HA' User Interface are described in the chapter \ref{vdev}. 
It  is based on so called virtual devices plus some other supporting functions
like rooms, notifications, tagging etc. The interface providing all these functions is 
called Virtual Device API (vDev).

The intention of this vDev API  is to further simplify the use of Z-Way by AJAX based 
User interface implementations and to unify the user experience across
different wireless technologies.
  
The vDev API has the following objectives and functions.

\begin{itemize}
\item Provide a list of devices that are independent of the physical devices of a 
given wireless network With this devices of different networks appear similarly.
\item Show all functions of a physical device as one virtual device each. This simplifies
the use of the data and functions provides. The Z-Wave Device API still required 
a lot of domain knowledge about the different devices, their instances, command classes, 
scales. In the vDev API every function is represented by one virtual device only 
allowing simple loops for display.
\item Provide end user related context information. This allows e.g. to define profiles, 
rooms, application etc.
\item Handle events in an user friendly way to that they can be used a notifications on a UI.
\end{itemize}

There is an online manual for syntax of the various functions that can be accessed on 

\paragraph{http://docs.zwayhomeautomation.apiary.io/}
 

\section{The virtual device}

The virtual device in the vDev API is an object that had properties and offers functions. 
Both properties, variables and functions are unified and their syntax is independent of 
the physical nature of the device they are are referred to (If there is any).

\subsection{Types and Ids}

Every Virtual device is identified by a simple string type id. For all virtual devices 
that are related to physical Z-Wave devices the device name is auto-generated by the 
ZWaveGate module following this logic:


\subsection{Virtual Device Ids}

Auto-generated devices are named after their IDs in the physical network. For Z-Wave 
devices the naming is generated using the following logic.

\paragraph{ZWayVDev\_[Node ID]:[Instance ID]:[Command Class ID]:[Scale ID]}

The Node Id is the node id of the physical device, the Instance ID is the instance id 
of the device  or '0' if there is only one instance. The command class ID refers to the 
command class the function is embedded in. The scale id is usually '0' unless the virtual
device is generated from a Z-Wave device that supports multiple sensors with different 
scales in one single command class.

\subsection{Virtual Device Type}

Virtual devices can have a certain types. The type of the device can be chosen. For a list 
of the device types current supported in the Z-Way-HA API please refer to chapter
\ref{vdev}.
 
\subsection{Access to Virtual Devices}

Virtual devices can be access both on the server side using JS modules and on the client 
side using the JSON API. On the client they are encoded into a URL style for easier 
handling  in AJAX code. A typical client side command in the vDev API looks like

\paragraph{http://YOURIP:8083/ZAutomation/api/v1/devices/ZWayVDev\_6:0:37/command/off}

'api' points to the vDev API function, 'v1' is just a constant to allow future extensions. 
The devices are referred by a name that is automatically generated from the Z-Wave 
Device API. The vDev also unifies the commands 'command' and the parameters, here 'off'.

On the server side the very same command would be encoded in a JavaScript style.

\begin{lstlisting}[caption=Access vDevs]{}

vdevId = vdev.id;

vDev = this.controller.devices.get(vdevId);

vDevList = this.controller.devices.filter(function(x) { 
	return x.get("deviceType") === "switchBinary"; }); 

vDevTypes = this.controller.devices.map(function(x) { 
	return x.get("deviceType"); }); 
\end{lstlisting}

\subsection{Virtual Device Usage / Commands}

In case the virtual device is an actor it will accept and execute a command using the 
syntax:

\paragraph{Vdev.performCommand(„name of the command“)}

The name of the accepted command should depend on the device type and can again be defined 
free of restrictions when implementing the virtual device. For auto-generated devices 
derived from Z-Wave the following commands are typically implemented.

\begin{enumerate}
\item 'update': updates a sensor value
\item 'on': turns a device on.  Only valid for binary commands
\item 'off': turns a device off. Only valid for binary commands
\item 'exact': sets the device to an exact value. This will be a temperature for 
thermostats or a percentage value of motor controls or dimmers
\end{enumerate}

\subsection{Virtual Device Usage / Values}

Virtual devices have inner values. They are called metrics. A metric can be set and get. 
Each virtual device can define its own metrics. Metrics can be level, title icon and 
other device specific values like scale (%, kWh, ...)

\begin{lstlisting}
vDev.set("metrics:...", ...);  
vDev.get("metrics:...");
\end{lstlisting}


\subsection{How to create your own virtual devices}

Virtual devices can be created using modules or Javascript code in the browser itself. 
The following code sample demonstarte how to create and delete a virtual device. For more 
information about the module concept and the creating of modules and virtual device within 
modules please refer to chapter \ref{modules}.

\subsubsection{Register device}

\begin{lstlisting}[caption=Register Device]{Name} 
        vDev = this.controller.devices.create(vDevId, {
            deviceType: "deviceType",
            metrics: {
                level: "level",
                icon: "icon from lib or url"
                title: "Default title"
            }
        }, function (command, ...) {
                // handles actions with the widget
        });  
\end{lstlisting}

\subsubsection{Unregister device}

Devices can be deleted or unregistered  using the following command:

\paragraph{this.controller.devices.remove(vDevId)}

\subsubsection{Binding to metric changes}

The metric - the inner variables of the vDev a changed by the system automatically.
In order to perform certain functions on these changes the function needs to be 
bound to the change to the vdev. The syntax for this is

\paragraph{vDev.on('change:metrics:...", function (vDev) { ... });}

Unbinding then works as one can expect:

\paragraph{vDev.off(’change:metrics:...”, function (vDev) ... )} 


\section{Notifications and events}

Notifications are a special 
kind of event to inform the user on the GUI. This means that normale events are typically
describes with numbers or ids while notifications contain a human readable message. 
The creating of events and the reaction on events is describes in chapter \ref{automation}
 
The UI can be notified on the certain events.

\paragraph{this.controller.addNotification("....severity....", "....message....", "....origin....");} 

The parameters define
\begin{itemize}
\item severity is error, info, debug; 
\item origin describes which part of the system it is about: core, module, device, battery.
\end{itemize}

The controller can act on  notifications or disable them.

\paragraph{this.controller.on('notifications.push', this.handler);}
\paragraph{this.controller.off('notifications.push', this.handler);}



 

% fs.list(),fs.stat()

 

 % vDev API 

\chapter{Z-Way Data Model Reference}
\label{datamodel}

\section{zway}

\begin {itemize}
\item Description: zway is the Z-Way part of the object tree
\item Syntax:  zway.X with  X as child object
\item Child objects
\begin {itemize}
\item controller: controller object, see below for details
\item devices: devices list, see below for details
\item version: Z-Way.JS version
\item isRunning(): Check if Z-Way is running
\item isIdle(): Check if Z-Way is idle (no pending packets to send)
\item discover(): Start Z-Way discovery process
\item stop() : Stop Z-Way
\item InspectQueue() : Returns list of pending jobs in the queue.
\begin {itemize}
\item item: [timeout, flags, nodeId, description, progress, payload]
\item flags: [send count, wait wakeup, wait security, done, wait ACK, got ACK, wait response, got response, wait callback, got callback]
\end {itemize}
\item ProcessPendingCallbacks(): Process pending callbacks (result of setTimeout/setInterval or functions called via HTTP JSON API)
\item bind(function, bitmask): Bind function to be called on change of devices list/instances list/command classes list
\item unbind(function) : Unbind function previously bind with bind()
\item all function classes in \ref{FunctionClasses} are also methods of this data object
\end {itemize}
\end {itemize}

\section{controller}

You can access the data elements of "controller" in the demo user interface in menu 
"for experts + Controller Info"

\begin {itemize}
\item Description: Controller object
\item Syntax: controller.X with  X as child object
\item Child objects
\begin {itemize}
\item data: Data tree of the controller
\begin {itemize}
\item  APIVersion: Version of the Serial API
\item  SDK: System development kit version of the Transceiver firmware
\item  SISPresent: flase if SUIS is available
\item  SUCNodeId: Node ID of SUC if present
\item  ZWVersion: ZWave Version (firmware)
\item  ZWaveChip: The name of the Z-Wave transceiver chip
\item  ZWlibMajor / ZWlibMinor: library version 
\item  capabilities: array of function class ids
\item  controllerstate: flag to show inclusion mode etc
\item  countJobs: shall job be counted
\item  curSerialAPIAckTimeout10ms: timing parameter of serial interface
\item  curSerialAPIBytetimeout10ms: timing parameter of serial interface
\item  homeId:the home id of the controller
\item  isinOtherNetworks: flag to show if controller is real primary if in other network
\item  isPrimary: flag to show if controller is primary
\item  isRealprimary: flag to show if controller can be  primary
\item  isSUC: is SUC present
\item  lastExcludedDevice: node ID of last excluded device
\item  lastIncludedDevice: node ID of last included device
\item  libType library basis type
\item  manufacturerIS / manufacturerProductId / manufacturerProductTypeId: ids to identify the transceiver hardware
\item  memoryGetAddress
\item  memoryGetData
\item  nodeId: own node ID
\item  nonManagementJobs: number of non man. jobs
\item  oldSerialAPIAckTimeout10ms: default timing parameter of serial interface
\item  oldSerialAPIBytetimeout10ms: default timing parameter of serial interface
\item  softwareRevisonDate: written by compiler
\item  softwareRevisionID: written by compiler
\item  vendor: string of hardware vendor
\end {itemize}
\item AddNodeToNetwork(mode): Reference to zway.AddNodeToNetwork()
\item RemoveNodeFromNetwork(mode): Reference to zway.RemoveNodeFromNetwork()
\item ControllerChange(mode): Reference to zway.ControllerChange()
\item GetSUCNodeId(mode): Reference to zway.GetSUCNodeId()
\item SetSUCNodeId(nodeId): Assign SUC role to a device
\item SetSISNodeId(nodeId): Assign SIS role to a device
\item DisableSUCNodeId(nodeId): Revoke SUC/SIS role from a device
\item SendNodeInformation(nodeId): Reference to zway.SendNodeInformation()
\end {itemize}
\end {itemize}


\section{Devices}

The devices object contains the array of the device objects. Each device in the network - including the 
controller itself -  has a device object in Z-Way.

\begin {itemize}
\item Description: list of devices
\item Syntax:  X with  X as child object
\item Child objects
\begin {itemize}
\item [m]: Device object
\item length: Length of the list
\item SaveData(): Save Z-Way Z-Wave data for hot start on next run (in config/zddx/HOMEID-DevicesData.xml)
\end {itemize}
\end {itemize}
 

\section{Device}

The data object can be accesses in the demo UI in advanced mode of "Configuration"

\begin {itemize}
\item Description: the device object
\item Syntax:  device[n].X with  X as child object
\item Child objects
\begin {itemize}
\item id: (node) Id of the device
\item Data: Data tree of the device
\begin {itemize}
\item SDK: SDK used in the device
\item ZDDXMLFile: file of the Devcie Description Record
\item ZWLib: Z-Wave library used
\item ZWProtocolMajor / ZWProtocolMinor: Z-Wave protocol version
\item applicationMajor / ApplicationMinor: Application Version of devices firmware
\item basicType: basic Z-Wave device class
\item beam: wake up beam required
\item countFailed: statistics of failed packets sent (from start of process)
\item countSuccess: statistics of sucessful packets sent (from start of process)
\item deviceTypeString:
\item genericType: generic Z-Wave device class
\item infoProtocolSpecific
\item isAwake
\item isListening
\item isFailed
\item isRouting
\item isVirtual: 
\item keepAwake: flag is device need to be kept awake
\item lastRecevied:  timestamp of last packet received
\item lastSend:  timestamp of last sent operation
\item ManufacturerId / manufacturerProductID / manufacturerProductTypeId: ids used to identify the device
\item neightbours: list of neighbour nodes
\item nodeInfoFrame: nodeinformation frame in bytes
\item option: flag if optional command classes are present
\item queueLength: length of device specific send queue
\item sensor1000: flag if device is FLIRS with 1000 ms wakeup 
\item sensor250: flag if device is FLIRS with 250 ms wakeup 
\item specificType: specific Z-Wave device class
\end {itemize}
\item instances: Instances list of the device
\item RequestNodeInformation(): Request NIF
\item RequestNodeNeighbourUpdate(): Request routes update
\item InterviewForce(): Purge all command classes and start interview based on device's NIF
\item RemoveFailedNode(): Remove this node as failed. Device should be marked as failed to remove it with this function.
\item SendNoOperation(): Ping the device with empty packet
\item LoadXMLFile(file): Load new Z-Wave Device Description XML file. See http://pepper1.net/zwavedb/
\item GuestXML(): Return the list of all known Z-Wave Device Description XML files with match score. [score, file name, brand name, product name, photo]
\item WakeupQueue(): Pretend the device is awake and try to send packets
\item AssignReturnRoute(target): Send device new routes to target node
\item DeleteReturnRoute(): Clear routes in device
\item AssignSUCReturnRoute(): Inform device about SUC and route to reach it
\end {itemize}
\end {itemize}

\section{Instances}

Each device may have multiple instances (similar functions like switches, same type sensors, ...) If only one instance 
is present the id of this instance is 0. Command classes are located in instances only-

\begin {itemize}
\item Description: list of instances
\item Syntax:  device[n].instance[m].X with  X as child object
\item Child objects
\begin {itemize}
\item [m]: instance object
\item length: Length of the list
\item commandClasses: list of command classes of this instance. In case there is only one instance, this is equivalent to the list of command classes of the device. For details see below.
\item Data: data object of instance 
\begin {itemize}
\item dynamic: flag if instance is dynamic
\item genericType: generic Z-Wave device class of instance
\item specificType: specific Z-Wave device class of instance
\end {itemize}
\end {itemize}
\end {itemize}

\subsection{CommandClass}

This is the command class object. It contains public methods and public data elements that are described
in chapter \ref{ccs}

\begin {itemize}
\item Description: Command Class Implementation
\item Syntax:  device[n].instance[m].commandclass[id].X with  X as child object
\item Child objects
\begin {itemize}
\item id: Id of the Command Class of the instance of the device
\item data: Data tree of the Command Class
\begin {itemize}
\item interviewCounter: number of attempts left until interview is terminate even if not successful
\item interviewDone: flag if interview of the command class is finished
\item security: flag if command class is operated under special security mode
\item version: version of the command class implemented
\item supported: flag if CC is supported or only controlled
\item {commandclass data}: Command Class specific data - see chapter \ref{ccs} for details.
\end {itemize}
\item name: Command Class name
\item {Method}: Command Class method - see chapter \ref{ccs} for details.
\end {itemize}
\end {itemize}

\section{Data}

This is the description of the data object.

General note: Z-Way objects and it's decendents are NOT simple JS objects, but native JS objects, 
that does not allow object modification.

\begin {itemize}
\item name: Name of data tree element
\item updated: Update time
\item invalidated: Invalidate time
\item valueOf(): Returns value of the object (can be omitted to get object value)
\item invalidate(): Invalidate data value (mark is as not valid anymore)
\item bind(function (type[, arg]) {...}, [arg, [watchChildren=false]]): Bind function to a change of data tree element of its descendants
\item unbind(function): Unbind function bind previously with bind()
\end {itemize}

 

\chapter{Command Class Reference}
\label{ccs}


Command Classes are groups of wireless commands that allow using certain functions of a Z-Wave device.  
In Z-Way each Z-Wave device has a data holder entry for each Command Class supported. During the inclusion 
and interview of the device the Command Class structure is instantiated in the data holder and filled with 
certain data. Command Class commands change values of the corresponding data holder structure. The follow 
list shows the public commands of the Command Classes supported with their parameters and the data holder 
objects changed.

In Expert UI navigate to Expert tab under Configuration page of a device to execute commands of the supported Command Classes and visualizes all data holder elements in as tree in a simplified UI.
 
\chapter{Function Class Reference}
\label{FunctionClasses}
 



\end{document} 
