\chapter{Extending the systems beyond \zwave}
\label{extensions}

\section{IP-Cameras}

IP cameras transmit a video stream that can usually be accessed having dedicated mobile 
apps. Under certain circumstances it is possible to have the very same video stream in 
parallel within the \zwshui.
All \zway user interfaces (Web Browser and native apps for IOS and Android) are based on 
off-the-shelf HTML rendering engines supporting standard video and image formats. To 
display the video stream of a certain camera this stream must comply to commonly used 
public standards, such as \textbf{MJPEG}.

Certain cameras however use \textbf{proprietary video encoding} that can only be decoded 
by special native mobile apps of the manufacturers. These cameras can’t be supported by \zway.

\subsection{How to find out if a camera is supported by \zway?}

\begin{enumerate}
\item Check the manual if MJPEG is mentioned as encoding method for the video stream.
\item Check if there is a way to access the video stream using a standard web browser 
such as Google Chrome or Microsoft Internet Explorer.
\end{enumerate}

\subsection{How to prepare for integration?}

To integrate a camera into \zway, this camera needs to be setup first following the 
guidelines given in the manufacturers manual. As a result, there must be

\begin{itemize}
\item A login name (examples are “admin” or “user”)
\item A password for access (this usually needs to be setup
\item The IP address of the camera
\end{itemize}

\subsection{How to find the IP address of the camera?}

Most IP networks in private homes and offices assign IP devices a new IP address using 
the DHCP protocol. The router holds a list of IP addresses and will arbitrarily choose 
one address from the list to the new IP device. Even after a reboot this address will 
remain the same. If the setup process mentioned above does not reveal the IP address 
there are two common ways:
\begin{itemize}
\item Log into your IP routers user interface. Usually this interface will display
 all IP addresses assigned together with a name and/or type of the device.
\item Use an IP address scanner on your PC or notebook that will tell you all IP 
devices active in a network plus the type of IP device they ae assigned to. A valuable 
tool for this is called ``Angry IP Scanner’’ available for all PC platforms such as
 MAC OSX, Linux or Windows. It requires a Java Virtual Machine (JVM).
\end{itemize}

\subsection{How to integrate the camera into \zway?}


\begin{figure}
\begin{center}
\includegraphics[width=0.7\textwidth]{pngs/cap9/camera1.png}
\caption{Inclusion of predefined cameras}
\label{camera1}
\end{center}
\end{figure}

The \zwshui allows adding new devices. Log into the user interface of \zway, 
click on the setup menu (icon on the upper right side) and click on menu item 
\keystroke{Add New Device}. You see the dialog as shown in Figure \ref{camera1}. Now 
choose the IP camera symbol line and click \keystroke{+}.

\begin{figure}
\begin{center}
\includegraphics[width=0.7\textwidth]{pngs/cap9/camera2.png}
\caption{Generic camera module}
\label{camera2}
\end{center}
\end{figure}

Now you will find a list of IP camera types plus one generic IP camera option 
called \app{Web Camera}. This dialog is shown in Figure \ref{camera2}.

If you are lucky, the name of your camera is already on the list. If not, you can 
check in the app store if there is a new support app available for your camera type. 
Go to \menu{Setup > App > Online Apps} and filter for \app{Video Surveillance.} 
Figure \ref{camera3} shows the app store with camera filter.

\begin{figure}
\begin{center}
\includegraphics[width=0.7\textwidth]{pngs/cap9/camera3.png}
\caption{More camera support in App Store}
\label{camera3}
\end{center}
\end{figure}

If you find your camera type, just install the app and redo the steps above. Now you will 
find the new camera in the list to choose from.

Once you click on the camera of your choice there will be a setup wizard asking you as a 
minimum for the IP address of your camera, the login name, and the password. Therefore, 
you need this set of data from the setup process of your device.

\subsection{How to support a camera not on the list yet?}

If your camera type is neither on the list of preset cameras nor there is a new app in 
the online app store there is still a very good chance to get your camera integrated. 
However now there is more work needed to find the right commands controlling your camera. 
In this case, you will need to choose the generic camera type ``Web camera,’’ which requires 
the same set of information (IP address, login, password) but more than that.

\begin{figure}
\begin{center}
\includegraphics[width=0.7\textwidth]{pngs/cap9/camera4.png}
\caption{Web browser debug interface}
\label{camera4}
\end{center}
\end{figure}

{\em Note: This work requires some basic understanding of web pages, IP, and URLs. The 
generic web camera allows defining the URL to the video stream and---if the camera has 
these capabilities---links to tilt, turn, night vision control, etc.}

To find these URLs, you need to log into your camera using a generic web browser. We 
strongly recommend using Google Chrome because of the debugging capabilities. The 
following explanation assumes the Chrome browser, but other browser will have similar functions.

\begin{enumerate}
\item Open the JavaScript Debug console. You find this option on the browsers menu under
\menu{View > Developer}. Figure \ref{camera4} shows the web browser with debugger active.
\item Once the debugger is open pick the menu item \menu{network} of the debugger (see image below”
\item Now you use the cameras web interface for accessing the image/stream, tilting, moving, 
etc. Whenever you do this the URL needed will be sent from the web interface to the camera 
and becomes visible in the debugger. Take these URLs and copy them into the setup 
interface of \app{Web Camera}. The camera control in \zwave will call the same URL for 
control.
\end{enumerate}

\section {433 MHz devices}

\subsection {Introduction}

433 MHz wireless communication is an outdated wireless technology. It is single direction 
wireless communication only without confirmation of received packets and it does not have 
any security functions. It is not standardized and subject to jamming by other devices 
since 433 MHz is a non-regulated frequency range. However, it is still used is many low-end 
alarm and control systems and there is a quite large install base in the market. The 
biggest advantage of 433 MHz devices is its low price.

Due to its one-way wireless connection, sensors can only report values and actuators can 
only receive values. Unfortunately, there is no regulation in the frequency band. 
Every supplier has its own code set for actuators and sensors.

Typical suppliers of 433 MHz devices are Intertechno, Conecto, Mumbi, Homeeasy, Elro, 
Teldus, Brennenstuhl, Olympia and others.

\subsection {433 MHz Gateway}

In order to support 433 MHz devices special gateways are required. \zway has built-in support 
for the 433 MHz gateway from Popp. Figure \ref{gwhardware} shows this gateway hardware. 
The device is powered by an external standardized mini USB port and can be connected ot 
USB wall outlets, mini USB power supplies, etc.

\begin{figure}
\begin{center}
\includegraphics[width=0.5\textwidth]{pngs/cap9/433gateway.png}
\caption{Popp 433 MHz Gateway}
\label{gwhardware}
\end{center}
\end{figure}

Once configured and connected to the \zway System, this gateway allows to learn different 
code sets of different manufacturers. It can therefore be used universally for all kinds 
of 433 MHz devices from different manufacturers even if there is no detailed technical 
description of the code set used.

\zway can support multiple Popp 433 MHz gateways when one single gateway cannot cover 
the whole home. The Popp 433 MHz hub is connected to \zway using Wi-Fi. This means that 
there must be at least one Wi-Fi network to connect the 433 MHz and an IP connection to 
the \zway system (not necessarily Wi-Fi but cable Ethernet is possible too if there is 
a router between the cabled ethernet and the Wi-Fi).

\subsection {How to setup the 433 MHz Gateway}

First, install the app \app{RF433} from the online app store. For more information about the 
online app store, please refer to Chapter \ref{apps}.

\begin{figure}
\begin{center}
\includegraphics[width=0.9\textwidth]{pngs/cap9/433_1.png}
\caption{RF433 App Setup}
\label{433_1}
\end{center}
\end{figure}

Activate the RF433 app as shown in Figure \ref{433_1}. If you like you can change the 
operating IP from 8000 to any other IP port number available. However, its perfectly fine 
to keep it at 8000. Once the RF433 app is running on your \zway system you need to setup 
the 433 MHz gateway.

\subsubsection {Power Up the 433 MHz Gateway}

Place the 433 MHz connector on the place of choice and power it using a standard 5V USB 
power supply. Push the central button until the LED slowly blinks in purple indicating that 
the device is in configuration mode serving its own access point.

In this mode, the gateway acts as access point creating its own Wi-Fi network. The SSID of 
this Wi-Fi network is \textbf{gw433-xxx} with xx as some individual serial code.

You need to connect to this Wi-Fi network using any Wi-Fi capable device available (e.g. 
mobile phone, notebook, etc.). Now start a web browser on this device and open the page 

\murl{http://192.168.4.1 }

to access the configuration interface as shown in Figure \ref{433_2}. 
There is no password needed.

\begin{figure}
\begin{center}
\includegraphics[width=0.4\textwidth]{pngs/cap9/433_2.png}
\caption{433 MHz gateway web interface}
\label{433_2}
\end{center}
\end{figure}

\subsubsection {Configure the 433 MHz Gateway}

Click on \keystroke{Configuration} to access the setup dialog as shown in figure \ref{433_3}.

\begin{figure}
\begin{center}
\includegraphics[width=0.4\textwidth]{pngs/cap9/433_3.png}
\caption{433 MHz gateway setup dialog}
\label{433_3}
\end{center}
\end{figure}

\begin{itemize}
\item \textbf{Wifi-Settings}: Choose the SSID of your Homes Wi-Fi. This is the Wi-Fi that 
establishes the IP connection between the 433 MHz Gateway and the \zway controller.
You can scan for available SSIDs. In case the Wi-Fi network selected requires a password 
(or WPA key) enter this key as well. \item \textbf{DHCP Active}: In most case the Wi-Fi 
you connect to will run DHCP for automated IP address assignment. Only if you know for 
sure that there is no DHCP you need to setup a fixed IP address plus network mark plus 
gateway.
\item \textbf{Fixed IP Settings}: This setup is only needed if there is no DHCP service 
on the Wi-Fi selected.
\item \textbf{Hub Server/Port}: Here you need to configure the IP address of your 
\zway controller and the IP port address chosen during the setup of the RF433 app on \zway.
\end{itemize}

Activate the connection mode of the Gateway. The local Access point will be deactivated 
and the gateway connects (1) to the Wi-Fi and through the Wi-Fi to the (2) \zway controller. 
Success is indicated by blinking of the green LED.

You can always return to the configuration mode by a long push of the central button on 
the 433 MHz gateway. In this case any connection to the \zway controller is deactivated 
and the local Wi-Fi with SSID \textbf{gw433-xxx } is active again.
Here again the different LED codes of the Popp 433 MHz gateway:

\begin{itemize}
\item blue blinking: Configured but searching for connection to \zway controller
\item purple blinking: Gateway is in initial configuration mode. Connect to Wi-Fi APN gw433-XXXX and call URL http://192.168.4.1 for initial configuration
\item green permanent: connection established
\item green short blink: communication from/to gateway
\end{itemize}

\subsubsection {Teach In 433 MHz devices}

Once the 433 MHz gateway is configured corrected and is connected it is possible to 
teach-in 433 MHz devices. A new section of devices will appear on the ``Device’’ overview in 
the setup menu of the user interface as shown in Figure \ref{433_5}.

\begin{figure}
\begin{center}
\includegraphics[width=0.7\textwidth]{pngs/cap4/433_5.png}
\caption{433 MHz option in 'Devices'}
\label{433_5}
\end{center}
\end{figure}

After clicking the \keystroke{Add} button choose the type of 433 MHz device to teach in:
\begin{enumerate}
\item Binary Sensors such as door sensors and motion detectors
\item Remote Controls
\item Actuators like Smart Plugs
\end{enumerate}

While Sensors and Remote controls only send out commands the actuators receive commands 
only. To teach them into the system a little trick is needed. Each Smart Plug or other 
actuator comes with a small remote control sending exactly the commands expected by the 
actuator. In order to control an actuator, the associated remote control is required to 
issue commands that can be captured by \zway.

\begin{figure}
\begin{center}
\includegraphics[width=0.7\textwidth]{pngs/cap9/433_13.png}
\caption{433 MHz teach in}
\label{433_13}
\end{center}
\end{figure}

The remote control itself is handled in the same way. The only difference is the way the 
elements are created.

Once device type is selected and the teach in process has started, all the buttons of the 
remote control must be pressed, one after each other. Figure \ref{433_13} shows this moment.
The pulse train of the button commands are shown in the tech-in dialog and the number of 
the remote-control buttons can be assigned.

For Actuators the same process applied, but the status of the actuator needs to be 
assigned to the pulse train.

Another specialty concerns binary sensors. Some 433 MHz sensors send signals on open and on 
close. However, some other sensors only create one wireless command when the sensor 
trips. For alarm systems, this is enough but for Smart Home with User Interfaces this 
is not working since the element cannot show the actual status. For these kinds of 
sensors, there is an automatic switch back to off function after a defined time 
interval. Figure \ref{433_12}

\begin{figure}
\begin{center}
\includegraphics[width=0.7\textwidth]{pngs/cap9/433_12.png}
\caption{433 MHz teach in of a binary sensor}
\label{433_12}
\end{center}
\end{figure}

Once all pulse trains are captured, the new element need to be renames and assigned into a room.

The management function for 433 MHz devices allows accessing the setup and change it as shown in figure \ref{433_11}

\begin{figure}
\begin{center}
\includegraphics[width=0.7\textwidth]{pngs/cap9/433_11.png}
\caption{433 MHz device management}
\label{433_11}
\end{center}
\end{figure}

\section {EnOcean devices}

EnOcean is another wireless communication technology optimized for very low power consumption.

\zway has implemented support for EnOcean devices but limits its function to sensors and 
wall switches because of their battery-free and therefore maintenance-free design. Compared 
to \zwave, EnOcean is a quite simple protocol. There is no such thing like network 
inclusion or routing---every EnOcean device just sends out a specific datagram that 
includes a unique device id and the data (sensor values, switch status) of the specific 
function of the device. The encoding of these data is defined in so-called profiles. These 
profiles are identified by a three-byte value but they are not transmitted wirelessly. 
Hence the user must decide from his product knowledge what profile a certain device is 
using. The EnOcean receiver will use this information to decode the datagram and use the 
data. (This means that a wrong decision about the profile of a EnOcean device will lead 
to severe malfunctions of the system).


Every EnOcean receiver in proximity will always receive every datagram sent by a transmitter. 
This leads to two basic management functions of the EnOcean module:
\begin{enumerate}
\item select the right products (by their unique 4 Byte ID) to use---and ignore all others
\item define the correct profile by selecting the right product
\end{enumerate}


\begin{figure}
\begin{center}
\includegraphics[width=0.4\textwidth]{pngs/cap9/enocean1.png}
\caption{Popp EnOcean USB Stick}
\label{enocean1}
\end{center}
\end{figure}

To work with EnOcean devices, an EnOcean USB Stick is required. Please use the Popp 
EnOcean Stick (POPE12204) as shown in Figure \ref{enocean1} and plug it into the USB 
port\footnote{Other EnOcean sticks may work as well, but the correct function is not 
supported and they may stop working after \zway firmware updates.}.

Next, the EnOcean app must be installed from the app store and configured as 
shown in Figure \ref{enocean2}.

\begin{figure}
\begin{center}
\includegraphics[width=0.7\textwidth]{pngs/cap9/enocean2.png}
\caption{EnOcean App configuration}
\label{enocean2}
\end{center}
\end{figure}

Make sure to pick the right device name of the EnOcean USB Stick connected to your 
hardware. For Raspberry Pi-based platforms this is always \cmdline{/dev/USB0} but for other 
platforms this may be different. The internal name, ``zeno,’’ can be arbitrarily chosen. 
In case more than one EnOcean stick is operated this name needs to be unique.

\begin{figure}
\begin{center}
\includegraphics[width=0.7\textwidth]{pngs/cap9/enocean3.png}
\caption{EnOcean Teach In}
\label{enocean3}
\end{center}
\end{figure}

Now it is possible to ``teach-in’’ new products using the user interfaces ``Device’’ 
section \menu{Configuration > Device > EnOcean}.

As described above the first step is to select the right product. A list of manufacturers 
with their products are given to select from. Please note that the EnOcean module may 
support many more devices from other manufacturers as long as they have the same profile. 
A good example for this is a door window sensor (profile name D5-00-01). Multiple 
manufacturers off door-window sensors, some even in the same enclosure but some in 
slightly changed enclosure. Their EnOcean wireless capability is however similar.

You may find the profile name on the label of the unknown device or documented in the 
device specification of the manufacturer.

\textbf{Please not that without proper knowledge of the device and its profile it is 
impossible to operate the device!}

After selecting the right device, the user interface asks for the teach-in process 
as shown in Figure \ref{enocean3}. During this process, the new device must send 
out one datagram containing the unique ID. The user interface will give some 
hints how to generate such a datagram.

\begin{figure}
\begin{center}
\includegraphics[width=0.7\textwidth]{pngs/cap9/enocean4.png}
\caption{EnOcean Device Configuration after Teach-In}
\label{enocean4}
\end{center}
\end{figure}

Once this datagram was received, the EnOcean module will generate virtual devices 
according to the profile selected. You can change the names of the elements to be 
generated. Figure \ref{enocean4} shows this dialog.

Finally, as shown in Figure \ref{enocean5} one or multiple elements will appear in 
the elements view. One example of the wall controller device shows Figure \ref{enocean6}.

\begin{figure}
\begin{center}
\includegraphics[width=0.7\textwidth]{pngs/cap9/enocean5.png}
\caption{EnOcean Device Elements}
\label{enocean5}
\end{center}
\end{figure}

The menu option \menu{Devices} also offers a special interface to manage EnOcean devices as 
shown in Figure \ref{enocean6}. This dialog offers a list of all known EnOcean devices 
with an option to change the profile.
Furthermore, it is possible to manually select one of the valid profiles of EnOcean for
 a device not known to the standard user interface. Please note that profiles not known 
 to the standard UI are also not supported by the standard user interface and will not 
 leads to creating new UI elements. However, the device is still created in the API 
 and can be used by third-party software.

\begin{figure}
\begin{center}
\includegraphics[width=0.7\textwidth]{pngs/cap9/enocean6.png}
\caption{EnOcean Device Management}
\label{enocean6}
\end{center}
\end{figure}

For a list of all supported EnOcean devices, please refer to Annex \ref{annexenocean}.
Experienced Users and programmers may extend this list by adding their own profiles to \zway.
Chapter \ref{addenocean} describes how to do this.


\section {Other IP/Internet-based services}

\zway can work with external IP-based systems. Please refer to the app store description 
for more information about how to integrate third-party IP-based devices. Please refer 
to Chapter \ref{apps} for details.

