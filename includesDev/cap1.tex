\chapter{Introduction}
\label{cap1}
\index{Z-Wave Basics}
\index{Z-Wave Literature}


\section{Structure of the book}

This book describes all aspects of the \zway controller software solution. This include 
both the \zway software solution and the hardware \zway runs on. The book is structured as follows:

\vspace{10mm}
	{\fontsize{14}{16}\selectfont
		\menu{Start > Use > Extend > Manage > Customize > Contribute}
	}
\vspace{10mm}

The start section provides the necessary information to fire up a \zway-based controller. 
This is followed by the explanation of the daily user interface --- called \zwshui ---
both for standard web browser and as native app for mobile devices. The next section cover 
the options to extend the system by supporting 
more radio technologies, third-party solutions, and other applications.

\begin{figure}
\begin{center}
\includegraphics[width=0.4\textwidth]{pngs/cap1/zwbook.png}
\caption{Z-Wave Essentials}
\label{zwbook}
\end{center}
\end{figure}

The next chapter explains tools and processes to manage and troubleshoot \zwave networks, 
followed by explanations of how to customize the user interface to your specific needs.

The final section of the book is dedicated to developers and programmers who can, and are 
willing to, contribute to the project and/or design their own solutions based on \zway.

The need for technical understanding and knowledge increases from chapter to chapter.

Please note that this book will not provide any basic knowledge about the \zwave technology
as such. Please
refer to the book 'Z-Wave Essentials' as shown in figure \ref{zwbook} for an comprehensive
explanation of the \zwave technology. The book is available at amazon.com and many other 
book sellers. The ISBN number is 978-1545394640.


\section{History of Z-Way}

The history of \zway dates back into the ear of 2009. Two developers had done their own
private \zwave controller written in Python. When they got engaged they realized that
they should combine their solutions and create a second generation \zwave controller.
The work on this merger started in May 2010 and the result - a complete \zwave controller 
written in python was certified by the Z-Wave Alliance in March 2011.

\murl{http://products.z-wavealliance.org/products/85}

To allow porting of this code to small memory platforms the whole software was rewritten
in C and Javascript was used as scripting engine. The same time the code was updated according 
the new \zwave Plus certification process and finally certified as first \zwave Plus 
compatible controller in Fall of 2014.
 
\murl{http://products.z-wavealliance.org/products/1150}
 
After many improvements the year of 2017 brought the next major change. As first software 
again \zway in Version 3.0 supports the new innovative security architecture of \zwave 
called S2.

\section{Status of the document}

The manual is based on \zway software release >= 2.3.5. Some functions marked in blue text
text require \zway v3.0 and up.
