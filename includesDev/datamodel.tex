\chapter{Z-Way Data Model Reference}
\label{datamodel}

\section{zway}

\begin {itemize}
\item Description: zway is the Z-Way part of the object tree
\item Syntax:  zway.X with  X as child object
\item Child objects
\begin {itemize}
\item controller: controller object, see below for details
\item devices: devices list, see below for details
\item version: Z-Way.JS version
\item isRunning(): Check if Z-Way is running
\item isIdle(): Check if Z-Way is idle (no pending packets to send)
\item discover(): Start Z-Way discovery process
\item stop() : Stop Z-Way
\item InspectQueue() : Returns list of pending jobs in the queue.
\begin {itemize}
\item item: [timeout, flags, nodeId, description, progress, payload]
\item flags: [send count, wait wakeup, wait security, done, wait ACK, got ACK, wait response, got response, wait callback, got callback]
\end {itemize}
\item ProcessPendingCallbacks(): Process pending callbacks (result of setTimeout/setInterval or functions called via HTTP JSON API)
\item bind(function, bitmask): Bind function to be called on change of devices list/instances list/command classes list
\item unbind(function) : Unbind function previously bind with bind()
\item all function classes in \ref{FunctionClasses} are also methods of this data object
\end {itemize}
\end {itemize}

\section{controller}

You can access the data elements of "controller" in the demo user interface in menu 
"for experts + Controller Info"

\begin {itemize}
\item Description: Controller object
\item Syntax: controller.X with  X as child object
\item Child objects
\begin {itemize}
\item data: Data tree of the controller
\begin {itemize}
\item  APIVersion: Version of the Serial API
\item  SDK: System development kit version of the Transceiver firmware
\item  SISPresent: flase if SUIS is available
\item  SUCNodeId: Node ID of SUC if present
\item  ZWVersion: ZWave Version (firmware)
\item  ZWaveChip: The name of the Z-Wave transceiver chip
\item  ZWlibMajor / ZWlibMinor: library version 
\item  capabilities: array of function class ids
\item  controllerstate: flag to show inclusion mode etc
\item  countJobs: shall job be counted
\item  curSerialAPIAckTimeout10ms: timing parameter of serial interface
\item  curSerialAPIBytetimeout10ms: timing parameter of serial interface
\item  homeId:the home id of the controller
\item  isinOtherNetworks: flag to show if controller is real primary if in other network
\item  isPrimary: flag to show if controller is primary
\item  isRealprimary: flag to show if controller can be  primary
\item  isSUC: is SUC present
\item  lastExcludedDevice: node ID of last excluded device
\item  lastIncludedDevice: node ID of last included device
\item  libType library basis type
\item  manufacturerIS / manufacturerProductId / manufacturerProductTypeId: ids to identify the transceiver hardware
\item  memoryGetAddress
\item  memoryGetData
\item  nodeId: own node ID
\item  nonManagementJobs: number of non man. jobs
\item  oldSerialAPIAckTimeout10ms: default timing parameter of serial interface
\item  oldSerialAPIBytetimeout10ms: default timing parameter of serial interface
\item  softwareRevisonDate: written by compiler
\item  softwareRevisionID: written by compiler
\item  vendor: string of hardware vendor
\end {itemize}
\item AddNodeToNetwork(mode): Reference to zway.AddNodeToNetwork()
\item RemoveNodeFromNetwork(mode): Reference to zway.RemoveNodeFromNetwork()
\item ControllerChange(mode): Reference to zway.ControllerChange()
\item GetSUCNodeId(mode): Reference to zway.GetSUCNodeId()
\item SetSUCNodeId(nodeId): Assign SUC role to a device
\item SetSISNodeId(nodeId): Assign SIS role to a device
\item DisableSUCNodeId(nodeId): Revoke SUC/SIS role from a device
\item SendNodeInformation(nodeId): Reference to zway.SendNodeInformation()
\end {itemize}
\end {itemize}


\section{Devices}

The devices object contains the array of the device objects. Each device in the network - including the 
controller itself -  has a device object in Z-Way.

\begin {itemize}
\item Description: list of devices
\item Syntax:  X with  X as child object
\item Child objects
\begin {itemize}
\item [m]: Device object
\item length: Length of the list
\item SaveData(): Save Z-Way Z-Wave data for hot start on next run (in config/zddx/HOMEID-DevicesData.xml)
\end {itemize}
\end {itemize}
 

\section{Device}

The data object can be accesses in the demo UI in advanced mode of "Configuration"

\begin {itemize}
\item Description: the device object
\item Syntax:  device[n].X with  X as child object
\item Child objects
\begin {itemize}
\item id: (node) Id of the device
\item Data: Data tree of the device
\begin {itemize}
\item SDK: SDK used in the device
\item ZDDXMLFile: file of the Devcie Description Record
\item ZWLib: Z-Wave library used
\item ZWProtocolMajor / ZWProtocolMinor: Z-Wave protocol version
\item applicationMajor / ApplicationMinor: Application Version of devices firmware
\item basicType: basic Z-Wave device class
\item beam: wake up beam required
\item countFailed: statistics of failed packets sent (from start of process)
\item countSuccess: statistics of sucessful packets sent (from start of process)
\item deviceTypeString:
\item genericType: generic Z-Wave device class
\item infoProtocolSpecific
\item isAwake
\item isListening
\item isFailed
\item isRouting
\item isVirtual: 
\item keeAwake: flag is device need to be kept awake
\item lastRecevied:  timestamp of last packet received
\item lastSend:  timestamp of last sent operation
\item ManufacturerId / manufacturerProductID / manufacturerProductTypeId: ids used to identify the device
\item neightbours: list of neighbour nodes
\item nodeInfoFrame: nodeinformation frame in bytes
\item option: flag if optional command classes are present
\item queueLength: length of device specific send queue
\item sensor1000: flag if device is FLIRS with 1000 ms wakeup 
\item sensor250: flag if device is FLIRS with 250 ms wakeup 
\item specificType: specific Z-Wave device class
\end {itemize}
\item instances: Instances list of the device
\item RequestNodeInformation(): Request NIF
\item RequestNodeNeighbourUpdate(): Request routes update
\item InterviewForce(): Purge all command classes and start interview based on device's NIF
\item RemoveFailedNode(): Remove this node as failed. Device should be marked as failed to remove it with this function.
\item SendNoOperation(): Ping the device with empty packet
\item LoadXMLFile(file): Load new Z-Wave Device Description XML file. See http://pepper1.net/zwavedb/
\item GuestXML(): Return the list of all known Z-Wave Device Description XML files with match score. [score, file name, brand name, product name, photo]
\item WakeupQueue(): Pretend the device is awake and try to send packets
\item AssignReturnRoute(target): Send device new routes to target node
\item DeleteReturnRoute(): Clear routes in device
\item AssignSUCReturnRoute(): Inform device about SUC and route to reach it
\end {itemize}
\end {itemize}

\section{Instances}

Each device may have multiple instances (similar functions like switches, same type sensors, ...) If only one instance 
is present the id of this instance is 0. Command classes are located in instances only-

\begin {itemize}
\item Description: list of instances
\item Syntax:  device[n].instance[m].X with  X as child object
\item Child objects
\begin {itemize}
\item [m]: instance object
\item length: Length of the list
\item commandClasses: list of command classes of this instance. In case there is only one instance, this is equivalent to the list of command classes of the device. For details see below.
\item Data: data object of instance 
\begin {itemize}
\item dynamic: flag if instance is dynamic
\item genericType: generic Z-Wave device class of instance
\item specificType: specific Z-Wave device class of instance
\end {itemize}
\end {itemize}
\end {itemize}

\subsection{CommandClass}

This is the command class object. It contains public methods and public data elements that are described
in chapter \ref{ccs}

\begin {itemize}
\item Description: Command Class Implementation
\item Syntax:  device[n].instance[m].commandclass[id].X with  X as child object
\item Child objects
\begin {itemize}
\item id: Id of the Command Class of the instance of the device
\item data: Data tree of the Command Class
\begin {itemize}
\item interviewCounter: number of attempts left until interview is terminate even if not successful
\item interviewDone: flag if interview of the command class is finished
\item security: flag if command class is operated under special security mode
\item version: version of the command class implemented
\item supported: flag if CC is supported or only controlled
\item {commandclass data}: Command Class specific data - see chapter \ref{ccs} for details.
\end {itemize}
\item name: Command Class name
\item {Method}: Command Class method - see chapter \ref{ccs} for details.
\end {itemize}
\end {itemize}

\section{Data}

This is the description of the data object.

General note: Z-Way objects and it's decendents are NOT simple JS objects, but native JS objects, 
that does not allow object modification.

\begin {itemize}
\item name: Name of data tree element
\item updated: Update time
\item invalidated: Invalidate time
\item valueOf(): Returns value of the object (can be omitted to get object value)
\item invalidate(): Invalidate data value (mark is as not valid anymore)
\item bind(function (type[, arg]) {...}, [arg, [watchChildren=false]]): Bind function to a change of data tree element of its descendants
\item unbind(function): Unbind function bind previously with bind()
\end {itemize}

 
