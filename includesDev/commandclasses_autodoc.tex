

\section{Command Class Basic (0x20/32)}

\textit{Version 1, Supported and Controlled}
\newline

The Basic Command Class is the wildcard command class. Almost all Z-Wave devices support this command class but they interpret it's commands in different ways. A thermostat will handle a Basic Set Command in a different way than a Dimmer but both accept the Basic Set command and act. Used for generic interoperability between devices. You should always use more specific Command Classes where possible.
\newline

\noindent
Data holders:

\begin{itemize}
\item \textbf{level}: Generic switching level of the device controlled
\end{itemize}

\paragraph{Command Basic Get}
\begin{quote}Syntax: Get(successCallback = NULL, failureCallback = NULL)\end{quote}
\begin{quote}Description: Send Basic Get\end{quote}
\begin{quote}Parameter successCallback: Custom function to be called on function success. NULL if callback is not needed\end{quote}
\begin{quote}Parameter failureCallback: Custom function to be called on function failure. NULL if callback is not needed\end{quote}
\begin{quote}Report: level updated\end{quote}

\paragraph{Command Basic Set}
\begin{quote}Syntax: Set(value, successCallback = NULL, failureCallback = NULL)\end{quote}
\begin{quote}Description: Send Basic Set\end{quote}
\begin{quote}Parameter value: Value\end{quote}
\begin{quote}Parameter successCallback: Custom function to be called on function success. NULL if callback is not needed\end{quote}
\begin{quote}Parameter failureCallback: Custom function to be called on function failure. NULL if callback is not needed\end{quote}
\begin{quote}Report: level updated\end{quote}


\section{Command Class Wakeup (0x84/132)}

\textit{Version 2, Controlled}
\newline

Allows to manage periodical wakeup of sleeping battery operated device. Upon wakeup device will notify one node listed in nodeId. NB! If the device can wake up by interrupt (user interaction, button press, sensor trigger), it might happen that the device never wakes up. This can happen if you wake up the device by interrupe each time before internal chip wakeup period (usually from 1 to 4 minutes) reaches. (Z-Wave chip can not count for remaining time to next wakeup, so will restart timer again). This means that strictly speaking you can not rely on long time no wake up as an indicator of lost/damaged device or battery empty. NB! To save battery it is recommended to tune wakeup period to one week or even more for devices that do only need to report battery on wakeup (remote controls). For sensors it is recommended to have at least one hour wakeup period.
\newline

\noindent
Data holders:

\begin{itemize}
\item \textbf{interval}: Wakeup interval in seconds
\item \textbf{nodeId}: Node to notify about wakeup
\item \textbf{min}: Minimal possible wakeup interval
\item \textbf{max}: Maximal possible wakeup interval
\item \textbf{default}: Factory default wakeup interval
\item \textbf{step}: Step for wakeup interval (values are rounded to next or previous step)
\item \textbf{lastWakeup}: Last time the device has sent us wake notification (Unix timestamp)
\item \textbf{lastSleep}: Last time the device was sent into sleep mode (Unix timestamp)
\end{itemize}

\paragraph{Command Wakeup Get}
\begin{quote}Syntax: Get(successCallback = NULL, failureCallback = NULL)\end{quote}
\begin{quote}Description: Send Wakeup Get\end{quote}
\begin{quote}Parameter successCallback: Custom function to be called on function success. NULL if callback is not needed\end{quote}
\begin{quote}Parameter failureCallback: Custom function to be called on function failure. NULL if callback is not needed\end{quote}
\begin{quote}Report: interval and nodeId updated\end{quote}

\paragraph{Command Wakeup CapabilitiesGet}
\begin{quote}Syntax: CapabilitiesGet(successCallback = NULL, failureCallback = NULL)\end{quote}
\begin{quote}Description: Send Wakeup CapabilityGet\end{quote}
\begin{quote}Parameter successCallback: Custom function to be called on function success. NULL if callback is not needed\end{quote}
\begin{quote}Parameter failureCallback: Custom function to be called on function failure. NULL if callback is not needed\end{quote}
\begin{quote}Report: min, max, default, step updated\end{quote}

\paragraph{Command Wakeup Sleep}
\begin{quote}Syntax: Sleep(successCallback = NULL, failureCallback = NULL)\end{quote}
\begin{quote}Description: Send Wakeup NoMoreInformation (Sleep)\end{quote}
\begin{quote}Parameter successCallback: Custom function to be called on function success. NULL if callback is not needed\end{quote}
\begin{quote}Parameter failureCallback: Custom function to be called on function failure. NULL if callback is not needed\end{quote}
\begin{quote}Report: lastSleep updated\end{quote}

\paragraph{Command Wakeup Set}
\begin{quote}Syntax: Set(interval, notificationNodeId, successCallback = NULL, failureCallback = NULL)\end{quote}
\begin{quote}Description: Send Wakeup Set\end{quote}
\begin{quote}Parameter interval: Wakeup interval in seconds\end{quote}
\begin{quote}Parameter notificationNodeId: Node Id to be notified about wakeup\end{quote}
\begin{quote}Parameter successCallback: Custom function to be called on function success. NULL if callback is not needed\end{quote}
\begin{quote}Parameter failureCallback: Custom function to be called on function failure. NULL if callback is not needed\end{quote}
\begin{quote}Report: interval and nodeId updated\end{quote}


\section{Command Class NoOperation (0x00/0)}

Used to check if device is reachable by sending empty packet.

\section{Command Class Battery (0x80/128)}

\textit{Version 1, Controlled}
\newline

Allows monitoring the battery charging level of a device.
\newline

\noindent
Data holders:

\begin{itemize}
\item \textbf{last}: Last battery level reported (0..100\%)
\item \textbf{lastChange}: Time (UNIX timestamp) when the battery was replaced last time (time of the moment when the value reported was way bigger than previous one)
\item \textbf{history}: Subtree with history
\item \qquad\textbf{[\% value]}: Time when battery level reached this \% value (0, 10, 20,... 100)
\end{itemize}

\paragraph{Command Battery Get}
\begin{quote}Syntax: Get(successCallback = NULL, failureCallback = NULL)\end{quote}
\begin{quote}Description: Send Battery Get\end{quote}
\begin{quote}Parameter successCallback: Custom function to be called on function success. NULL if callback is not needed\end{quote}
\begin{quote}Parameter failureCallback: Custom function to be called on function failure. NULL if callback is not needed\end{quote}
\begin{quote}Report: last updated. lastChange updated if battery level is way higher than it was before, history updated if reached next 10\% step\end{quote}


\section{Command Class ManufacturerSpecific (0x72/114)}

\textit{Version 2, Supported and Controlled}
\newline

Reports vendor information, product type and ID and device serial number.
\newline

\noindent
Data holders:

\begin{itemize}
\item \textbf{vendorId}: Vendor ID assigned by Sigma Designs
\item \textbf{vendor}: Vendor name
\item \textbf{productId}: Product ID
\item \textbf{productType}: Product Type ID
\item \textbf{serialNumber}: Product Serial Number
\end{itemize}

\paragraph{Command ManufacturerSpecific Get}
\begin{quote}Syntax: Get(successCallback = NULL, failureCallback = NULL)\end{quote}
\begin{quote}Description: Send ManufacturerSpecific Get\end{quote}
\begin{quote}Parameter successCallback: Custom function to be called on function success. NULL if callback is not needed\end{quote}
\begin{quote}Parameter failureCallback: Custom function to be called on function failure. NULL if callback is not needed\end{quote}


\paragraph{Command ManufacturerSpecific DeviceIdGet}
\begin{quote}Syntax: DeviceIdGet(type, successCallback = NULL, failureCallback = NULL)\end{quote}
\begin{quote}Description: Send ManufacturerSpecific Device Id Get\end{quote}
\begin{quote}Parameter type: Device Id type to request\end{quote}
\begin{quote}Parameter successCallback: Custom function to be called on function success. NULL if callback is not needed\end{quote}
\begin{quote}Parameter failureCallback: Custom function to be called on function failure. NULL if callback is not needed\end{quote}



\section{Command Class Proprietary (0x88/136)}

\textit{Version 1, Controlled}
\newline

Allows to transfer manufacturer proprietary data. Data format is manufacturer specific.
\newline

\noindent
Data holders:

\begin{itemize}
\item \textbf{bytes}: Binary bytes array of raw data
\end{itemize}

\paragraph{Command Proprietary Get}
\begin{quote}Syntax: Get(successCallback = NULL, failureCallback = NULL)\end{quote}
\begin{quote}Description: Send Proprietary Get\end{quote}
\begin{quote}Parameter successCallback: Custom function to be called on function success. NULL if callback is not needed\end{quote}
\begin{quote}Parameter failureCallback: Custom function to be called on function failure. NULL if callback is not needed\end{quote}


\paragraph{Command Proprietary Set}
\begin{quote}Syntax: Set(data, successCallback = NULL, failureCallback = NULL)\end{quote}
\begin{quote}Description: Send Proprietary Set\end{quote}
\begin{quote}Parameter data: Data to set\end{quote}
\begin{quote}Parameter successCallback: Custom function to be called on function success. NULL if callback is not needed\end{quote}
\begin{quote}Parameter failureCallback: Custom function to be called on function failure. NULL if callback is not needed\end{quote}



\section{Command Class Configuration (0x70/112)}

\textit{Version 1, Controlled}
\newline

Used to set certian configuration valeus that change the behavior of the device. Z-Wave requires that every device works out of the box without further configuration. However different configuration value significantly enhance the value a device. Z-Wave does not provide any information about the configuration values by wireless commands. User have to look into the device manual to learn about configuration parameters. The Device Description Record (ZDDX), incoprotated by Z-Way gives information about valid parameters and the meaning of the values to be set.
\newline

\noindent
Data holders:

\begin{itemize}
\item \textbf{[paramId]}: Configuration parameter subtree.
\item \qquad\textbf{val}: Value assigned
\item \qquad\textbf{size}: Size of that parameter (1, 2 or 4 bytes)
\end{itemize}

\paragraph{Command Configuration Get}
\begin{quote}Syntax: Get(parameter, successCallback = NULL, failureCallback = NULL)\end{quote}
\begin{quote}Description: Send Configuration Get\end{quote}
\begin{quote}Parameter parameter: Parameter number (from 1 to 255)\end{quote}
\begin{quote}Parameter successCallback: Custom function to be called on function success. NULL if callback is not needed\end{quote}
\begin{quote}Parameter failureCallback: Custom function to be called on function failure. NULL if callback is not needed\end{quote}
\begin{quote}Report: parameter subtree updated or created if absent\end{quote}

\paragraph{Command Configuration Set}
\begin{quote}Syntax: Set(parameter, value, size = 0, successCallback = NULL, failureCallback = NULL)\end{quote}
\begin{quote}Description: Send Configuration Set\end{quote}
\begin{quote}Parameter parameter: Parameter number (from 1 to 255)\end{quote}
\begin{quote}Parameter value: Value to be sent (negative and positive values are accepted, but will be stripped to size)\end{quote}
\begin{quote}Parameter size: Size of the value (1, 2 or 4 bytes). Use 0 to guess from previously reported value if any. 0 means use size previously obtained Get\end{quote}
\begin{quote}Parameter successCallback: Custom function to be called on function success. NULL if callback is not needed\end{quote}
\begin{quote}Parameter failureCallback: Custom function to be called on function failure. NULL if callback is not needed\end{quote}
\begin{quote}Report: parameter subtree updated or created if absent\end{quote}

\paragraph{Command Configuration SetDefault}
\begin{quote}Syntax: SetDefault(parameter, successCallback = NULL, failureCallback = NULL)\end{quote}
\begin{quote}Description: Send Configuration SetDefault\end{quote}
\begin{quote}Parameter parameter: Parameter number to be set to device default\end{quote}
\begin{quote}Parameter successCallback: Custom function to be called on function success. NULL if callback is not needed\end{quote}
\begin{quote}Parameter failureCallback: Custom function to be called on function failure. NULL if callback is not needed\end{quote}
\begin{quote}Report: parameter subtree updated or created if absent\end{quote}


\section{Command Class SensorBinary (0x30/48)}

\textit{Version 2, Controlled}
\newline

Allows receive binary sensor states.
\newline

\noindent
Data holders:

\begin{itemize}
\item \textbf{typemask}: Internal. Bit mask of the supported types
\item \textbf{[sensorType]}: Subtree for sensor type Id
\item \qquad\textbf{sensorTypeString}: Description of sensor type
\item \qquad\textbf{level}: Triggered/idle status
\end{itemize}

\paragraph{Command SensorBinary Get}
\begin{quote}Syntax: Get(sensorType = -1, successCallback = NULL, failureCallback = NULL)\end{quote}
\begin{quote}Description: Send SensorBinary Get\end{quote}
\begin{quote}Parameter sensorType: Type of sensor to query information for. 0xFF to query information for the first available sensor type. -1 to query information for all supported sensor types\end{quote}
\begin{quote}Parameter successCallback: Custom function to be called on function success. NULL if callback is not needed\end{quote}
\begin{quote}Parameter failureCallback: Custom function to be called on function failure. NULL if callback is not needed\end{quote}
\begin{quote}Report: sensorType subtree updated\end{quote}


\section{Command Class Association (0x85/133)}

\textit{Version 2, Supported and Controlled}
\newline

Allows to manage the association groups: adding and removing nodeIDs in the association groups.
\newline

\noindent
Data holders:

\begin{itemize}
\item \textbf{groups}: Number of association groups in the device
\item \textbf{[groupId]}: Group subtree, where groupId = 1..groups
\item \qquad\textbf{max}: Number of nodes the group can hold
\item \qquad\textbf{nodes}: Array with nodes in the group
\item \qquad\textbf{nodesToFollow}: Internal
\item \textbf{specificGroup}: Number of specific association groups in the device
\end{itemize}

\paragraph{Command Association Get}
\begin{quote}Syntax: Get(groupId = 0, successCallback = NULL, failureCallback = NULL)\end{quote}
\begin{quote}Description: Send Association Get\end{quote}
\begin{quote}Parameter groupId: Group Id (from 1 to 255). 0 requests all groups\end{quote}
\begin{quote}Parameter successCallback: Custom function to be called on function success. NULL if callback is not needed\end{quote}
\begin{quote}Parameter failureCallback: Custom function to be called on function failure. NULL if callback is not needed\end{quote}
\begin{quote}Report: Subtree corresponding to the group updated\end{quote}

\paragraph{Command Association Set}
\begin{quote}Syntax: Set(groupId, includeNode, successCallback = NULL, failureCallback = NULL)\end{quote}
\begin{quote}Description: Send Association Set (Add)\end{quote}
\begin{quote}Parameter groupId: Group Id (from 1 to 255)\end{quote}
\begin{quote}Parameter includeNode: Node to be added to the group\end{quote}
\begin{quote}Parameter successCallback: Custom function to be called on function success. NULL if callback is not needed\end{quote}
\begin{quote}Parameter failureCallback: Custom function to be called on function failure. NULL if callback is not needed\end{quote}
\begin{quote}Report: Subtree corresponding to the group updated\end{quote}

\paragraph{Command Association Remove}
\begin{quote}Syntax: Remove(groupId, excludeNode, successCallback = NULL, failureCallback = NULL)\end{quote}
\begin{quote}Description: Send Association Remove\end{quote}
\begin{quote}Parameter groupId: Group Id (from 1 to 255)\end{quote}
\begin{quote}Parameter excludeNode: Node to be removed from the group\end{quote}
\begin{quote}Parameter successCallback: Custom function to be called on function success. NULL if callback is not needed\end{quote}
\begin{quote}Parameter failureCallback: Custom function to be called on function failure. NULL if callback is not needed\end{quote}
\begin{quote}Report: Subtree corresponding to the group updated\end{quote}

\paragraph{Command Association GroupingsGet}
\begin{quote}Syntax: GroupingsGet(successCallback = NULL, failureCallback = NULL)\end{quote}
\begin{quote}Description: Send Association GroupingsGet\end{quote}
\begin{quote}Parameter successCallback: Custom function to be called on function success. NULL if callback is not needed\end{quote}
\begin{quote}Parameter failureCallback: Custom function to be called on function failure. NULL if callback is not needed\end{quote}
\begin{quote}Report: Update number of supported groups and interview all groups\end{quote}


\section{Command Class Meter (0x32/50)}

\textit{Version 4, Controlled}
\newline

Allows to read different kind of meters. Z-Wave differentiates different meter types and different meter scales. Please refer to the file translations/Scales.xml for details about possible meter types and values.
\newline

\noindent
Data holders:

\begin{itemize}
\item \textbf{scalemask}: Internal. Bit mask with supported scales
\item \textbf{resettable}: Flag to indicate of the meter can be resetted
\item \textbf{[scaleId]}: Meter scale subtree
\item \qquad\textbf{scale}: Meter scale id
\item \qquad\textbf{scaleString}: Meter scale name
\item \qquad\textbf{sensorType}: Sensor type id
\item \qquad\textbf{sensorTypeString}: Sensor type name
\item \qquad\textbf{val}: Meter value
\item \qquad\textbf{ratetype}: Rate type
\item \qquad\textbf{delta}: Delta from the last value requested
\item \qquad\textbf{previous}: Previous value requested
\end{itemize}

\paragraph{Command Meter Get}
\begin{quote}Syntax: Get(scale = -1, successCallback = NULL, failureCallback = NULL)\end{quote}
\begin{quote}Description: Send Meter Get\end{quote}
\begin{quote}Parameter scale: Desired scale. -1 for all scales\end{quote}
\begin{quote}Parameter successCallback: Custom function to be called on function success. NULL if callback is not needed\end{quote}
\begin{quote}Parameter failureCallback: Custom function to be called on function failure. NULL if callback is not needed\end{quote}
\begin{quote}Report: scale subtree updated\end{quote}

\paragraph{Command Meter Reset}
\begin{quote}Syntax: Reset(successCallback = NULL, failureCallback = NULL)\end{quote}
\begin{quote}Description: Send Meter Reset\end{quote}
\begin{quote}Parameter successCallback: Custom function to be called on function success. NULL if callback is not needed\end{quote}
\begin{quote}Parameter failureCallback: Custom function to be called on function failure. NULL if callback is not needed\end{quote}
\begin{quote}Report: scale subtree updated\end{quote}

\paragraph{Command Meter Supported}
\begin{quote}Syntax: Supported(successCallback = NULL, failureCallback = NULL)\end{quote}
\begin{quote}Description: Send Meter SupportedGet\end{quote}
\begin{quote}Parameter successCallback: Custom function to be called on function success. NULL if callback is not needed\end{quote}
\begin{quote}Parameter failureCallback: Custom function to be called on function failure. NULL if callback is not needed\end{quote}



\section{Command Class Meter Pulse (0x35/53)}

\textit{Version 1, Controlled}
\newline

Allows to gather information from pulse meters.
\newline

\noindent
Data holders:

\begin{itemize}
\item \textbf{val}: Meter pulse value
\end{itemize}

\paragraph{Command MeterPulse Get}
\begin{quote}Syntax: Get(successCallback = NULL, failureCallback = NULL)\end{quote}
\begin{quote}Description: Send MeterPulse Get\end{quote}
\begin{quote}Parameter successCallback: Custom function to be called on function success. NULL if callback is not needed\end{quote}
\begin{quote}Parameter failureCallback: Custom function to be called on function failure. NULL if callback is not needed\end{quote}



\section{Command Class SensorMultilevel (0x31/49)}

\textit{Version 11, Controlled}
\newline

Allows to read different kind of sensor. Z-Wave differentiates different sensor types and different scales of this sensor. Please refer to the file /translations/scales.xml for details about possible sensor types and values.
\newline

\noindent
Data holders:

\begin{itemize}
\item \textbf{typemask}: Internal. Bit mask of the supported types
\item \textbf{[sensorTypeId]}: Subtree for sensor type Id
\item \qquad\textbf{sensorTypeString}: Description of sensor type
\item \qquad\textbf{scale}: Scale Id
\item \qquad\textbf{scaleString}: Scale description
\item \qquad\textbf{val}: Value
\item \qquad\textbf{size}: Internal. Size of the value (1, 2 or 4 bytes)
\item \qquad\textbf{precision}: Internal. Precision used in value (number of digits after decimal dot)
\item \qquad\textbf{deviceScale}: Internal. Scale Id on the device's side (if local conversion is used, like C->F)
\end{itemize}

\paragraph{Command SensorMultilevel Get}
\begin{quote}Syntax: Get(sensorType = -1, successCallback = NULL, failureCallback = NULL)\end{quote}
\begin{quote}Description: Send SensorMultilevel Get\end{quote}
\begin{quote}Parameter sensorType: Type of sensor to be requested. -1 means all sensor types supported by the device\end{quote}
\begin{quote}Parameter successCallback: Custom function to be called on function success. NULL if callback is not needed\end{quote}
\begin{quote}Parameter failureCallback: Custom function to be called on function failure. NULL if callback is not needed\end{quote}
\begin{quote}Report: sensorTypeId subtree updated\end{quote}


\section{Command Class Sensor Configuration (0x9E/158)}

\textit{Version 1, Controlled}
\newline

Allows to configure senors specific configuration like trigger level. Works in conjunction with SensorMultilevel Command Class. In modern devices replaced by Configuration Command Class.
\newline

\noindent
Data holders:

\begin{itemize}
\item \textbf{sensorType}: Sensor type Id
\item \textbf{sensorTypeString}: Sensor type descirption
\item \textbf{val}: Trigger value
\item \textbf{scale}: Scale of trigger value
\item \textbf{scaleString}: Scale description
\item \textbf{size}: Internal. Size of the value (1, 2 or 4 bytes)
\item \textbf{precision}: Internal. Precision used in value (number of digits after decimal dot)
\end{itemize}

\paragraph{Command SensorConfiguration Get}
\begin{quote}Syntax: Get(successCallback = NULL, failureCallback = NULL)\end{quote}
\begin{quote}Description: Send SensorConfiguration Get\end{quote}
\begin{quote}Parameter successCallback: Custom function to be called on function success. NULL if callback is not needed\end{quote}
\begin{quote}Parameter failureCallback: Custom function to be called on function failure. NULL if callback is not needed\end{quote}
\begin{quote}Report: all dataholders are updated\end{quote}

\paragraph{Command SensorConfiguration Set}
\begin{quote}Syntax: Set(mode, value, successCallback = NULL, failureCallback = NULL)\end{quote}
\begin{quote}Description: Send SensorConfiguration Set\end{quote}
\begin{quote}Parameter mode: Value set mode\end{quote}
\begin{quote}Parameter value: Value\end{quote}
\begin{quote}Parameter successCallback: Custom function to be called on function success. NULL if callback is not needed\end{quote}
\begin{quote}Parameter failureCallback: Custom function to be called on function failure. NULL if callback is not needed\end{quote}
\begin{quote}Report: all dataholders are updated\end{quote}


\section{Command Class SwitchAll (0x27/39)}

\textit{Version 1, Supported and Controlled}
\newline

Controls the behavior of a actuator on Switch All commands. Also allows to send Switch All commands.
\newline

\noindent
Data holders:

\begin{itemize}
\item \textbf{mode}: Which type of SwitchAll On/Off commands to react on: 0 for none, 1 to reacto on Off only, 2 to react on On only, 255 to react on both
\item \textbf{onOff}: Allows to trigger SwitchAll On/Off commands from other devices. Set to False on Off command received and True on On command.
\end{itemize}

\paragraph{Command SwitchAll Get}
\begin{quote}Syntax: Get(successCallback = NULL, failureCallback = NULL)\end{quote}
\begin{quote}Description: Send SwitchAll Get\end{quote}
\begin{quote}Parameter successCallback: Custom function to be called on function success. NULL if callback is not needed\end{quote}
\begin{quote}Parameter failureCallback: Custom function to be called on function failure. NULL if callback is not needed\end{quote}
\begin{quote}Report: mode updated\end{quote}

\paragraph{Command SwitchAll Set}
\begin{quote}Syntax: Set(mode, successCallback = NULL, failureCallback = NULL)\end{quote}
\begin{quote}Description: Send SwitchAll Set\end{quote}
\begin{quote}Parameter mode: SwitchAll Mode: see definitions below\end{quote}
\begin{quote}Parameter successCallback: Custom function to be called on function success. NULL if callback is not needed\end{quote}
\begin{quote}Parameter failureCallback: Custom function to be called on function failure. NULL if callback is not needed\end{quote}
\begin{quote}Report: mode updated\end{quote}

\paragraph{Command SwitchAll SetOn}
\begin{quote}Syntax: SetOn(successCallback = NULL, failureCallback = NULL)\end{quote}
\begin{quote}Description: Send SwitchAll Set On\end{quote}
\begin{quote}Parameter successCallback: Custom function to be called on function success. NULL if callback is not needed\end{quote}
\begin{quote}Parameter failureCallback: Custom function to be called on function failure. NULL if callback is not needed\end{quote}


\paragraph{Command SwitchAll SetOff}
\begin{quote}Syntax: SetOff(successCallback = NULL, failureCallback = NULL)\end{quote}
\begin{quote}Description: Send SwitchAll Set Off\end{quote}
\begin{quote}Parameter successCallback: Custom function to be called on function success. NULL if callback is not needed\end{quote}
\begin{quote}Parameter failureCallback: Custom function to be called on function failure. NULL if callback is not needed\end{quote}



\section{Command Class SwitchBinary (0x25/37)}

\textit{Version 1, Supported and Controlled}
\newline

Allows to control On/Off switches, actuators, electrical power switches and trap On/Off control commands from other devices.
\newline

\noindent
Data holders:

\begin{itemize}
\item \textbf{level}: State: False for Off, True for On
\end{itemize}

\paragraph{Command SwitchBinary Get}
\begin{quote}Syntax: Get(successCallback = NULL, failureCallback = NULL)\end{quote}
\begin{quote}Description: Send SwitchBinary Get\end{quote}
\begin{quote}Parameter successCallback: Custom function to be called on function success. NULL if callback is not needed\end{quote}
\begin{quote}Parameter failureCallback: Custom function to be called on function failure. NULL if callback is not needed\end{quote}
\begin{quote}Report: level updated\end{quote}

\paragraph{Command SwitchBinary Set}
\begin{quote}Syntax: Set(value, successCallback = NULL, failureCallback = NULL)\end{quote}
\begin{quote}Description: Send SwitchBinary Set\end{quote}
\begin{quote}Parameter value: Value\end{quote}
\begin{quote}Parameter successCallback: Custom function to be called on function success. NULL if callback is not needed\end{quote}
\begin{quote}Parameter failureCallback: Custom function to be called on function failure. NULL if callback is not needed\end{quote}
\begin{quote}Report: level updated\end{quote}


\section{Command Class SwitchMultilevel (0x26/38)}

\textit{Version 1, Supported and Controlled}
\newline

Allows to control all actuators with multilevel switching functions, primarily Dimmers and Motor Controlling devices as well as trap dim events sent by remotes.
\newline

\noindent
Data holders:

\begin{itemize}
\item \textbf{level}: State 0...99 = 0...100\%, 255 for On on last value (or on maximum - device specific)
\item \textbf{startChange}: Dimming up or down. Updated on dimming start. Allows to trap events from remotes to controller.
\item \textbf{stopChange}: Updated on dimming end. Allows to trap events from remotes to controller.
\item \textbf{prevLevel}: Internal
\item \textbf{primary}: Unused
\item \textbf{secondary}: Unsued
\end{itemize}

\paragraph{Command SwitchMultilevel Get}
\begin{quote}Syntax: Get(successCallback = NULL, failureCallback = NULL)\end{quote}
\begin{quote}Description: Send SwitchMultilevel Get\end{quote}
\begin{quote}Parameter successCallback: Custom function to be called on function success. NULL if callback is not needed\end{quote}
\begin{quote}Parameter failureCallback: Custom function to be called on function failure. NULL if callback is not needed\end{quote}
\begin{quote}Report: level updated\end{quote}

\paragraph{Command SwitchMultilevel Set}
\begin{quote}Syntax: Set(level, duration = 0xff, successCallback = NULL, failureCallback = NULL)\end{quote}
\begin{quote}Description: Send SwitchMultilevel Set\end{quote}
\begin{quote}Parameter level: Level to be set\end{quote}
\begin{quote}Parameter duration: Duration of change:. 0 instantly. 0x01...0x7f in seconds. 0x80...0xfe in minutes mapped to 1...127 (value 0x80=128 is 1 minute). 0xff use device factory default\end{quote}
\begin{quote}Parameter successCallback: Custom function to be called on function success. NULL if callback is not needed\end{quote}
\begin{quote}Parameter failureCallback: Custom function to be called on function failure. NULL if callback is not needed\end{quote}
\begin{quote}Report: level updated\end{quote}

\paragraph{Command SwitchMultilevel StartLevelChange}
\begin{quote}Syntax: StartLevelChange(dir, duration = 0xff, ignoreStartLevel = TRUE, startLevel = 50, incdec = 0, step = 0xff, successCallback = NULL, failureCallback = NULL)\end{quote}
\begin{quote}Description: Send SwitchMultilevel StartLevelChange\end{quote}
\begin{quote}Parameter dir: Direction of change: 0 to incrase, 1 to decrase\end{quote}
\begin{quote}Parameter duration: Duration of change:. 0 instantly. 0x01...0x7f in seconds. 0x80...0xfe in minutes mapped to 1...127 (value 0x80=128 is 1 minute). 0xff use device factory default\end{quote}
\begin{quote}Parameter ignoreStartLevel: If set to True, device will ignore start level value and will use it's curent value\end{quote}
\begin{quote}Parameter startLevel: Start level to change from\end{quote}
\begin{quote}Parameter incdec: Increment/decrement type for step:. 0 Increment. 1 Decrement. 2 Reserved. 3 No Inc/Dec\end{quote}
\begin{quote}Parameter step: Step to be used in level change in percentage. 0...99 mapped to 1...100\%. 0xff uses device factory default\end{quote}
\begin{quote}Parameter successCallback: Custom function to be called on function success. NULL if callback is not needed\end{quote}
\begin{quote}Parameter failureCallback: Custom function to be called on function failure. NULL if callback is not needed\end{quote}
\begin{quote}Report: level updated\end{quote}

\paragraph{Command SwitchMultilevel StopLevelChange}
\begin{quote}Syntax: StopLevelChange(successCallback = NULL, failureCallback = NULL)\end{quote}
\begin{quote}Description: Send SwitchMultilevel StopLevelChange\end{quote}
\begin{quote}Parameter successCallback: Custom function to be called on function success. NULL if callback is not needed\end{quote}
\begin{quote}Parameter failureCallback: Custom function to be called on function failure. NULL if callback is not needed\end{quote}
\begin{quote}Report: level updated\end{quote}


\section{Command Class MultiChannelAssociation (0x8E/142)}

\textit{Version 3, Supported and Controlled}
\newline

This is an extention to the Association Command Class. It follows the same logic as the Association Command Class and has the same commands but accepts different instance values.
\newline

\noindent
Data holders:

\begin{itemize}
\item \textbf{groups}: Number of association groups in the device (can be smaller than the number of groups in Association)
\item \textbf{[groupId]}: Group subtree, where groupId = 1..groups
\item \qquad\textbf{max}: Number of nodes/instances the group can hold
\item \qquad\textbf{nodesInstances}: Array with nodes/instances in the group. Each pair is represented by two elements (node, instance).
\item \qquad\textbf{nodesInstancesToFollow}: Internal
\end{itemize}

\paragraph{Command MultiChannelAssociation Get}
\begin{quote}Syntax: Get(groupId = 0, successCallback = NULL, failureCallback = NULL)\end{quote}
\begin{quote}Description: Send MultiChannelAssociation Get\end{quote}
\begin{quote}Parameter groupId: Group Id (from 1 to 255). 0 requests all groups\end{quote}
\begin{quote}Parameter successCallback: Custom function to be called on function success. NULL if callback is not needed\end{quote}
\begin{quote}Parameter failureCallback: Custom function to be called on function failure. NULL if callback is not needed\end{quote}
\begin{quote}Report: Subtree corresponding to the group updated\end{quote}

\paragraph{Command MultiChannelAssociation Set}
\begin{quote}Syntax: Set(groupId, includeNode, includeInstance, successCallback = NULL, failureCallback = NULL)\end{quote}
\begin{quote}Description: Send MultiChannelAssociation Set (Add)\end{quote}
\begin{quote}Parameter groupId: Group Id (from 1 to 255)\end{quote}
\begin{quote}Parameter includeNode: Node to be added to the group\end{quote}
\begin{quote}Parameter includeInstance: Instance of the node to be added to the group\end{quote}
\begin{quote}Parameter successCallback: Custom function to be called on function success. NULL if callback is not needed\end{quote}
\begin{quote}Parameter failureCallback: Custom function to be called on function failure. NULL if callback is not needed\end{quote}
\begin{quote}Report: Subtree corresponding to the group updated\end{quote}

\paragraph{Command MultiChannelAssociation Remove}
\begin{quote}Syntax: Remove(groupId, excludeNode, excludeInstance, successCallback = NULL, failureCallback = NULL)\end{quote}
\begin{quote}Description: Send MultiChannelAssociation Remove\end{quote}
\begin{quote}Parameter groupId: Group Id (from 1 to 255)\end{quote}
\begin{quote}Parameter excludeNode: Node to be removed from the group\end{quote}
\begin{quote}Parameter excludeInstance: Instance of the node to be removed from the group\end{quote}
\begin{quote}Parameter successCallback: Custom function to be called on function success. NULL if callback is not needed\end{quote}
\begin{quote}Parameter failureCallback: Custom function to be called on function failure. NULL if callback is not needed\end{quote}
\begin{quote}Report: Subtree corresponding to the group updated\end{quote}

\paragraph{Command MultiChannelAssociation GroupingsGet}
\begin{quote}Syntax: GroupingsGet(successCallback = NULL, failureCallback = NULL)\end{quote}
\begin{quote}Description: Send MultiChannelAssociation GroupingsGet\end{quote}
\begin{quote}Parameter successCallback: Custom function to be called on function success. NULL if callback is not needed\end{quote}
\begin{quote}Parameter failureCallback: Custom function to be called on function failure. NULL if callback is not needed\end{quote}



\section{Command Class MultiChannel (0x60/96)}

\textit{Version 4, Supported and Controlled}
\newline

Allows to communicate with internal parts of device called channels or instances. Implemented transparently by the library.
\newline

\noindent
Data holders:

\begin{itemize}
\item \textbf{endPoints}: Number of endpoints
\item \qquad\textbf{[endPointId]}: Endpoint ID
\item \textbf{aggregated}: Number of aggregated endpoints
\item \qquad\textbf{[endPointId]}: Aggregated endpoint ID (numbering starts from endPoints + 1)
\item \textbf{dynamic}: Flag describing if endpoins are dynamic (their number and type can change over time)
\item \textbf{identical}: Internal. Flag describing if endpoins are  identical
\item \textbf{myInstance}: Internal
\item \textbf{doneIds}: Internal
\end{itemize}

\paragraph{Command MultiChannel Get}
\begin{quote}Syntax: Get(ccId, successCallback = NULL, failureCallback = NULL)\end{quote}
\begin{quote}Description: Send MultiChannel Get (MultiInstance V1 command). Reports number of channels supporting a defined Command Class. Depricated by MutliChannel V2, needed for old devices only\end{quote}
\begin{quote}Parameter ccId: Command Class Id in question\end{quote}
\begin{quote}Parameter successCallback: Custom function to be called on function success. NULL if callback is not needed\end{quote}
\begin{quote}Parameter failureCallback: Custom function to be called on function failure. NULL if callback is not needed\end{quote}


\paragraph{Command MultiChannel EndpointFind}
\begin{quote}Syntax: EndpointFind(generic, specific, successCallback = NULL, failureCallback = NULL)\end{quote}
\begin{quote}Description: Send MultiChannel Endpoint Find. Note that MultiChannel EndpointFind Report is not supported as useless. But one can still trap the response packet in logs\end{quote}
\begin{quote}Parameter generic: Generic type in search\end{quote}
\begin{quote}Parameter specific: Specific type in search\end{quote}
\begin{quote}Parameter successCallback: Custom function to be called on function success. NULL if callback is not needed\end{quote}
\begin{quote}Parameter failureCallback: Custom function to be called on function failure. NULL if callback is not needed\end{quote}


\paragraph{Command MultiChannel EndpointGet}
\begin{quote}Syntax: EndpointGet(successCallback = NULL, failureCallback = NULL)\end{quote}
\begin{quote}Description: Send MultiChannel Endpoint Get. Get the number of available endpoints\end{quote}
\begin{quote}Parameter successCallback: Custom function to be called on function success. NULL if callback is not needed\end{quote}
\begin{quote}Parameter failureCallback: Custom function to be called on function failure. NULL if callback is not needed\end{quote}


\paragraph{Command MultiChannel CapabilitiesGet}
\begin{quote}Syntax: CapabilitiesGet(endpoint, successCallback = NULL, failureCallback = NULL)\end{quote}
\begin{quote}Description: Send MultiChannel Capabilities Get. Request information about the specified endpoint\end{quote}
\begin{quote}Parameter endpoint: Endpoint in question\end{quote}
\begin{quote}Parameter successCallback: Custom function to be called on function success. NULL if callback is not needed\end{quote}
\begin{quote}Parameter failureCallback: Custom function to be called on function failure. NULL if callback is not needed\end{quote}


\paragraph{Command MultiChannel AggregatedMembersGet}
\begin{quote}Syntax: AggregatedMembersGet(endpoint, successCallback = NULL, failureCallback = NULL)\end{quote}
\begin{quote}Description: Send MultiChannel Aggregated Members Get. Request information about endpoints in the specified aggregated endpoint (v4 and above)\end{quote}
\begin{quote}Parameter endpoint: Aggregated endpoint in question\end{quote}
\begin{quote}Parameter successCallback: Custom function to be called on function success. NULL if callback is not needed\end{quote}
\begin{quote}Parameter failureCallback: Custom function to be called on function failure. NULL if callback is not needed\end{quote}



\section{Command Class Node Naming (0x77/119)}

\textit{Version 1, Supported and Controlled}
\newline

Allows assigning a readable string for a name and a location to a physical device. The two strings are stored inside the device and can be obtained upon request. There are no restrictions to the name except the maximum length up to 16 characters.
\newline

\noindent
Data holders:

\begin{itemize}
\item \textbf{nodename}: Node name
\item \textbf{nameEncoding}: NodeName encoding
\item \textbf{location}: Location
\item \textbf{locationEncoding}: Location encoding
\end{itemize}

\paragraph{Command NodeNaming Get}
\begin{quote}Syntax: Get(successCallback = NULL, failureCallback = NULL)\end{quote}
\begin{quote}Description: Send NodeNaming GetName and GetLocation\end{quote}
\begin{quote}Parameter successCallback: Custom function to be called on function success. NULL if callback is not needed\end{quote}
\begin{quote}Parameter failureCallback: Custom function to be called on function failure. NULL if callback is not needed\end{quote}
\begin{quote}Report: nodename, nameEncoding, location and locationEncoding updated\end{quote}

\paragraph{Command NodeNaming GetName}
\begin{quote}Syntax: GetName(successCallback = NULL, failureCallback = NULL)\end{quote}
\begin{quote}Description: Send NodeNaming GetName\end{quote}
\begin{quote}Parameter successCallback: Custom function to be called on function success. NULL if callback is not needed\end{quote}
\begin{quote}Parameter failureCallback: Custom function to be called on function failure. NULL if callback is not needed\end{quote}
\begin{quote}Report: nodename and nameEncoding updated\end{quote}

\paragraph{Command NodeNaming GetLocation}
\begin{quote}Syntax: GetLocation(successCallback = NULL, failureCallback = NULL)\end{quote}
\begin{quote}Description: Send NodeNaming GetLocation\end{quote}
\begin{quote}Parameter successCallback: Custom function to be called on function success. NULL if callback is not needed\end{quote}
\begin{quote}Parameter failureCallback: Custom function to be called on function failure. NULL if callback is not needed\end{quote}
\begin{quote}Report: location and locationEncoding updated\end{quote}

\paragraph{Command NodeNaming SetName}
\begin{quote}Syntax: SetName(name, successCallback = NULL, failureCallback = NULL)\end{quote}
\begin{quote}Description: Send NodeNaming SetName\end{quote}
\begin{quote}Parameter name: Value\end{quote}
\begin{quote}Parameter successCallback: Custom function to be called on function success. NULL if callback is not needed\end{quote}
\begin{quote}Parameter failureCallback: Custom function to be called on function failure. NULL if callback is not needed\end{quote}
\begin{quote}Report: nodename and nameEncoding updated\end{quote}

\paragraph{Command NodeNaming SetLocation}
\begin{quote}Syntax: SetLocation(location, successCallback = NULL, failureCallback = NULL)\end{quote}
\begin{quote}Description: Send NodeNaming SetLocation\end{quote}
\begin{quote}Parameter location: Value\end{quote}
\begin{quote}Parameter successCallback: Custom function to be called on function success. NULL if callback is not needed\end{quote}
\begin{quote}Parameter failureCallback: Custom function to be called on function failure. NULL if callback is not needed\end{quote}
\begin{quote}Report: location and locationEncoding updated\end{quote}


\section{Command Class Thermostat SetPoint (0x43/67)}

\textit{Version 3, Controlled}
\newline

Allows to set a certain setpoint to a thermostat (set temperature to maintain). The command class can be applied to different kind of thermostats (heating, cooling, ...), hence it has various modes.
\newline

\noindent
Data holders:

\begin{itemize}
\item \textbf{[modeId]}: Subtree for mode
\item \qquad\textbf{modeName}: Mode description
\item \qquad\textbf{scale}: Scale Id
\item \qquad\textbf{scaleString}: Scale description
\item \qquad\textbf{val}: Temperature to maintain
\item \qquad\textbf{setVal}: Last set temperature to maintain (might differ from val until thermostat wakeup)
\item \qquad\textbf{min}: Minimal temperature value supported by the device
\item \qquad\textbf{max}: Maximal temperature value supported by the device
\item \qquad\textbf{size}: Internal. Size of the value (1, 2 or 4 bytes)
\item \qquad\textbf{precision}: Internal. Precision used in value (number of digits after decimal dot)
\item \qquad\textbf{deviceScale}: Internal. Scale Id on the device side (if local conversion is used, like C->F)
\item \qquad\textbf{deviceScaleString}: Internal. Scale description of the device
\item \textbf{modemask}: Internal. Bit mask with supported modes
\item \textbf{danfossBugFlag}: Internal
\end{itemize}

\paragraph{Command ThermostatSetPoint Get}
\begin{quote}Syntax: Get(mode = -1, successCallback = NULL, failureCallback = NULL)\end{quote}
\begin{quote}Description: Send ThermostatSetPoint Get\end{quote}
\begin{quote}Parameter mode: Thermostat Mode. -1 requests for all modes\end{quote}
\begin{quote}Parameter successCallback: Custom function to be called on function success. NULL if callback is not needed\end{quote}
\begin{quote}Parameter failureCallback: Custom function to be called on function failure. NULL if callback is not needed\end{quote}
\begin{quote}Report: modeId subtree updated\end{quote}

\paragraph{Command ThermostatSetPoint Set}
\begin{quote}Syntax: Set(mode, value, successCallback = NULL, failureCallback = NULL)\end{quote}
\begin{quote}Description: Send ThermostatSetPoint Set\end{quote}
\begin{quote}Parameter mode: Thermostat Mode\end{quote}
\begin{quote}Parameter value: temperature\end{quote}
\begin{quote}Parameter successCallback: Custom function to be called on function success. NULL if callback is not needed\end{quote}
\begin{quote}Parameter failureCallback: Custom function to be called on function failure. NULL if callback is not needed\end{quote}
\begin{quote}Report: modeId subtree updated\end{quote}


\section{Command Class Thermostat Mode (0x40/64)}

\textit{Version 3, Controlled}
\newline

Allows to switch a heating/cooling actuator in different modes.
\newline

\noindent
Data holders:

\begin{itemize}
\item \textbf{modemask}: Internal. Bit mask with supported modes
\item \textbf{mode}: Current mode
\item \textbf{[modeId]}: Mode subtree
\item \qquad\textbf{modeName}: Mode description
\end{itemize}

\paragraph{Command ThermostatMode Get}
\begin{quote}Syntax: Get(successCallback = NULL, failureCallback = NULL)\end{quote}
\begin{quote}Description: Send ThermostatMode Get\end{quote}
\begin{quote}Parameter successCallback: Custom function to be called on function success. NULL if callback is not needed\end{quote}
\begin{quote}Parameter failureCallback: Custom function to be called on function failure. NULL if callback is not needed\end{quote}


\paragraph{Command ThermostatMode Set}
\begin{quote}Syntax: Set(mode, successCallback = NULL, failureCallback = NULL)\end{quote}
\begin{quote}Description: Send ThermostatMode Set\end{quote}
\begin{quote}Parameter mode: Thermostat Mode\end{quote}
\begin{quote}Parameter successCallback: Custom function to be called on function success. NULL if callback is not needed\end{quote}
\begin{quote}Parameter failureCallback: Custom function to be called on function failure. NULL if callback is not needed\end{quote}



\section{Command Class Thermostat Fan Mode (0x44/68)}

\textit{Version 4, Controlled}
\newline

Allows to controls fan modes in thermostats.
\newline

\noindent
Data holders:

\begin{itemize}
\item \textbf{modemask}: Internal. Bit mask with supported modes
\item \textbf{mode}: Current mode
\item \textbf{[modeId]}: Mode subtree
\item \qquad\textbf{modeName}: Mode description
\item \textbf{on}: Reports if fan is currently On (True) or Off (False)
\end{itemize}

\paragraph{Command ThermostatFanMode Get}
\begin{quote}Syntax: Get(successCallback = NULL, failureCallback = NULL)\end{quote}
\begin{quote}Description: Send ThermostatFanMode Get\end{quote}
\begin{quote}Parameter successCallback: Custom function to be called on function success. NULL if callback is not needed\end{quote}
\begin{quote}Parameter failureCallback: Custom function to be called on function failure. NULL if callback is not needed\end{quote}
\begin{quote}Report: mode and on updated\end{quote}

\paragraph{Command ThermostatFanMode Set}
\begin{quote}Syntax: Set(on, mode, successCallback = NULL, failureCallback = NULL)\end{quote}
\begin{quote}Description: Send ThermostatFanMode Set\end{quote}
\begin{quote}Parameter on: TRUE to turn fan on (and set mode), FALSE to comletely turn off (mode is ignored)\end{quote}
\begin{quote}Parameter mode: Thermostat Fan Mode\end{quote}
\begin{quote}Parameter successCallback: Custom function to be called on function success. NULL if callback is not needed\end{quote}
\begin{quote}Parameter failureCallback: Custom function to be called on function failure. NULL if callback is not needed\end{quote}
\begin{quote}Report: mode and on updated\end{quote}


\section{Command Class Thermostat Fan State (0x45/69)}

\textit{Version 2, Controlled}
\newline

Allows to determine the operating state of the fan. V2 is not yet implemented.
\newline

\noindent
Data holders:

\begin{itemize}
\item \textbf{state}: Fan current state (0 Off, 1 Running)
\end{itemize}

\paragraph{Command ThermostatFanState Get}
\begin{quote}Syntax: Get(successCallback = NULL, failureCallback = NULL)\end{quote}
\begin{quote}Description: Send ThermostatFanState Get\end{quote}
\begin{quote}Parameter successCallback: Custom function to be called on function success. NULL if callback is not needed\end{quote}
\begin{quote}Parameter failureCallback: Custom function to be called on function failure. NULL if callback is not needed\end{quote}
\begin{quote}Report: state and on updated\end{quote}


\section{Command Class Thermostat Operating State (0x42/66)}

\textit{Version 2, Controlled}
\newline

Allows to determine the operating state of the thermostat and state change history.
\newline

\noindent
Data holders:

\begin{itemize}
\item \textbf{state}: Current operation state
\item \textbf{statemask}: Internal. Bit mask of supported logs for each state
\item \textbf{[stateId]}: Subtree with state log info
\item \qquad\textbf{today}: Number of minutes thermostat was in this state today (since 0:00)
\item \qquad\textbf{yesterday}: Number of minutes thermostat was in this state yesterday (since 0:00)
\end{itemize}

\paragraph{Command ThermostatOperatingState Get}
\begin{quote}Syntax: Get(successCallback = NULL, failureCallback = NULL)\end{quote}
\begin{quote}Description: Send ThermostatOperatingState Get\end{quote}
\begin{quote}Parameter successCallback: Custom function to be called on function success. NULL if callback is not needed\end{quote}
\begin{quote}Parameter failureCallback: Custom function to be called on function failure. NULL if callback is not needed\end{quote}
\begin{quote}Report: state updated\end{quote}

\paragraph{Command ThermostatOperatingState LoggingGet}
\begin{quote}Syntax: LoggingGet(state, successCallback = NULL, failureCallback = NULL)\end{quote}
\begin{quote}Description: Send ThermostatOperatingState Logging Get\end{quote}
\begin{quote}Parameter state: State number to get logging for. 0 to get log for all supported states\end{quote}
\begin{quote}Parameter successCallback: Custom function to be called on function success. NULL if callback is not needed\end{quote}
\begin{quote}Parameter failureCallback: Custom function to be called on function failure. NULL if callback is not needed\end{quote}
\begin{quote}Report: stateId subtree updated updated\end{quote}


\section{Command Class Alarm Sensor (0x9C/156)}

\textit{Version 1, Controlled}
\newline

Deprecated Command Class. Now Alarm/Notification is used instead.
\newline

\noindent
Data holders:

\begin{itemize}
\item \textbf{alarmMap}: Internal. Bit mask of supported alarm types
\item \textbf{alarms}: Unused
\item \textbf{[alarmTypeId]}: Alarm type subtree
\item \qquad\textbf{srcId}: Source of event
\item \qquad\textbf{sensorState}: Alarm state
\item \qquad\textbf{sensorTime}: Alarm time (according to the sender)
\item \qquad\textbf{typeString}: Name of alarm type
\end{itemize}

\paragraph{Command AlarmSensor SupportedGet}
\begin{quote}Syntax: SupportedGet(successCallback = NULL, failureCallback = NULL)\end{quote}
\begin{quote}Description: Send AlarmSensor SupportedGet\end{quote}
\begin{quote}Parameter successCallback: Custom function to be called on function success. NULL if callback is not needed\end{quote}
\begin{quote}Parameter failureCallback: Custom function to be called on function failure. NULL if callback is not needed\end{quote}
\begin{quote}Report: List of supported types updated\end{quote}

\paragraph{Command AlarmSensor Get}
\begin{quote}Syntax: Get(type = -1, successCallback = NULL, failureCallback = NULL)\end{quote}
\begin{quote}Description: Send AlarmSensor Get. Requests the status of the alarm sensor of a given type\end{quote}
\begin{quote}Parameter type: Alarm type to get. -1 means get all types\end{quote}
\begin{quote}Parameter successCallback: Custom function to be called on function success. NULL if callback is not needed\end{quote}
\begin{quote}Parameter failureCallback: Custom function to be called on function failure. NULL if callback is not needed\end{quote}
\begin{quote}Report: Alarm type subtree updated\end{quote}


\section{Command Class Door Lock (0x62/98)}

\textit{Version 2, Controlled}
\newline

Allows to operate an electronic door lock. This Command Class is ALWAYS encapsulated in Security. Door lock modes are the following:. 

- 0x00 Door Unsecured (Open). 

- 0x01 Door Unsecured with timeout. 

- 0x10 Door Unsecured for inside Door Handles. 

- 0x11 Door Unsecured for inside Door Handles with timeout. 

- 0x20 Door Unsecured for outside Door Handles. 

- 0x21 Door Unsecured for outside Door Handles with timeout. 

- 0xFE Door/Lock Mode Unknown (bolt not fully retracted/engaged). 

- 0xFF Door Secured (closed).
\newline

\noindent
Data holders:

\begin{itemize}
\item \textbf{mode}: Operating mode of the lock
\item \textbf{insideMode}: Bit mask describing if a specific handles (1..4) can open the door from inside
\item \textbf{outsideMode}: Bit mask describing if a specific handles (1..4) can open the door from outside
\item \textbf{lockMinutes}: Time remaind before autolock (minutes, 0xFE for no autolock)
\item \textbf{lockSeconds}: Time remaind before autolock (seconds, 0xFE for no autolock)
\item \textbf{condition}: Bit mask describing lock components: bit 0: Door Open(0)/Close(1), bit 1: Bolt Locked(0)/Unlocked(1), bit 2: Latch Open(0)/Close(1)
\item \textbf{insideState}: Bit mask describing if a specific handles (1..4) can open the door from inside
\item \textbf{outsideState}: Bit mask describing if a specific handles (1..4) can open the door from outside
\item \textbf{timeoutMinutes}: Timeout for autolock (minutes, 0xFE for no autolock)
\item \textbf{timeoutSeconds}: Timeout for autolock (seconds, 0xFE for no autolock)
\item \textbf{opType}: 0x01 for constant operation, 0x02 for autolock
\end{itemize}

\paragraph{Command DoorLock Get}
\begin{quote}Syntax: Get(successCallback = NULL, failureCallback = NULL)\end{quote}
\begin{quote}Description: Send DoorLock Get\end{quote}
\begin{quote}Parameter successCallback: Custom function to be called on function success. NULL if callback is not needed\end{quote}
\begin{quote}Parameter failureCallback: Custom function to be called on function failure. NULL if callback is not needed\end{quote}
\begin{quote}Report: mode, insideMode, outsideMode, lockMinutes, lockSeconds and condition updated\end{quote}

\paragraph{Command DoorLock ConfigurationGet}
\begin{quote}Syntax: ConfigurationGet(successCallback = NULL, failureCallback = NULL)\end{quote}
\begin{quote}Description: Send DoorLock Configuration Get\end{quote}
\begin{quote}Parameter successCallback: Custom function to be called on function success. NULL if callback is not needed\end{quote}
\begin{quote}Parameter failureCallback: Custom function to be called on function failure. NULL if callback is not needed\end{quote}
\begin{quote}Report: insideState, outsideState, timeoutMinutes, timeoutSeconds, opType updated\end{quote}

\paragraph{Command DoorLock Set}
\begin{quote}Syntax: Set(mode, successCallback = NULL, failureCallback = NULL)\end{quote}
\begin{quote}Description: Send DoorLock Set\end{quote}
\begin{quote}Parameter mode: Lock mode\end{quote}
\begin{quote}Parameter successCallback: Custom function to be called on function success. NULL if callback is not needed\end{quote}
\begin{quote}Parameter failureCallback: Custom function to be called on function failure. NULL if callback is not needed\end{quote}
\begin{quote}Report: mode, insideMode, outsideMode, lockMinutes, lockSeconds and condition updated\end{quote}

\paragraph{Command DoorLock ConfigurationSet}
\begin{quote}Syntax: ConfigurationSet(opType, outsideState, insideState, lockMin, lockSec, successCallback = NULL, failureCallback = NULL)\end{quote}
\begin{quote}Description: Send DoorLock Configuration Set\end{quote}
\begin{quote}Parameter opType: Operation type\end{quote}
\begin{quote}Parameter outsideState: State of outside door handle\end{quote}
\begin{quote}Parameter insideState: State of inside door handle\end{quote}
\begin{quote}Parameter lockMin: Lock after a specified time (minutes part)\end{quote}
\begin{quote}Parameter lockSec: Lock after a specified time (seconds part)\end{quote}
\begin{quote}Parameter successCallback: Custom function to be called on function success. NULL if callback is not needed\end{quote}
\begin{quote}Parameter failureCallback: Custom function to be called on function failure. NULL if callback is not needed\end{quote}
\begin{quote}Report: insideState, outsideState, timeoutMinutes, timeoutSeconds, opType updated\end{quote}


\section{Command Class Door Lock Logging (0x4C/76)}

\textit{Version 1, Controlled}
\newline

Allows to receive reports about all successful and failed activities of the electronic door lock. Event types are the following:. 

- 1 Lock Command: Keypad access code verified lock command. 

- 2 Unlock Command: Keypad access code verified unlock command. 

- 3 Lock Command: Keypad lock button pressed. 

- 4 Unlock command: Keypad unlock button pressed. 

- 5 Lock Command: Keypad access code out of schedule. 

- 6 Unlock Command: Keypad access code out of schedule. 

- 7 Keypad illegal access code entered. 

- 8 Key or latch operation locked (manual). 

- 9 Key or latch operation unlocked (manual). 

- 10 Auto lock operation. 

- 11 Auto unlock operation. 

- 12 Lock Command: Z-Wave access code verified. 

- 13 Unlock Command: Z-Wave access code verified. 

- 14 Lock Command: Z-Wave (no code). 

- 15 Unlock Command: Z-Wave (no code). 

- 16 Lock Command: Z-Wave access code out of schedule. 

- 17 Unlock Command Z-Wave access code out of schedule. 

- 18 Z-Wave illegal access code entered. 

- 19 Key or latch operation locked (manual). 

- 20 Key or latch operation unlocked (manual). 

- 21 Lock secured. 

- 22 Lock unsecured. 

- 23 User code added. 

- 24 User code deleted. 

- 25 All user codes deleted. 

- 26 Master code changed. 

- 27 User code changed. 

- 28 Lock reset. 

- 29 Configuration changed. 

- 30 Low battery. 

- 31 New Battery installed.
\newline

\noindent
Data holders:

\begin{itemize}
\item \textbf{maxRecords}: Maximum number of records the lock can store. Olded records are reused first.
\item \textbf{[recordId]}: Subtree storing log record
\item \qquad\textbf{time}: Time of the event
\item \qquad\textbf{event}: Event type
\item \qquad\textbf{uId}: UserID (from UserCode Command Class)
\item \qquad\textbf{eventString}: Event type description
\end{itemize}

\paragraph{Command DoorLockLogging Get}
\begin{quote}Syntax: Get(record = 0, successCallback = NULL, failureCallback = NULL)\end{quote}
\begin{quote}Description: Send DoorLockLogging Get\end{quote}
\begin{quote}Parameter record: Record number to get, or 0 to get last records\end{quote}
\begin{quote}Parameter successCallback: Custom function to be called on function success. NULL if callback is not needed\end{quote}
\begin{quote}Parameter failureCallback: Custom function to be called on function failure. NULL if callback is not needed\end{quote}
\begin{quote}Report: record subtree updated\end{quote}


\section{Command Class User Code (0x63/99)}

\textit{Version 1, Controlled}
\newline

Allows to define individual user entry code in electrnic door lock.
\newline

\noindent
Data holders:

\begin{itemize}
\item \textbf{maxUsers}: Maximum number of supported users
\item \textbf{[userId]}: User subtree
\item \qquad\textbf{code}: User code
\item \qquad\textbf{status}: Status of the user: 0 for available (no code set), 1 for occupied (code set), 2 for reserved by administrator
\item \qquad\textbf{hasCode}: Flag if a valid code is set (in case device reports occupied, but code is not valid (less than 4 symbols) or code not set but old is still reported by the device)
\end{itemize}

\paragraph{Command UserCode Get}
\begin{quote}Syntax: Get(user = -1, successCallback = NULL, failureCallback = NULL)\end{quote}
\begin{quote}Description: Send UserCode Get\end{quote}
\begin{quote}Parameter user: User index to get code for (1...maxUsers). -1 to get codes for all users\end{quote}
\begin{quote}Parameter successCallback: Custom function to be called on function success. NULL if callback is not needed\end{quote}
\begin{quote}Parameter failureCallback: Custom function to be called on function failure. NULL if callback is not needed\end{quote}
\begin{quote}Report: userId subtree updated\end{quote}

\paragraph{Command UserCode Set}
\begin{quote}Syntax: Set(user, code, status, successCallback = NULL, failureCallback = NULL)\end{quote}
\begin{quote}Description: Send UserCode Set\end{quote}
\begin{quote}Parameter user: User index to set code for (1...maxUsers). 0 means set for all users\end{quote}
\begin{quote}Parameter code: Code to set (4...10 characters long)\end{quote}
\begin{quote}Parameter status: Code status to set\end{quote}
\begin{quote}Parameter successCallback: Custom function to be called on function success. NULL if callback is not needed\end{quote}
\begin{quote}Parameter failureCallback: Custom function to be called on function failure. NULL if callback is not needed\end{quote}
\begin{quote}Report: userId subtree updated\end{quote}

\paragraph{Command UserCode SetRaw}
\begin{quote}Syntax: SetRaw(user, code, status, successCallback = NULL, failureCallback = NULL)\end{quote}
\begin{quote}Description: Send UserCode Set (raw)\end{quote}
\begin{quote}Parameter user: User index to set code for (1...maxUsers). 0 means set for all users\end{quote}
\begin{quote}Parameter code: Code to set (4...10 bytes long)\end{quote}
\begin{quote}Parameter status: Code status to set\end{quote}
\begin{quote}Parameter successCallback: Custom function to be called on function success. NULL if callback is not needed\end{quote}
\begin{quote}Parameter failureCallback: Custom function to be called on function failure. NULL if callback is not needed\end{quote}
\begin{quote}Report: userId subtree updated\end{quote}


\section{Command Class Time (0x8A/138)}

\textit{Version 2, Supported and Controlled}
\newline

Allows to report to devices in Z-Wave network time and date as well as time zone offset and daylight savings parameters. The data formats are based on the International Standard ISO 8601.
\paragraph{Command Time TimeGet}
\begin{quote}Syntax: TimeGet(successCallback = NULL, failureCallback = NULL)\end{quote}
\begin{quote}Description: Send Time TimeGet\end{quote}
\begin{quote}Parameter successCallback: Custom function to be called on function success. NULL if callback is not needed\end{quote}
\begin{quote}Parameter failureCallback: Custom function to be called on function failure. NULL if callback is not needed\end{quote}


\paragraph{Command Time DateGet}
\begin{quote}Syntax: DateGet(successCallback = NULL, failureCallback = NULL)\end{quote}
\begin{quote}Description: Send Time DateGet\end{quote}
\begin{quote}Parameter successCallback: Custom function to be called on function success. NULL if callback is not needed\end{quote}
\begin{quote}Parameter failureCallback: Custom function to be called on function failure. NULL if callback is not needed\end{quote}


\paragraph{Command Time OffsetGet}
\begin{quote}Syntax: OffsetGet(successCallback = NULL, failureCallback = NULL)\end{quote}
\begin{quote}Description: Send Time TimeOffsetGet\end{quote}
\begin{quote}Parameter successCallback: Custom function to be called on function success. NULL if callback is not needed\end{quote}
\begin{quote}Parameter failureCallback: Custom function to be called on function failure. NULL if callback is not needed\end{quote}



\section{Command Class Time Parameters (0x8B/139)}

\textit{Version 1, Controlled}
\newline

Used to set date and time. Time zone offset and daylight savings may be set in the Time Command Class. The data formats are based on the International Standard ISO 8601.
\paragraph{Command TimeParameters Get}
\begin{quote}Syntax: Get(successCallback = NULL, failureCallback = NULL)\end{quote}
\begin{quote}Description: Send TimeParameters Get\end{quote}
\begin{quote}Parameter successCallback: Custom function to be called on function success. NULL if callback is not needed\end{quote}
\begin{quote}Parameter failureCallback: Custom function to be called on function failure. NULL if callback is not needed\end{quote}


\paragraph{Command TimeParameters Set}
\begin{quote}Syntax: Set(successCallback = NULL, failureCallback = NULL)\end{quote}
\begin{quote}Description: Send TimeParameters Set\end{quote}
\begin{quote}Parameter successCallback: Custom function to be called on function success. NULL if callback is not needed\end{quote}
\begin{quote}Parameter failureCallback: Custom function to be called on function failure. NULL if callback is not needed\end{quote}



\section{Command Class Clock (0x81/129)}

\textit{Version 1, Supported and Controlled}
\newline

Sync clock on the device with controller system clock.
\paragraph{Command Clock Get}
\begin{quote}Syntax: Get(successCallback = NULL, failureCallback = NULL)\end{quote}
\begin{quote}Description: Send Clock Get\end{quote}
\begin{quote}Parameter successCallback: Custom function to be called on function success. NULL if callback is not needed\end{quote}
\begin{quote}Parameter failureCallback: Custom function to be called on function failure. NULL if callback is not needed\end{quote}
\begin{quote}Report: reported value ignored\end{quote}

\paragraph{Command Clock Set}
\begin{quote}Syntax: Set(successCallback = NULL, failureCallback = NULL)\end{quote}
\begin{quote}Description: Send Clock Set\end{quote}
\begin{quote}Parameter successCallback: Custom function to be called on function success. NULL if callback is not needed\end{quote}
\begin{quote}Parameter failureCallback: Custom function to be called on function failure. NULL if callback is not needed\end{quote}



\section{Command Class Scene Activation (0x2B/43)}

\textit{Version 1, Supported and Controlled}
\newline

Allows to activate scenes on devices and trap scene activation events from remotes.
\newline

\noindent
Data holders:

\begin{itemize}
\item \textbf{currentScene}: Scene activated from remote
\item \textbf{dimmingDuration}: Dimming duration for the activated scene
\end{itemize}

\paragraph{Command SceneActivation Set}
\begin{quote}Syntax: Set(sceneId, dimmingDuration = 0xff, successCallback = NULL, failureCallback = NULL)\end{quote}
\begin{quote}Description: Send SceneActivation Set\end{quote}
\begin{quote}Parameter sceneId: Scene Id\end{quote}
\begin{quote}Parameter dimmingDuration: Dimming duration\end{quote}
\begin{quote}Parameter successCallback: Custom function to be called on function success. NULL if callback is not needed\end{quote}
\begin{quote}Parameter failureCallback: Custom function to be called on function failure. NULL if callback is not needed\end{quote}



\section{Command Class Scene Controller Conf (0x2D/45)}

\textit{Version 1, Controlled}
\newline

Allows to set scene Id to be activated using SceneActivation Command Class on a remote.
\newline

\noindent
Data holders:

\begin{itemize}
\item \textbf{[groupId]}: Subtree for a given association group number (defined by Association Command Class)
\item \qquad\textbf{scene}: Scene to activate for all devices in the group
\item \qquad\textbf{duration}: Duration for scene activation
\end{itemize}

\paragraph{Command SceneControllerConf Get}
\begin{quote}Syntax: Get(group = 0, successCallback = NULL, failureCallback = NULL)\end{quote}
\begin{quote}Description: Send SceneControllerConf Get\end{quote}
\begin{quote}Parameter group: Group Id. 0 requests all groups\end{quote}
\begin{quote}Parameter successCallback: Custom function to be called on function success. NULL if callback is not needed\end{quote}
\begin{quote}Parameter failureCallback: Custom function to be called on function failure. NULL if callback is not needed\end{quote}
\begin{quote}Report: group subtree updated\end{quote}

\paragraph{Command SceneControllerConf Set}
\begin{quote}Syntax: Set(group, scene, duration = 0x0, successCallback = NULL, failureCallback = NULL)\end{quote}
\begin{quote}Description: Send SceneControllerConf Set\end{quote}
\begin{quote}Parameter group: Group Id\end{quote}
\begin{quote}Parameter scene: Scene Id\end{quote}
\begin{quote}Parameter duration: Duration\end{quote}
\begin{quote}Parameter successCallback: Custom function to be called on function success. NULL if callback is not needed\end{quote}
\begin{quote}Parameter failureCallback: Custom function to be called on function failure. NULL if callback is not needed\end{quote}
\begin{quote}Report: group subtree updated\end{quote}


\section{Command Class Scene Actuator Conf (0x2C/44)}

\textit{Version 1, Controlled}
\newline

Allows to configure actuators to set specified level on a given scene activation by SceneActivation Command Class.
\newline

\noindent
Data holders:

\begin{itemize}
\item \textbf{[sceneId]}: Subtree for scene
\item \qquad\textbf{level}: Level to set on scene activation
\item \qquad\textbf{dimming}: Default dimming duration to use
\item \textbf{currentScene}: Currently activated scene
\end{itemize}

\paragraph{Command SceneActuatorConf Get}
\begin{quote}Syntax: Get(scene = 0, successCallback = NULL, failureCallback = NULL)\end{quote}
\begin{quote}Description: Send SceneActuatorConf Get\end{quote}
\begin{quote}Parameter scene: Scene Id. 0 means get current scene\end{quote}
\begin{quote}Parameter successCallback: Custom function to be called on function success. NULL if callback is not needed\end{quote}
\begin{quote}Parameter failureCallback: Custom function to be called on function failure. NULL if callback is not needed\end{quote}
\begin{quote}Report: scene subtree updated, currentScene updated (if scene = 0)\end{quote}

\paragraph{Command SceneActuatorConf Set}
\begin{quote}Syntax: Set(scene, level, dimming = 0xff, override = TRUE, successCallback = NULL, failureCallback = NULL)\end{quote}
\begin{quote}Description: Send SceneActuatorConf Set\end{quote}
\begin{quote}Parameter scene: Scene Id\end{quote}
\begin{quote}Parameter level: Level\end{quote}
\begin{quote}Parameter dimming: Dimming\end{quote}
\begin{quote}Parameter override: If false then the current settings in the device is associated with the Scene Id. If true then the Level value is used\end{quote}
\begin{quote}Parameter successCallback: Custom function to be called on function success. NULL if callback is not needed\end{quote}
\begin{quote}Parameter failureCallback: Custom function to be called on function failure. NULL if callback is not needed\end{quote}
\begin{quote}Report: scene subtree updated\end{quote}


\section{Command Class Indicator (0x87/135)}

\textit{Version 1, Controlled}
\newline

Operates the indicator on the device if available. Can be used to identify a device or use the indicator for special purposes (show away/at home mode).
\newline

\noindent
Data holders:

\begin{itemize}
\item \textbf{stat}: Status of the indicator
\end{itemize}

\paragraph{Command Indicator Get}
\begin{quote}Syntax: Get(successCallback = NULL, failureCallback = NULL)\end{quote}
\begin{quote}Description: Send Indicator Get\end{quote}
\begin{quote}Parameter successCallback: Custom function to be called on function success. NULL if callback is not needed\end{quote}
\begin{quote}Parameter failureCallback: Custom function to be called on function failure. NULL if callback is not needed\end{quote}
\begin{quote}Report: stat updated\end{quote}

\paragraph{Command Indicator Set}
\begin{quote}Syntax: Set(val, successCallback = NULL, failureCallback = NULL)\end{quote}
\begin{quote}Description: Send Indicator Set\end{quote}
\begin{quote}Parameter val: Value to set\end{quote}
\begin{quote}Parameter successCallback: Custom function to be called on function success. NULL if callback is not needed\end{quote}
\begin{quote}Parameter failureCallback: Custom function to be called on function failure. NULL if callback is not needed\end{quote}
\begin{quote}Report: stat updated\end{quote}


\section{Command Class Protection (0x75/117)}

\textit{Version 2, Controlled}
\newline

Allows to disable local and RF control of the device.
\newline

\noindent
Data holders:

\begin{itemize}
\item \textbf{state}: Local control state (0 = Unprotected, 1 = Protected by sequence, 2 = Protected)
\item \textbf{rfState}: Control via RF state (0 = Unprotected, 1 = No RF control, 2 = No RF response at all)
\item \textbf{exclusive}: Flag describing if exclusive control via RF is supported
\item \textbf{timeout}: Flag describing if timeout of protection of control via RF is supported
\item \textbf{stateCap}: Requires Z-Wave specification re-read. Please contact Z-Wave.Me support
\item \textbf{rfStateCap}: Requires Z-Wave specification re-read. Please contact Z-Wave.Me support
\item \textbf{exclusiveCap}: Requires Z-Wave specification re-read. Please contact Z-Wave.Me support
\item \textbf{timeoutCap}: Requires Z-Wave specification re-read. Please contact Z-Wave.Me support
\end{itemize}

\paragraph{Command Protection Get}
\begin{quote}Syntax: Get(successCallback = NULL, failureCallback = NULL)\end{quote}
\begin{quote}Description: Send Protection Get\end{quote}
\begin{quote}Parameter successCallback: Custom function to be called on function success. NULL if callback is not needed\end{quote}
\begin{quote}Parameter failureCallback: Custom function to be called on function failure. NULL if callback is not needed\end{quote}
\begin{quote}Report: state, rfState updated\end{quote}

\paragraph{Command Protection Set}
\begin{quote}Syntax: Set(state, rfState = 0, successCallback = NULL, failureCallback = NULL)\end{quote}
\begin{quote}Description: Send Protection Set\end{quote}
\begin{quote}Parameter state: Local control protection state\end{quote}
\begin{quote}Parameter rfState: RF control protection state\end{quote}
\begin{quote}Parameter successCallback: Custom function to be called on function success. NULL if callback is not needed\end{quote}
\begin{quote}Parameter failureCallback: Custom function to be called on function failure. NULL if callback is not needed\end{quote}
\begin{quote}Report: state and rfState updated\end{quote}

\paragraph{Command Protection ExclusiveGet}
\begin{quote}Syntax: ExclusiveGet(successCallback = NULL, failureCallback = NULL)\end{quote}
\begin{quote}Description: Send Protection Exclusive Control Get\end{quote}
\begin{quote}Parameter successCallback: Custom function to be called on function success. NULL if callback is not needed\end{quote}
\begin{quote}Parameter failureCallback: Custom function to be called on function failure. NULL if callback is not needed\end{quote}


\paragraph{Command Protection ExclusiveSet}
\begin{quote}Syntax: ExclusiveSet(controlNodeId, successCallback = NULL, failureCallback = NULL)\end{quote}
\begin{quote}Description: Send Protection Exclusive Control Set\end{quote}
\begin{quote}Parameter controlNodeId: Node Id to have exclusive control over destination node\end{quote}
\begin{quote}Parameter successCallback: Custom function to be called on function success. NULL if callback is not needed\end{quote}
\begin{quote}Parameter failureCallback: Custom function to be called on function failure. NULL if callback is not needed\end{quote}


\paragraph{Command Protection TimeoutGet}
\begin{quote}Syntax: TimeoutGet(successCallback = NULL, failureCallback = NULL)\end{quote}
\begin{quote}Description: Send Protection Timeout Get\end{quote}
\begin{quote}Parameter successCallback: Custom function to be called on function success. NULL if callback is not needed\end{quote}
\begin{quote}Parameter failureCallback: Custom function to be called on function failure. NULL if callback is not needed\end{quote}


\paragraph{Command Protection TimeoutSet}
\begin{quote}Syntax: TimeoutSet(timeout, successCallback = NULL, failureCallback = NULL)\end{quote}
\begin{quote}Description: Send Protection Timeout Set\end{quote}
\begin{quote}Parameter timeout: Timeout in seconds. 0 is no timer set. -1 is infinite timeout. max value is 191 minute (11460 seconds). values above 1 minute are rounded to 1 minute boundary\end{quote}
\begin{quote}Parameter successCallback: Custom function to be called on function success. NULL if callback is not needed\end{quote}
\begin{quote}Parameter failureCallback: Custom function to be called on function failure. NULL if callback is not needed\end{quote}



\section{Command Class Schedule Entry Lock (0x4E/78)}

\textit{Version 3, Controlled}
\newline

Allows to define individual time intervals for access to a door lock per user. Refers to users defined by User Code Command Class.
\newline

\noindent
Data holders:

\begin{itemize}
\item \textbf{weekDaySlots}: Number of weekday slots supported
\item \textbf{yearSlots}: Number of date slots supported
\item \textbf{[userId]}: Subtree for userId
\item \qquad\textbf{Weekday}: Subtree for weekday schedule
\item \qquad\qquad\textbf{[slotId]}: Subtree slotId
\item \qquad\qquad\qquad\textbf{dayOfWeek}: Day of week
\item \qquad\qquad\qquad\textbf{startHour}: Start hour
\item \qquad\qquad\qquad\textbf{startMinute}: Start minute
\item \qquad\qquad\qquad\textbf{stopHour}: Stop hour
\item \qquad\qquad\qquad\textbf{stopMinute}: Stop minute
\item \qquad\textbf{Year}: Subtree for date schedule
\item \qquad\qquad\textbf{[slotId]}: Subtree slotId
\item \qquad\qquad\qquad\textbf{startYear}: Start year
\item \qquad\qquad\qquad\textbf{startMonth}: Start month
\item \qquad\qquad\qquad\textbf{startDay}: Start day
\item \qquad\qquad\qquad\textbf{startHour}: Start hour
\item \qquad\qquad\qquad\textbf{startMinute}: Start minute
\item \qquad\qquad\qquad\textbf{stopYear}: Stop year
\item \qquad\qquad\qquad\textbf{stopMonth}: Stop month
\item \qquad\qquad\qquad\textbf{stopDay}: Stop day
\item \qquad\qquad\qquad\textbf{stopHour}: Stop hour
\item \qquad\qquad\qquad\textbf{stopMinute}: Stop minute
\end{itemize}

\paragraph{Command ScheduleEntryLock Enable}
\begin{quote}Syntax: Enable(user, enable, successCallback = NULL, failureCallback = NULL)\end{quote}
\begin{quote}Description: Send ScheduleEntryLock Enable(All)\end{quote}
\begin{quote}Parameter user: User to enable/disable schedule for. 0 to enable/disable for all users\end{quote}
\begin{quote}Parameter enable: TRUE to enable schedule, FALSE otherwise\end{quote}
\begin{quote}Parameter successCallback: Custom function to be called on function success. NULL if callback is not needed\end{quote}
\begin{quote}Parameter failureCallback: Custom function to be called on function failure. NULL if callback is not needed\end{quote}


\paragraph{Command ScheduleEntryLock WeekdayGet}
\begin{quote}Syntax: WeekdayGet(user, slot, successCallback = NULL, failureCallback = NULL)\end{quote}
\begin{quote}Description: Send ScheduleEntryLock Weekday Get\end{quote}
\begin{quote}Parameter user: User to get schedule for. 0 to get for all users\end{quote}
\begin{quote}Parameter slot: Slot to get schedule for. 0 to get for all slots\end{quote}
\begin{quote}Parameter successCallback: Custom function to be called on function success. NULL if callback is not needed\end{quote}
\begin{quote}Parameter failureCallback: Custom function to be called on function failure. NULL if callback is not needed\end{quote}
\begin{quote}Report: userId->Weekday->slotId subtree updated\end{quote}

\paragraph{Command ScheduleEntryLock WeekdaySet}
\begin{quote}Syntax: WeekdaySet(user, slot, dayOfWeek, startHour, startMinute, stopHour, stopMinute, successCallback = NULL, failureCallback = NULL)\end{quote}
\begin{quote}Description: Send ScheduleEntryLock Weekday Set\end{quote}
\begin{quote}Parameter user: User to set schedule for\end{quote}
\begin{quote}Parameter slot: Slot to set schedule for\end{quote}
\begin{quote}Parameter dayOfWeek: Weekday number (0..6). 0 = Sunday. . 6 = Saturday\end{quote}
\begin{quote}Parameter startHour: Hour when schedule starts (0..23)\end{quote}
\begin{quote}Parameter startMinute: Minute when schedule starts (0..59)\end{quote}
\begin{quote}Parameter stopHour: Hour when schedule stops (0..23)\end{quote}
\begin{quote}Parameter stopMinute: Minute when schedule stops (0..59)\end{quote}
\begin{quote}Parameter successCallback: Custom function to be called on function success. NULL if callback is not needed\end{quote}
\begin{quote}Parameter failureCallback: Custom function to be called on function failure. NULL if callback is not needed\end{quote}
\begin{quote}Report: userId->Weekday->slotId subtree updated\end{quote}

\paragraph{Command ScheduleEntryLock YearGet}
\begin{quote}Syntax: YearGet(user, slot, successCallback = NULL, failureCallback = NULL)\end{quote}
\begin{quote}Description: Send ScheduleEntryLock Year Get\end{quote}
\begin{quote}Parameter user: User to enable/disable schedule for. 0 to get for all users\end{quote}
\begin{quote}Parameter slot: Slot to get schedule for. 0 to get for all slots\end{quote}
\begin{quote}Parameter successCallback: Custom function to be called on function success. NULL if callback is not needed\end{quote}
\begin{quote}Parameter failureCallback: Custom function to be called on function failure. NULL if callback is not needed\end{quote}
\begin{quote}Report: userId->Year->slotId subtree updated\end{quote}

\paragraph{Command ScheduleEntryLock YearSet}
\begin{quote}Syntax: YearSet(user, slot, startYear, startMonth, startDay, startHour, startMinute, stopYear, stopMonth, stopDay, stopHour, stopMinute, successCallback = NULL, failureCallback = NULL)\end{quote}
\begin{quote}Description: Send ScheduleEntryLock Year Set\end{quote}
\begin{quote}Parameter user: User to set schedule for\end{quote}
\begin{quote}Parameter slot: Slot to set schedule for\end{quote}
\begin{quote}Parameter startYear: Year in current century when schedule starts (0..99)\end{quote}
\begin{quote}Parameter startMonth: Month when schedule starts (1..12)\end{quote}
\begin{quote}Parameter startDay: Day when schedule starts (1..31)\end{quote}
\begin{quote}Parameter startHour: Hour when schedule starts (0..23)\end{quote}
\begin{quote}Parameter startMinute: Minute when schedule starts (0..59)\end{quote}
\begin{quote}Parameter stopYear: Year in current century when schedule stops (0..99)\end{quote}
\begin{quote}Parameter stopMonth: Month when schedule stops (1..12)\end{quote}
\begin{quote}Parameter stopDay: Day when schedule stops (1..31)\end{quote}
\begin{quote}Parameter stopHour: Hour when schedule stops (0..23)\end{quote}
\begin{quote}Parameter stopMinute: Minute when schedule stops (0..59)\end{quote}
\begin{quote}Parameter successCallback: Custom function to be called on function success. NULL if callback is not needed\end{quote}
\begin{quote}Parameter failureCallback: Custom function to be called on function failure. NULL if callback is not needed\end{quote}
\begin{quote}Report: userId->Year->slotId subtree updated\end{quote}


\section{Command Class Climate Control Schedule (0x46/70)}

\textit{Version 1, Supported and Controlled}
\newline

Obsolete but still partly implemented for legacy support.
\newline

\noindent
Data holders:

\begin{itemize}
\item \textbf{overrideType}: Type of current override
\item \textbf{overrideState}: State of override
\end{itemize}

\paragraph{Command ClimateControlSchedule OverrideGet}
\begin{quote}Syntax: OverrideGet(successCallback = NULL, failureCallback = NULL)\end{quote}
\begin{quote}Description: Send ClimateControlSchedule Override Get\end{quote}
\begin{quote}Parameter successCallback: Custom function to be called on function success. NULL if callback is not needed\end{quote}
\begin{quote}Parameter failureCallback: Custom function to be called on function failure. NULL if callback is not needed\end{quote}


\paragraph{Command ClimateControlSchedule OverrideSet}
\begin{quote}Syntax: OverrideSet(overrideType, overrideState, successCallback = NULL, failureCallback = NULL)\end{quote}
\begin{quote}Description: Send ClimateControlSchedule Override Set\end{quote}
\begin{quote}Parameter overrideType: Override type to set. (0 – no override, 1 – temporary override, 2 – permanent override)\end{quote}
\begin{quote}Parameter overrideState: Override state. -128 (0x80) ... -1 (0xFF): setpoint -12.8 ... -0.1 degrees. 0 (0x00): setpoint. 1 (0x01) ... 120 (0x78): setpoint +0.1 ... +12 degrees. 121 (0x79): frost protection. 122 (0x7A): energy saving. 123 (0x7B) ... 126 (0x7D): reserved. 127 (0x7F): unused\end{quote}
\begin{quote}Parameter successCallback: Custom function to be called on function success. NULL if callback is not needed\end{quote}
\begin{quote}Parameter failureCallback: Custom function to be called on function failure. NULL if callback is not needed\end{quote}



\section{Command Class MeterTableMonitor (0x3D/61)}

\textit{Version 2, Controlled}
\newline

Allows to read historical and accumulated values in physical units from a water meter or other metering device (gas, electric etc.) and thereby enabling automatic meter reading capabilities.
\newline

\noindent
Data holders:

\begin{itemize}
\item \textbf{adminId}: Meter administrator ID
\item \textbf{Id}: Customer ID
\item \textbf{rateType}: Type of rate (export or import)
\item \textbf{payMeter}: Specifies the way the account is done
\item \textbf{meterType}: Meter type
\item \textbf{meterTypeString}: Meter description
\item \textbf{dataSetMask}: Internal. Bit mask with type of data set supported
\item \textbf{dataSetHistoryMask}: Internal. Bit mask with type of data set history supported
\item \textbf{maxHistory}: Max number of records the device can store
\item \textbf{statusMask}: Internal. Bit mask with type of events supported
\item \textbf{maxEvents}: Max number of events the device can store
\item \textbf{[dataSetId]}: Subtree for data set
\item \qquad\textbf{val}: Meter value for this data set
\item \qquad\textbf{time}: Time corresponding to the value
\item \qquad\textbf{scale}: Scale ID
\item \qquad\textbf{scaleString}: Scale desctiption
\item \qquad\textbf{history}: Requires Z-Wave specification re-read. Please contact Z-Wave.Me support
\item \textbf{status}: Subtree with statuses
\item \qquad\textbf{[statuseId]}: Subtree with specific status ID
\item \qquad\qquad\textbf{statusString}: Status descirption
\item \qquad\qquad\textbf{active}: Requires Z-Wave specification re-read. Please contact Z-Wave.Me support
\item \qquad\qquad\textbf{time}: Requires Z-Wave specification re-read. Please contact Z-Wave.Me support
\end{itemize}

\paragraph{Command MeterTableMonitor StatusDateGet}
\begin{quote}Syntax: StatusDateGet(maxResults = 0, startDate, endDate, successCallback = NULL, failureCallback = NULL)\end{quote}
\begin{quote}Description: Send StatusTableMonitor Status Get for a range of dates\end{quote}
\begin{quote}Parameter maxResults: Maximum number of entries to get from log. 0 means all matching entries\end{quote}
\begin{quote}Parameter startDate: Start date and time (local UNIX time)\end{quote}
\begin{quote}Parameter endDate: End date and time (local UNIX time)\end{quote}
\begin{quote}Parameter successCallback: Custom function to be called on function success. NULL if callback is not needed\end{quote}
\begin{quote}Parameter failureCallback: Custom function to be called on function failure. NULL if callback is not needed\end{quote}


\paragraph{Command MeterTableMonitor StatusDepthGet}
\begin{quote}Syntax: StatusDepthGet(maxResults = 0, successCallback = NULL, failureCallback = NULL)\end{quote}
\begin{quote}Description: Send StatusTableMonitor Status Get for specified depth\end{quote}
\begin{quote}Parameter maxResults: Number of entries to get from log. 0 means current status only. 0xFF means all entries\end{quote}
\begin{quote}Parameter successCallback: Custom function to be called on function success. NULL if callback is not needed\end{quote}
\begin{quote}Parameter failureCallback: Custom function to be called on function failure. NULL if callback is not needed\end{quote}


\paragraph{Command MeterTableMonitor CurrentDataGet}
\begin{quote}Syntax: CurrentDataGet(setId = 0, successCallback = NULL, failureCallback = NULL)\end{quote}
\begin{quote}Description: Send StatusTableMonitor Current Data Get\end{quote}
\begin{quote}Parameter setId: Index of dataset to get data for. 0 to get data for all supported datasets\end{quote}
\begin{quote}Parameter successCallback: Custom function to be called on function success. NULL if callback is not needed\end{quote}
\begin{quote}Parameter failureCallback: Custom function to be called on function failure. NULL if callback is not needed\end{quote}


\paragraph{Command MeterTableMonitor HistoricalDataGet}
\begin{quote}Syntax: HistoricalDataGet(setId = 0, maxResults = 0, startDate, endDate, successCallback = NULL, failureCallback = NULL)\end{quote}
\begin{quote}Description: Send StatusTableMonitor Historical Data Get\end{quote}
\begin{quote}Parameter setId: Index of dataset to get data for. 0 to get data for all supported datasets\end{quote}
\begin{quote}Parameter maxResults: Maximum number of entries to get from log. 0 means all matching entries\end{quote}
\begin{quote}Parameter startDate: Start date and time (local UNIX time)\end{quote}
\begin{quote}Parameter endDate: End date and time (local UNIX time)\end{quote}
\begin{quote}Parameter successCallback: Custom function to be called on function success. NULL if callback is not needed\end{quote}
\begin{quote}Parameter failureCallback: Custom function to be called on function failure. NULL if callback is not needed\end{quote}



\section{Command Class Alarm (0x71/113)}

\textit{Version 5, Controlled}
\newline

Also known as Notification Command Class. Used to report alarm events from binary sensors. Starting from version 3 all types are strictly defines:. 

- 0x01 Smoke. 

- 0x02 CO. 

- 0x03 CO2. 

- 0x04 Heat. 

- 0x05 Water. 

- 0x06 Access Control. 

- 0x07 Burglar. 

- 0x08 Power Management. 

- 0x09 System. 

- 0x0a Emergency. 

- 0x0b Clock.
\newline

\noindent
Data holders:

\begin{itemize}
\item \textbf{V1supported}: boolean flag saying if version 1 (deprecated) is supported
\item \textbf{V1event }: structure to store V1 events
\item \qquad\textbf{alarmType}: V1 alarm type
\item \qquad\textbf{level}: V1 status
\item \textbf{typeMask}: bit mask of supported alarm types
\item \textbf{[typeId]}: subtree to store events of specific alarm types
\item \qquad\textbf{typeString}: name of the alarm type
\item \qquad\textbf{status}: flag with alarm status (alarm enabled/disabled)
\item \qquad\textbf{eventMask}: bit mask of supported events of this alarm type
\item \qquad\textbf{event}: last event ID
\item \qquad\textbf{eventString}: last event name
\item \qquad\textbf{eventParameters}: last event parameters
\item \qquad\textbf{eventSequence}: internal
\end{itemize}

\paragraph{Command Alarm Get}
\begin{quote}Syntax: Get(type = 0, event = 0, successCallback = NULL, failureCallback = NULL)\end{quote}
\begin{quote}Description: Send Alarm Get. Requests the status of a specific event of a specific alarm type\end{quote}
\begin{quote}Parameter type: Type of alarm to get level for. 0 to get level for all supported alarms (v2 and higher). 0xFF to get level for first supported alarm (v2 and higher)\end{quote}
\begin{quote}Parameter event: Notification event to get level for. This argument is ignored prior to Notification v3. Must be 0 if type is 0xFF\end{quote}
\begin{quote}Parameter successCallback: Custom function to be called on function success. NULL if callback is not needed\end{quote}
\begin{quote}Parameter failureCallback: Custom function to be called on function failure. NULL if callback is not needed\end{quote}
\begin{quote}Report: Alarm type subtree updated\end{quote}

\paragraph{Command Alarm Set}
\begin{quote}Syntax: Set(type, level, successCallback = NULL, failureCallback = NULL)\end{quote}
\begin{quote}Description: Send Alarm Set (v2 and higher). Enable/disable alarms of a specific type\end{quote}
\begin{quote}Parameter type: Type of alarm to set level for\end{quote}
\begin{quote}Parameter level: Level to set (0x0 = off, 0xFF = on, other values are reserved)\end{quote}
\begin{quote}Parameter successCallback: Custom function to be called on function success. NULL if callback is not needed\end{quote}
\begin{quote}Parameter failureCallback: Custom function to be called on function failure. NULL if callback is not needed\end{quote}
\begin{quote}Report: Alarm type subtree updated\end{quote}


\section{Command Class PowerLevel (0x73/115)}

\textit{Version 1, Supported and Controlled}
\newline

Used to set device power level and to test the link to a other devices in the network.
\newline

\noindent
Data holders:

\begin{itemize}
\item \textbf{level}: Current power level (0 for normal power, 1..9 for -1..-9 dBm)
\item \textbf{timeout}: Timeout of the power level set (after timeout the device turns back to normal power)
\item \textbf{[nodeId]}: Subtree with report of a test with nodeId
\item \qquad\textbf{status}: Current test status (0 = Failed, 1 = Successfully finished, 2 = In progress)
\item \qquad\textbf{totalFrames}: Total frames sent
\item \qquad\textbf{acknowledgedFrames}: Acknowledged frames from total sent
\end{itemize}

\paragraph{Command PowerLevel Get}
\begin{quote}Syntax: Get(successCallback = NULL, failureCallback = NULL)\end{quote}
\begin{quote}Description: Send PowerLevel Get\end{quote}
\begin{quote}Parameter successCallback: Custom function to be called on function success. NULL if callback is not needed\end{quote}
\begin{quote}Parameter failureCallback: Custom function to be called on function failure. NULL if callback is not needed\end{quote}
\begin{quote}Report: level and timeout updated\end{quote}

\paragraph{Command PowerLevel Set}
\begin{quote}Syntax: Set(level, timeout, successCallback = NULL, failureCallback = NULL)\end{quote}
\begin{quote}Description: Send PowerLevel Set\end{quote}
\begin{quote}Parameter level: Power level to set (from 0 to 9)\end{quote}
\begin{quote}Parameter timeout: Timeout in seconds\end{quote}
\begin{quote}Parameter successCallback: Custom function to be called on function success. NULL if callback is not needed\end{quote}
\begin{quote}Parameter failureCallback: Custom function to be called on function failure. NULL if callback is not needed\end{quote}
\begin{quote}Report: level and timeout updated\end{quote}

\paragraph{Command PowerLevel TestNodeGet}
\begin{quote}Syntax: TestNodeGet(successCallback = NULL, failureCallback = NULL)\end{quote}
\begin{quote}Description: Send PowerLevel Test Node Get\end{quote}
\begin{quote}Parameter successCallback: Custom function to be called on function success. NULL if callback is not needed\end{quote}
\begin{quote}Parameter failureCallback: Custom function to be called on function failure. NULL if callback is not needed\end{quote}
\begin{quote}Report: subtree with report for the given node updated\end{quote}

\paragraph{Command PowerLevel TestNodeSet}
\begin{quote}Syntax: TestNodeSet(testNodeId, level, frameCount, successCallback = NULL, failureCallback = NULL)\end{quote}
\begin{quote}Description: Send PowerLevel Test Node Set. Starts sending specified number of NOP packets to a given device at a given power level. Once finished, unsolicited report MIGHT be sent by the device (at any time you can use TestNodeGet)\end{quote}
\begin{quote}Parameter testNodeId: Node to set test packets to\end{quote}
\begin{quote}Parameter level: Power level to use (from 0 to 9)\end{quote}
\begin{quote}Parameter frameCount: Number of test frames to send (from 1 to 65535)\end{quote}
\begin{quote}Parameter successCallback: Custom function to be called on function success. NULL if callback is not needed\end{quote}
\begin{quote}Parameter failureCallback: Custom function to be called on function failure. NULL if callback is not needed\end{quote}
\begin{quote}Report: subtree with report for the given node updated\end{quote}


\section{Command Class Z-Wave Plus Info (0x5E/94)}

\textit{Version 2, Supported and Controlled}
\newline

Describes device Z-Wave Plus role and type.
\newline

\noindent
Data holders:

\begin{itemize}
\item \textbf{plusVersion}: Z-Wave Plus version
\item \textbf{roleType}: Z-Wave Plus role type
\item \textbf{roleTypeString}: Z-Wave Plus role type description
\item \textbf{nodeType}: Z-Wave Plus node type
\item \textbf{installerIcon}: Icon for installer
\item \textbf{userIcon}: Icon for user
\end{itemize}

\paragraph{Command ZWavePlusInfo Get}
\begin{quote}Syntax: Get(successCallback = NULL, failureCallback = NULL)\end{quote}
\begin{quote}Description: Send ZWave+ Info Get\end{quote}
\begin{quote}Parameter successCallback: Custom function to be called on function success. NULL if callback is not needed\end{quote}
\begin{quote}Parameter failureCallback: Custom function to be called on function failure. NULL if callback is not needed\end{quote}



\section{Command Class Firmware Update (0x7A/122)}

\textit{Version 4, Controlled}
\newline

Allows to update firmware of the device (OTA, Over-The-Air upgrade).
\newline

\noindent
Data holders:

\begin{itemize}
\item \textbf{upgradeable}: Flag representing if the firmware is upgradable
\item \textbf{firmwareCount}: Number of firmwares that can be updated using OTA (for multi chip devices, 0 is Z-Wave chip only)
\item \textbf{updateStatus}: Indicated the status of the update process
\item \textbf{waitTime}: Time the device will take before rebooting with newly upgraded firmware
\item \textbf{manufacturerId}: Manufacturere ID
\item \textbf{firmwareId}: Firmware Id
\item \textbf{firmware[n]}: Firmware Id of firmware [n]
\item \textbf{checksum}: Checksum of the firmware
\item \textbf{fragmentTransmitted}: Number of fragments transmitted (useful to make progress bar)
\item \textbf{fragmentCount}: Number of fragments to be transmitted in total (useful to make progress bar)
\item \textbf{fragmentSize}: Internal
\item \textbf{firmwareData}: Internal
\end{itemize}

\paragraph{Command FirmwareUpdate Get}
\begin{quote}Syntax: Get(successCallback = NULL, failureCallback = NULL)\end{quote}
\begin{quote}Description: Send Firmware Metadata Get\end{quote}
\begin{quote}Parameter successCallback: Custom function to be called on function success. NULL if callback is not needed\end{quote}
\begin{quote}Parameter failureCallback: Custom function to be called on function failure. NULL if callback is not needed\end{quote}
\begin{quote}Report: upgradeable, firmwareCount, updateStatus, manufacturerId, manufacturerId, firmwareId, firmware[n], checksum updated\end{quote}

\paragraph{Command FirmwareUpdate Perform}
\begin{quote}Syntax: Perform(manufacturerId, firmwareId, firmwareTarget, data, successCallback = NULL, failureCallback = NULL)\end{quote}
\begin{quote}Description: Send Firmware Update Request Get. On process start Z-Way sets fragmentCount:. devices.N.instances.0.commandClasses.122.data.fragmentCount = 3073 (0x00000c01). Then it asks the device to start the process. The device can refuse it (i.e. if local confirmation timed out). If confirmed, the device will send us a report with adjusted fragment size (if it wants Z-Way to send by smaller packets) and report "Ready" (updateStatus = 255, see below). devices.N.instances.0.commandClasses.122.data.updateStatus = 255 (0x000000ff). devices.N.instances.0.commandClasses.122.data.fragmentCount = 3277 (0x00000ccd). At this point fragmentTransmitted == 0. devices.N.instances.0.commandClasses.122.data.fragmentTransmitted =0. Then device starts asking Z-Way for different packets. Z-Way will update fragmentTransmitted to allow track the process. Once done (fragmentCount == fragmentTransmitted), the device will send again a report if the flashing was successful. updateStatus contains the status: checksum error = 0, assemble error = 1, success, restart manually = 254, success, automatic restart = 255. waitTime refers to the time device will take to reboot. devices.N.instances.0.commandClasses.122.data.updateStatus = 255 (0x000000ff). devices.N.instances.0.commandClasses.122.data.waitTime = 5 (0x00000005)\end{quote}
\begin{quote}Parameter manufacturerId: Manufacturer Id (2 bytes)\end{quote}
\begin{quote}Parameter firmwareId: Firmware Id (2 bytes)\end{quote}
\begin{quote}Parameter firmwareTarget: Firmware target number (0 for main chip, 1..255 for additional chips). Used only for CC v3 and above\end{quote}
\begin{quote}Parameter data: Firmware image data in binary format (use hex2bin to convert from Intel Hex)\end{quote}
\begin{quote}Parameter successCallback: Custom function to be called on function success. NULL if callback is not needed\end{quote}
\begin{quote}Parameter failureCallback: Custom function to be called on function failure. NULL if callback is not needed\end{quote}
\begin{quote}Report: updateStatus, waitTime, fragmentCount, fragmentTransmitted updated\end{quote}


\section{Command Class Association Group Information (0x59/89)}

\textit{Version 1, Supported and Controlled}
\newline

Describes association groups defined by Association Command Class and command sent to group members.
\newline

\noindent
Data holders:

\begin{itemize}
\item \textbf{[groupId]}: Subtree for grouId
\item \qquad\textbf{groupName}: Group name
\item \qquad\textbf{profile}: Group profile Id
\item \qquad\textbf{mode}: Internal. Reserved.
\item \qquad\textbf{eventCode}: Internal. Reserved
\item \qquad\textbf{commands}: Subtree for commands
\item \qquad\qquad\textbf{[commandClassId]}: Command Class Id of the command sent to group members
\item \qquad\qquad\qquad\textbf{[commandId]}: Command Id corresponding to Command Class Id
\item \textbf{dynamic}: Flag describing if the list can change and periodic request to update information is suggested
\end{itemize}

\paragraph{Command AssociationGroupInformation GetInfo}
\begin{quote}Syntax: GetInfo(groupId, successCallback = NULL, failureCallback = NULL)\end{quote}
\begin{quote}Description: Send AGI Get Info\end{quote}
\begin{quote}Parameter groupId: Group Id to get info for (0 for all groups)\end{quote}
\begin{quote}Parameter successCallback: Custom function to be called on function success. NULL if callback is not needed\end{quote}
\begin{quote}Parameter failureCallback: Custom function to be called on function failure. NULL if callback is not needed\end{quote}


\paragraph{Command AssociationGroupInformation GetName}
\begin{quote}Syntax: GetName(groupId, successCallback = NULL, failureCallback = NULL)\end{quote}
\begin{quote}Description: Send AGI Get Name\end{quote}
\begin{quote}Parameter groupId: Group Id to get info for (0 for all groups)\end{quote}
\begin{quote}Parameter successCallback: Custom function to be called on function success. NULL if callback is not needed\end{quote}
\begin{quote}Parameter failureCallback: Custom function to be called on function failure. NULL if callback is not needed\end{quote}


\paragraph{Command AssociationGroupInformation GetCommands}
\begin{quote}Syntax: GetCommands(groupId, successCallback = NULL, failureCallback = NULL)\end{quote}
\begin{quote}Description: Send AGI Get Commands\end{quote}
\begin{quote}Parameter groupId: Group Id to get info for (0 for all groups)\end{quote}
\begin{quote}Parameter successCallback: Custom function to be called on function success. NULL if callback is not needed\end{quote}
\begin{quote}Parameter failureCallback: Custom function to be called on function failure. NULL if callback is not needed\end{quote}



\section{Command Class SwitchColor (0x33/51)}

\textit{Version 3, Controlled}
\newline

Allows to control color for multicolor lights including LED bulbs and LED strips. Device reports it's capabilities:. 

- 0 Warm White (0x00...0xFF: 0...100\%). 

- 1 Cold White (0x00:...0xFF: 0...100\%). 

- 2 Red (0x00...0xFF: 0...100\%). 

- 3 Green (0x00...0xFF: 0...100\%). 

- 4 Blue (0x00...0xFF: 0...100\%). 

- 5 Amber (for 6ch Color mixing) (0x00...0xFF: 0...100\%). 

- 6 Cyan (for 6ch Color mixing) (0x00...0xFF: 0...100\%). 

- 7 Purple (for 6ch Color mixing) (0x00...0xFF: 0...100\%). 

- 8 Indexed Color (0x00...0x0FF: Color Index 0...255).
\newline

\noindent
Data holders:

\begin{itemize}
\item \textbf{capabilityMask}: Internal. Bit mask with supported capabilities
\item \textbf{[capabilityId]}: Subtree for capabilityId
\item \qquad\textbf{capabilityString}: Capability description
\item \qquad\textbf{level}: Level of capability
\end{itemize}

\paragraph{Command SwitchColor Get}
\begin{quote}Syntax: Get(capabilityId, successCallback = NULL, failureCallback = NULL)\end{quote}
\begin{quote}Description: Send SwitchColor Get\end{quote}
\begin{quote}Parameter capabilityId: Capability Id\end{quote}
\begin{quote}Parameter successCallback: Custom function to be called on function success. NULL if callback is not needed\end{quote}
\begin{quote}Parameter failureCallback: Custom function to be called on function failure. NULL if callback is not needed\end{quote}


\paragraph{Command SwitchColor Set}
\begin{quote}Syntax: Set(capabilityId, state, duration = 0xff, successCallback = NULL, failureCallback = NULL)\end{quote}
\begin{quote}Description: Send SwitchColor Set\end{quote}
\begin{quote}Parameter capabilityId: Capability Id\end{quote}
\begin{quote}Parameter state: State to be set for the capability\end{quote}
\begin{quote}Parameter duration: Duration of change:. 0 instantly. 0x01...0x7f in seconds. 0x80...0xfe in minutes mapped to 1...127 (value 0x80=128 is 1 minute). 0xff use device factory default\end{quote}
\begin{quote}Parameter successCallback: Custom function to be called on function success. NULL if callback is not needed\end{quote}
\begin{quote}Parameter failureCallback: Custom function to be called on function failure. NULL if callback is not needed\end{quote}


\paragraph{Command SwitchColor SetMultiple}
\begin{quote}Syntax: SetMultiple(capabilityIds, states, duration = 0xff, successCallback = NULL, failureCallback = NULL)\end{quote}
\begin{quote}Description: Send SwitchColor SetMultiple\end{quote}
\begin{quote}Parameter capabilityIds: Array of capabilities to set\end{quote}
\begin{quote}Parameter states: Array of state values to be set for the capabilities\end{quote}
\begin{quote}Parameter duration: Duration of change:. 0 instantly. 0x01...0x7f in seconds. 0x80...0xfe in minutes mapped to 1...127 (value 0x80=128 is 1 minute). 0xff use device factory default\end{quote}
\begin{quote}Parameter successCallback: Custom function to be called on function success. NULL if callback is not needed\end{quote}
\begin{quote}Parameter failureCallback: Custom function to be called on function failure. NULL if callback is not needed\end{quote}


\paragraph{Command SwitchColor StartStateChange}
\begin{quote}Syntax: StartStateChange(capabilityId, dir, ignoreStartLevel = TRUE, startLevel = 50, successCallback = NULL, failureCallback = NULL)\end{quote}
\begin{quote}Description: Send SwitchColor StartStateChange\end{quote}
\begin{quote}Parameter capabilityId: Capability Id to start changing state for\end{quote}
\begin{quote}Parameter dir: Direction of change: 0 to incrase, 1 to decrase\end{quote}
\begin{quote}Parameter ignoreStartLevel: If set to True, device will ignore start level value and will use it's curent value\end{quote}
\begin{quote}Parameter startLevel: Start level to change from\end{quote}
\begin{quote}Parameter successCallback: Custom function to be called on function success. NULL if callback is not needed\end{quote}
\begin{quote}Parameter failureCallback: Custom function to be called on function failure. NULL if callback is not needed\end{quote}


\paragraph{Command SwitchColor StopStateChange}
\begin{quote}Syntax: StopStateChange(capabilityId, successCallback = NULL, failureCallback = NULL)\end{quote}
\begin{quote}Description: Send SwitchColor StopStateChange\end{quote}
\begin{quote}Parameter capabilityId: Capability Id to stop changing state for\end{quote}
\begin{quote}Parameter successCallback: Custom function to be called on function success. NULL if callback is not needed\end{quote}
\begin{quote}Parameter failureCallback: Custom function to be called on function failure. NULL if callback is not needed\end{quote}



\section{Command Class BarrierOperator (0x66/102)}

\textit{Version 1, Controlled}
\newline

Allows to control barriers and garage doors as well as their signal lamps.
\newline

\noindent
Data holders:

\begin{itemize}
\item \textbf{state}: Barrier state
\item \textbf{signalMask}: Internal. Bit mask of available signals
\item \textbf{[signalId]}: Subtree for signal
\item \qquad\textbf{signalTypeString}: Signal description
\item \qquad\textbf{state}: Signal state
\end{itemize}

\paragraph{Command BarrierOperator Get}
\begin{quote}Syntax: Get(successCallback = NULL, failureCallback = NULL)\end{quote}
\begin{quote}Description: Send BarrierOperator Get\end{quote}
\begin{quote}Parameter successCallback: Custom function to be called on function success. NULL if callback is not needed\end{quote}
\begin{quote}Parameter failureCallback: Custom function to be called on function failure. NULL if callback is not needed\end{quote}


\paragraph{Command BarrierOperator Set}
\begin{quote}Syntax: Set(state, successCallback = NULL, failureCallback = NULL)\end{quote}
\begin{quote}Description: Send BarrierOperator Set\end{quote}
\begin{quote}Parameter state: State to set\end{quote}
\begin{quote}Parameter successCallback: Custom function to be called on function success. NULL if callback is not needed\end{quote}
\begin{quote}Parameter failureCallback: Custom function to be called on function failure. NULL if callback is not needed\end{quote}


\paragraph{Command BarrierOperator SignalGet}
\begin{quote}Syntax: SignalGet(signalType, successCallback = NULL, failureCallback = NULL)\end{quote}
\begin{quote}Description: Send BarrierOperator Signal Get\end{quote}
\begin{quote}Parameter signalType: Signal subsystem type to get state for\end{quote}
\begin{quote}Parameter successCallback: Custom function to be called on function success. NULL if callback is not needed\end{quote}
\begin{quote}Parameter failureCallback: Custom function to be called on function failure. NULL if callback is not needed\end{quote}


\paragraph{Command BarrierOperator SignalSet}
\begin{quote}Syntax: SignalSet(signalType, state, successCallback = NULL, failureCallback = NULL)\end{quote}
\begin{quote}Description: Send BarrierOperator Signal Set\end{quote}
\begin{quote}Parameter signalType: Signal subsystem type to set state for\end{quote}
\begin{quote}Parameter state: State to set\end{quote}
\begin{quote}Parameter successCallback: Custom function to be called on function success. NULL if callback is not needed\end{quote}
\begin{quote}Parameter failureCallback: Custom function to be called on function failure. NULL if callback is not needed\end{quote}



\section{Command Class SimpleAVControl (0x94/148)}

\textit{Version 4, Supported and Controlled}
\newline

Allows to control A/V devices.
\newline

\noindent
Data holders:

\begin{itemize}
\item \textbf{bitmask}: Bit mask with supported keys. Refer to Expert UI pyzw\_zwave.js or Sigma Designs documentation for description of buttons.
\item \textbf{bitmasks}: Internal
\item \textbf{sequenceNumber}: Internal
\item \textbf{reportsNumber}: Internal
\end{itemize}

\paragraph{Command SimpleAVControl Set}
\begin{quote}Syntax: Set(keyAttribute, avCommand, successCallback = NULL, failureCallback = NULL)\end{quote}
\begin{quote}Description: Send SimpleAVControl Set\end{quote}
\begin{quote}Parameter keyAttribute: 0 for key Down, 1 for key Up, 2 for key Alive (repeated every 100...200 ms)\end{quote}
\begin{quote}Parameter avCommand: Command to be sent. One of 465 predefined in Z-Wave protocol\end{quote}
\begin{quote}Parameter successCallback: Custom function to be called on function success. NULL if callback is not needed\end{quote}
\begin{quote}Parameter failureCallback: Custom function to be called on function failure. NULL if callback is not needed\end{quote}



\section{Command Class Security (0x98/152)}

\textit{Version 1, Supported and Controlled}
\newline

This Command Class is transparently implemented in the library. There are no functions to execute.
\newline

\noindent
Data holders:

\begin{itemize}
\item \textbf{controller->data->secureControllerId}: Node Id of secure controller: node that established secure channel when we are secondary controller (this data is on controller data tree)
\item \textbf{device->data->secureChannelEstablished}: Flag describing if security interview was successful and secure channel is established (this data is on device data tree)
\item \textbf{secureNodeInfoFrame}: Secure Node Information Frame
\item \textbf{securityAbandoned}: Security interview failed
\item \textbf{scheme}: Secure scheme supported
\item \textbf{securityRequested}: Internal
\item \textbf{rNonce}: Internal
\item \textbf{rNonceAckWait}: Internal
\item \textbf{canStream}: Internal
\item \textbf{firstPart}: Internal
\item \textbf{sequenceId}: Internal
\item \textbf{toFollow}: Internal
\end{itemize}

\paragraph{Command Security Inject}
\begin{quote}Syntax: Inject(data, successCallback = NULL, failureCallback = NULL)\end{quote}
\begin{quote}Description: Send Security Inject\end{quote}
\begin{quote}Parameter data: Data to set\end{quote}
\begin{quote}Parameter successCallback: Custom function to be called on function success. NULL if callback is not needed\end{quote}
\begin{quote}Parameter failureCallback: Custom function to be called on function failure. NULL if callback is not needed\end{quote}



\section{Command Class CRC16 (0x56/86)}

\textit{Version 1, Supported and Controlled}
\newline

This Command Class is transparently implemented in the library to use better 16 bits packet checksum. There are no functions to execute.
\newline

\noindent
Data holders:

\begin{itemize}
\item \textbf{crc16Requested}: Internal
\end{itemize}


\section{Command Class MultiCmd (0x8F/143)}

\textit{Version 1, Supported and Controlled}
\newline

This Command Class is transparently implemented in the library to save battery life time. There are no functions to execute.
\newline

\noindent
Data holders:

\begin{itemize}
\item \textbf{maxNum}: Max number of packets to be encapsulated. Can be tunned to lower (to workaround buggy devices, 1 to turn off) or rise (to get bettery performance)
\end{itemize}


\section{Command Class Supervision (0x6C/108)}

\textit{Version 1, Supported and Controlled}
\newline

This Command Class is transparently implemented in the library to guarantee delivery report on every command (even on Set). There are no functions to execute.
\newline

\noindent
Data holders:

\begin{itemize}
\item \textbf{[sessionId]}: Subtree with session status
\item \qquad\textbf{status}: Current session status (0 = Not supported, 1 = Working, 2 = Fail, 3 = Busy, 255 = Success)
\item \qquad\textbf{duration}: Expected time to finish the operation
\item \qquad\textbf{moreStatusUpdates}: True if more updates on the session status are expected
\item \textbf{lastSession}: Internal
\end{itemize}


\section{Command Class Application Status (0x22/34)}

\textit{Version 1, Supported and Controlled}
\newline

This Command Class is transparently implemented in the library to retry on device Busy report. There are no functions to execute.

\section{Command Class Version (0x86/134)}

\textit{Version 2, Supported and Controlled}
\newline

Allows to get version of each Command Class supported by the device as well as firmware version.
\newline

\noindent
Data holders:

\begin{itemize}
\item \textbf{commandClass->data->version}: Version of specific Command Class (this data is on Command Class data tree)
\item \textbf{ZWLib}: SDK library type
\item \textbf{ZWProtocolMajor}: SDK version major
\item \textbf{ZWProtocolMinor}: SDK version minor
\item \textbf{SDK}: SDK description
\item \textbf{applicationMajor}: Application version major
\item \textbf{applicationMinor}: Application version minor
\item \textbf{hardwareVersion}: Hardware revision of the device
\item \textbf{firmwareCount}: Number of chips (firmwares) in the device (excluding Z-Wave chip)
\item \textbf{[firmwareId]}: Subtree for firmwareId information
\item \qquad\textbf{major}: Additional chip application version major
\item \qquad\textbf{minor}: Additional chip application version major
\end{itemize}


\section{Command Class DeviceResetLocally (0x5A/90)}

\textit{Version 1, Supported and Controlled}
\newline

Reports to the controller that device was resetted locally (using local button operation).
\newline

\noindent
Data holders:

\begin{itemize}
\item \textbf{reset}: Becomes True if the device sent us DeviceResetLocally notification. This means the device is certainly not in our network anymore
\end{itemize}


\section{Command Class Central Scene (0x5B/91)}

\textit{Version 3, Supported and Controlled}
\newline

Allows to receive central controller oriented scene actions. Scenes are triggered by pushing a button on a remote control or wall controller. Note that Z-Way supports only V1, but in most cases you don't need it to be enabled in the NIF. Controlled version is V3.
\newline

\noindent
Data holders:

\begin{itemize}
\item \textbf{maxScenes}: Number of scenes supported
\item \textbf{slowRefreshSupport}: Flag to indicate if the device supports Slow Refresh mode
\item \textbf{slowRefresh}: Flag to indicate if the device is currently in Slow Refresh mode
\item \textbf{currentScene}: Last activated scene
\item \textbf{keyAttribute}: Button (or key) action: 0 for key press, 1 for key release, 2 for key held down (should bre repeated at least every 200ms)
\item \textbf{sequence}: Internal. To ignore duplicate packats.
\item \textbf{sceneSupportedKeyAttributesMask}: Holds the list of supported key attributes for each scene
\item \qquad\textbf{[sceneId]}: Array of supported key attributes for Scene Id: 0 for 1 press, 1 for release after hold, 2 for hold, 3..6 for 2..5 presses
\end{itemize}
