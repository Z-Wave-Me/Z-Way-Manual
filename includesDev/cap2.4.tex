\section{JSON-API}
\label{jsapi}
JSON API allows to execute commands on server side using HTTP POST requests 
(currently GET requests are also allowed, but this might be deprecated in future). 
The command to execute is taken from the URL.

All functions are executed in form 

\paragraph{http://YOURIP:8083/$<$URL$>$}


\subsection{/ZWaveAPI/Run/$<$command$>$}

This executes zway.$<$command$>$ in JavaScript engine of the server. As an example to 
switch ON a device no 2 using the command class BASIC (The ID of the command class BASIC is 0x20, for more
information about the IDs of certain command classes please refer to the Annex) its 
possible to write:

\begin{quote}/ZWaveAPI/Run/devices[2].instances[0].commandClasses[0x20].Set(255)\end{quote}

or

\begin{quote}/ZWaveAPI/Run/devices[2].instances[0].Basic.Set(255)\end{quote}



The Z-Way Expert GUI has a JavaScript command runCmd($<$command$>$) to simplify such 
operations. This function  is accessable in the Javascript console of your web browser 
(in Chrome you find the JavaScript  console unter View-$>$Debug-$>$JS Console). Using 
this feature the command in JS console would look like

\begin{quote}runCmd('devices[2].instances[0].Basic.Set(255)')\end{quote}


The usual way to access a command class is using the format \\'devices[nodeId].instances[instanceId].
commandClasses[commandclassId]'.
There are ways to simplify the syntax:

\begin{itemize}
\item 'devices[nodeId].instances[instanceId].Basic' is equivalent to \\
'devices[nodeId].instances[instanceId].commandClasses[0x20]'
\item the instances[0] can be obmitted: 'devices[nodeId].instances[instanceId].Basic' 
then turns into 'devices[nodeId].Basic'
\end{itemize} 

Each Instance object has a device property that refers to the parent device it belongs to. 
Each Command Class Object has a device and an instance property that refers to the instance 
and the device this command class belongs to.
 

Data holder object have properties value, updateTime, invalidateTime, name, but for 
compatibility with JS and previous versions we have valueOf() method (allows to 
omit .value in JS code, hence write "data.level == 255"), updated (alias to updateTime), 
invalidated (alias toinvalidateTime). These aliases are not enumerated if the dataholder 
is requested (data.level returns {value: 255, name: "level", updatedTime: 12345678, 
invalidatedTime: 12345678}).

\subsection{/ZWaveAPI/InspectQueue}

This function is used to visualize the Z-Way job queue. This is for debugging only but 
very useful to understand the current state of Z-Way engine.

\subsection{/ZWaveAPI/Data/$<$timestamp$>$}
Returns an associative array of changes in Z-Way data tree since $<$timestamp$>$. The 
array consists of ($<$path$>$: $<$JSON object$>$) object pairs. The client is supposed 
to assign the new $<$JSON object$>$ to the subtree with the $<$path$>$ discarding previous 
content of that subtree. Zero (0) can be used instead of $<$timestamp$>$ to obtain the 
full Z-Way data tree.

The tree have same structure as the backend tree (Figure \ref{zwaystructure}) with 
one additional root element "updateTime" which contains the time of latest update. 
This "updateTime" value should be used in the next request for changes. 
All timestamps (including updateTime) corresponds to server local time.

Each node in the tree contains the following elements:
\begin{itemize}
\item value — the value itself
\item updateTime — timestamp of the last update of this particular value
\item invalidateTime — timestamp when the value was invalidated by issuing a Get 
command to a device and expecting a Report command from the device
\end{itemize}

The object looks like:
\begin{lstlisting}[caption=JSON Data Structure]{Name}
{
"[path from the root]": [updated subtree],
"[path from the root]": [updated subtree],
...
updateTime: [current timestamp]
}
\end{lstlisting}

Examples for Commands to update the data tree look like:

\begin{quote}Get all data: /ZWaveAPI/Data/0\end{quote}

\begin{quote} Get updates since 134500000 (Unix timestamp): /ZWaveAPI/Data/134500000\end{quote}

Please note that during data updates some values are updated by big subtrees. For example, 
in Meter Command Class value of a scale is always updated as a scale subtree by 
[scale].val object (containing scale and type descriptions).


\subsection{Handling of updates coming from Z-Way}

A good design of a UI is linking UI objects (label, textbox, slider, ...) to a certain 
path in the tree object. Any update of a subtree linked to UI will then update the UI too. 
This is called bindings.

For web applications Z-Way Web UI contains a library called jQuery.triggerPath 
(extention of jQuery written by Z-Wave.Me), that allows to make such links between 
objects in the tree and HTML DOM objects. Use

\begin{quote}var tree;\end{quote}
\begin{quote}jQuery.triggerPath.init(tree);\end{quote}

during web application initialization to attach the library to a tree object. Then run

\begin{quote}jQuery([objects selector]).bindPath([path with regexp], [updater function], 
[additional arguments]);\end{quote}

to make binding between path changes and updater function. The updater function would be 
called upon changes in the desired object with this pointing to the DOM object itself, 
first argument pointing to the updated object in the tree, second argument is the exact 
path of this object (fulfilling the regexp) and all other arguments 
copies additional arguments. RegExp allows only few control characters: * 
is a wildcard, (1|2|3) - is 1 or 
2 or 3.  

\textbf{Please not that the use of the triggerpath extension is one option to handle the incoming
data. You can also extract all the interesting values right when the data is received and 
bind update functions to them.}