\chapter{Special topics for Developers}

\section{Authentication}
\label{cap:authentication}

In order to access API one need to authenticate itself. \zway uses sessions
to authenticate users. Session can be obtained by sending login and password
in JSON format using POST to URL \texttt{/ZAutomation/api/v1/login}. User
credentials should look like \texttt{\{"login":"admin", "password":"admin"\}}

In return the session will be sent in two forms:
\begin{itemize}
\item as \textbf{data.sid} field in JSON structure,
\item as a cookie called \textbf{ZWAYSession}.
\end{itemize}

Example of successful login will look like:
\begin{lstlisting}[caption=Successful login reply]
{
    "data": {
        "sid": "ba69cb5b-b2fd-5ce0-5b75-9bae3e8bc369",
        "id": 1,
        "role": 1,
        "name": "Administrator",
        "lang": "en",
        "color": "#dddddd",
        "dashboard": [],
        "interval": 2000,
        "rooms": [
            0
        ],
        "hide_all_device_events": false,
        "hide_system_events": false,
        "hide_single_device_events": []
    },
    "code": 200,
    "message": "200 OK",
    "error": null
}
\end{lstlisting}

\begin{lstlisting}[caption=Wrong login/password reply]
{
    "data": null,
    "code": 401,
    "message": "401 Unauthorized",
    "error": "User login/password is wrong."
}
\end{lstlisting}

One obtained, the session can be sent to the server via cookie (as set by
\zway) or via HTTP header.

An elegant way to execute a command using URL including authentication is:

\begin{quote} 
\cmdline{wget --auth-no-challenge --user=admin --password=pwd IP:8083/ZAutomation/api/v1/...}
\end{quote}  

As jquery function the authentication will look like this:

\begin{lstlisting}[caption=Login with jQuery]
$('img').click(function() {
$.ajax({
url: "https://find.z-wave.me/ZWave.zway/Run/devices[3].Basic.Set(255)",
username: "56033/admin",
password: "pwd",
xhrFields: { withCredentials: true }
});
});
\end{lstlisting}

\section{How to write own Apps for \zway}
\label{developownapps}

According to Chapter \ref{cap:modules} apps have two core files:

\begin{itemize}
\item module.js
\item index.js
\end{itemize}

The following chapter explains these two files more in detail. 

\subsection{module.js}

Module.js defines the general behavior of the app and the interface to the user side.
Table \ref{tab:module.json} 
shows the structure of the file module.js with an explanation of each line item.

\begin{table}
\begin{tabular}{|p{0.45\textwidth}|p{0.5\textwidth}|}
\hline
\{ "singleton" : false, &
Boolean to set if there can be multiple Instances of the
module allowed or not \\
"dependencies": [], & 
An array list of all module names from which this module is dependent. Modules in this list
should be 'singleton'.Thew module cannot be instantiated if at least one of the modules
in the list does not have an instance. \\
\hline
"category": "automation\_basics", & 
The app category this module is shown in the app store. Known app store categories are:
'basic\_gateway\_modules',
'legacy\_products\_workaround', 'support\_external\_ui',
'support\_external\_dev', 'automation\_basic',
'device\_enhancements', 'developers\_stuff',
'complex\_applications', 'automation', 'security',
'peripherals', 'surveillance', 'logging', 'scripting',
'scheduling', 'climate', 'environment', 'scenes',
'notifications', 'tagging' \\
\hline
"author": "ZWave.Me", & Author name of the Module \\
\hline
"homepage": "http://razberry.zwave. me", &
If you have a news homepage, it can be linked here. \\
\hline
"icon": "icon.png", & Name of the icon which is shown for this module
on the UI \\
\hline
"moduleName": "AppClassName", & Module name have to the same like the class
reference \\
\hline
"version": "1.0.0", &Version number of this module \\
\hline
"maturity": "beta", & Status if the app is still in development or released \\
\hline
"repository": \{ Repository optional "type": "git", Kind of the repository
"source":
https://github.com/ZWaveMe/ homeautomation \}, &
Address of the repository \\
\hline
"defaults" : \{ "title" : "\_\_m\_title\_\_", & The title placeholder for the Language files \\
\hline
"description" : \{ "\_m\_descr\_\_" \}, &
The description placeholder for the language files \\
\hline
"schema" : \{\}, & Description of the data structure of the form for
instantiating the module. See explanation of schema for details \\
\hline
"options" : \{\}\} & Showing options of the setup form\\
\hline
"description" : \{ "\_m\_descr\_\_" \}, &
The description placeholder for the language files \\
\hline
\end{tabular}
\caption{Module.json details} 
\label{tab:module.json}
\end{table}	

\subsection{Schema}

The \textbf{schema} is a JSON Structure to define the user interface of the module.
It lists all input parameters and  options to be shown in the setup dialog of the app:

\begin{lstlisting}[caption=Schema Structure]
{
"schema": {
"type": "object",
"properties": {
}
}
}
\end{lstlisting}


The structure of the schema is the following. Inside the 'properties' space the single 'properties' 
can be defined. They become the parameter of the module during the initiation and they 
are shown as configuration parameters in the setup dialog. There are different
types of input parameters:


\subsubsection{Primitive data types like integer, float or string}

\begin{lstlisting}[caption=Schema Structure Simple Type]
{
//Parametername
"name": {

"type": "array",
"items": {
"title": "Device",
"type": "radio",
//array of choosable items
"enum": ["Adult", "Child"],
"default": "Child",
"required": true
}
},
//Parametername
"name": {
"type": "integer",
"required": true
},
}
\end{lstlisting}

\subsubsection{Name Spaces  - Enumerations with a choice}

\begin{lstlisting}[caption=Schema Structure Enumerations with a choice]
{

//Parametername
"name": {
"field": "enum",
"datasource": "namespaces",
//special namespacedestination
"enum": "namespaces:devices\_all:deviceId",
"required": true
},
}
\end{lstlisting}



Name spaces refer to the internal \zway structure. It allows to list elements from 
the \zway data model and filter it. The statement 
\texttt{"namespaces:devices\_all:deviceId"} will offer a selection of all devices.

Namespaces can also be combined like

\murl{namespaces:devices\_doorlock:deviceId,namespaces:devices\_switchBinary:deviceId}

which means devices doorlock and all binary switches.
Namespaces can also be REST paths like

\murl{server:port/v1/namespaces/{devices\_DEVICETYPE}.{PATH}}


\subsection{The file index.js}

Thew file index.js contains the application as such. It can include other js files is needed 
but \zway will always look for a index.js file to load first. Table \ref{tab:detailindexjs}
list the basic structure of index.js with the minimum functions.


\begin{table}
\begin{tabular}{|p{0.45\textwidth}|p{0.5\textwidth}|}
\hline
function AppClassName (id, controller) {AppClassName.super\_.call(this, id, controller);} &
Constructor method:
This line is a call of the Superconstructor. It has always to be first line of the constructor \\
\hline
inherits(AppClassName, AutomationModule); &
inheration call: \\
\hline
\_module = AppClassName; &
Thew definition of the class reference \\
\hline
AppClassName.prototype.init = function (config) {
AppClassName.super\_.prototype.init.call(this, config);
var self = this; }; &
Initialization method: Variable to refer to in the class in own
methods (this is context dependent in JavaScript).
Here you can register 'listeners' for the event bus. For details on event bus please 
refer to chapter \ref{cap:eventbus} \\
\hline
AppClassName.prototype.stop = function () {
AppClassName.super\_.prototype.stop.call(this);}; &
Destroy method:
Here you have to unregister 'listeners'.  \\
\hline
AppClassName.prototype.Methodname=
function(parameter){ } &
Own Methods: Write your own Methods here. \\
\hline
\end{tabular}
\caption{Details of index.js} 
\label{tab:detailindexjs}
\end{table}	
	

More information e.g. about the list of probe types etc. you find on 

\murl{http://docs.zwayhomeautomation.apiary.io/}

\section{Write you own Device Description Files}
\label {newddr} 

This part of the manual is not yet published because the service for creating own Device 
Description Files is not yet available.

\section{Extending EnOcean}
\label{addenocean}

How to include a new EnOcean Device (example Hoppe Door handle)

(1) Check if the profile is in \cmdline{/opt/z-way-server/config/Profile.xml}. Not that 
you just need to know the profile the given product supports. There is no way 
to find out automatically! 

\begin{lstlisting}[caption=EnOcean Profile Entry]{Name}
<Profile rorg="0xf6" func="0x10" type="0x00" rorgDescription="RPS Telegram" 
funcDescription="Mechanical Handle" typeDescription="Window Handle">
<Field offset="0" size="4" name="windowHandle" type="int" 
description="Movement of the window handle" short="WIN" />
</Profile>
\end{lstlisting}

(2) Add the device record of the device to 
\begin{quote} 
\cmdline{/opt/z-wave-server/htdocs/smarthome/storage/data/devices\_encoean.json}. 
\end{quote} 
Here the rorg, funcid and type are set. Now the device record will be created and the 
right values are changed on message reception. Now you need to make sure the right  
element is rendered and updated. This is 
in \cmdline{/opt/z-wave-server/automation/modules/Enocean/index.js}

First add a filter to catch the events and call the correct function:

\begin{lstlisting}[caption=Catch Device IDs]{Name}
if (matchDevice(0xf6, 0x10, 0x01)) {
// Hoppe Window handle
windowHandle("contact", "window", "Windor Handle"); 
}
\end{lstlisting}
now you add the function that handles the value changes and renders the element 
accordingly. For the window handle we use the binary sensor element but overwrite 
the status information according to the information of the window handle moves.  

\begin{lstlisting}[caption=Handle Device]{Name} 
function windowHandle(dh, type, title) {
	var vDev = self.controller.devices.create({
	deviceId: vDevIdPrefix + type,
	defaults: {
		deviceType: 'sensorBinary',
		metrics: {
			probeTitle: type,
			scaleTitle: '',
			icon: type,
			level: '',
			title: title
		}
	},
	overlay: {},
	handler: function(command) {},
	moduleId: self.id
});

if (vDev) {
	self.dataBind(self.gateDataBinding, self.zeno, nodeId, dh, 
	function(type) {
		try {
			if (this. handleValue == 13)
				vDev.set("metrics:level", "tilt");
			if (this. handleValue == 15)
				vDev.set("metrics:level", "closed");
			if (this. handleValue == 12 || this. handleValue == 14)
				vDev.set("metrics:level", "open");
		} catch (e) {}
	}, "value");
}
}
\end{lstlisting}