\chapter{The JavaScript Engine}
\label{cap:js}
\index{JavaScript Engine}


The \zway core function engine is the so called JavaScript (JS) automation system.
It uses the APIs of the technology-dependent 'drivers' and delivers all the functions
and interface for running a Smart Home controller:
\begin{itemize}
\item It \textbf{unifies the functions and properties of the physical devices} to a common 
device structure, the virtual Device (vDev).
\item It allows to \textbf{create own virtual devices} not related to physical devices.
\item It allows to \textbf{dynamically load 'plugin' modules}  - also written in Javascript or 
even in native C - that extend the function of the JS core and deliver furter functionality.
Users see these modules as apps. For more information about apps please refer to chapter 
\ref{apps}.
\item It organizes the \textbf{communication between the virtual devices} as event bus. Every 
vDev can inject events into this bus and every vDev can read events from this bus.
\item It provides a \textbf{higher layer API} that is 
is used among others by the Smart Home User Interface.
\end{itemize}

This chapter explains the different building blocks of the JS Engine:
\begin{itemize}
\item Javascript Interpreter/Compiler Core Engine
\item Virtual Devices
\item Event Bus
\item User Interface elements referring to the different virtual devices
\item The structure of the Apps
\end{itemize}

\section{JavaScript API}

\zway uses the JavaScript engine provided by Google referred to as V8. You find more 
information about this JavaScript implementation on 

\murl{https://code.google.com/p/v8/.}

V8 implements JavaScript according to the specification ECMA 5
\footnote {http://www.ecma-international.org/publications/standards/Ecma-262.htm}.

Please note that this V8 core engine only implements the very basic JS functions and need to 
be extended to be usable in a Smart Home environment.

Z-Way extends the basic functionality provided by V8 with plenty of application-specific functions.

Javascript code can be executed on the server side and certain functions of the JS core are 
available on the client side as well since most modern web browsers have a built-in Javascript
engine as well. The bridge between the server side JS and the web browser client is a built-in
web server. This is the same embedded web server serving all the web browsers HTML pages etc.

There are three ways to run JavaScript code in Z-Way backend.
\begin{itemize}
\item It can be executed in via web browser using a special URL string containitg the code to be executed (using /JS/Run/ prefix),
\item It can be stored as a file on the server in automation folder and executed using 'executeFile' function,
\item It can be implemented as an App running.
\end{itemize}

All options have their pros and cons. Running JS code via the browser is a very nice and convenient way to test things but the code is not persistent across \zway restarts.

Storing the it in a file allows to run it on \zway start (if 'executeFile("myfile.js")' is placed in main.js) but is not really convenient to distribute.

Writing a module requires more knowledge, but includes a nice graphical interface for App configuration. Upload your App in the \zway App Store for easy deployment and distribution of your App.

Check \zway App Store on \murl{https://developer.z-wave.me/} for more information. There are many other open source Apps made by the community.

Accessing a server side JS function from the web browsers client side is easy. Just call 

\murl{http://YOURIP:8083/JS/Run/<any JS code>}

Please note that all accesses using the embedded webserver require an authentication of the 
web browsers instance. Please refer to chapter \ref{cap:authentication} for details how to 
authenticate in the \zway web server.

\zway offers one central object with the name 'zway' . This object encapsulates all the 
Z-Wave variables and functions known from the Z-Wave Device API describes in chapter \ref{c2}.

Hence its possible to use the very same functions of the Z-Wave Device API using the JS
engine. The zway objects internal structure is shown in figure \ref{zwaystructure} and the data elements 
are describes in Annex \ref{datamodel}.

The functions can be accessed using the web browsers function like

\murl{http://YOURIP:8083/JS/Run/zway.devices[x].*}

Due to the scripting nature of JavaScript it is possible to 'inject' code at run time
using the interface. Here a nice example how to use the Java Script 
setInterval function:

\begin{listingverbatim}[Polling device \#2]
/JS/Run/setInterval(function() { 
	zway.devices[2].Basic.Get();
}, 300*1000);
\end{listingverbatim}

This code will, once 'executed' as a URL within a web browser, calls the Get() command
of the command class Basic of Node ID 2 every 300 seconds.  

A very powerful function of the JS API is the ability to bind functions to certain
values of the device tree. they get then executed when the value changes. Here an 
example for this binding. The device No. 3 has a command class SensorMultilevel that offers
the variable level. The following call - both available on the client side 
and on the server side - will bind a simple alert function to the change of 
the variable.

\begin{listingverbatim}[Bind a function]
zway.devices[3].SensorMultilevel.data[1].val.bind(function() { 
	debugPrint('CHANGED TO: ' + this.value + '\n'); 
});  
\end{listingverbatim}



\section{\zway extensions to the JavaScript Core}

\zway provides some extensions to the JS core that are not part of the ECMA 
functionality mentioned above.

\subsection{HTTP Access}

The JavaScript implementation of \zway allows directly accessing HTTP objects.

The http request is much like jQuery.ajax(): r = http.request(options);

Here's the list of options:
\begin{itemize}
\item url - required. Url you want to request (might be http, https, or maybe even ftp);
\item method – optional. HTTP method to use (currently one of GET, POST, HEAD). If not 
specified, GET is used;
\item headers – optional. Object containing additional headers to pass to server:

\begin{listingverbatim}
headers: {
    "Content-Type": "text/xml",
    "X-Requested-With": "RaZberry/1.5.0"
}
\end{listingverbatim}

\item data – used only for POST requests. Data to post to the server. May be either a
string (to post raw data) or an object with keys and values (will be serialized as 
'key1=value1\&key2=value2\&…');
\item auth – optional. Provides credentials for basic authentication. It is an object 
containing login and password:

\begin{listingverbatim}
auth: {
    login: 'username',
    password: 'secret'
}
\end{listingverbatim}

\item contentType – optional. Allows overriding content type returned by the server for 
parsing data (see below);
\item async – optional. Specifies whether request should be sent asynchronously. Default 
is false. In case of synchronous request, result is returned immediately (as function 
return value); otherwise, function exits immediately, and response is delivered later 
thru callbacks.
\item success, error and complete – optional, valid only for async requests. Success 
callback is called after successful request, error is called on failure, complete is 
called nevertheless (even if success/error callback produces exception, so it is like 
the 'finally' statement);
\end{itemize}

Response (as stated above) is delivered either as function return value, or as callback 
parameter. It is always an object containing the following members:

\begin{itemize}
\item status – HTTP status code (or -1 if some non-HTTP error occurred). Status codes 
from 200 to 299 are considered success;
\item statusText – status string;
\item URL – response URL (might differ from URL requested in case of server redirects);
\item headers – object containing all the headers returned by server;
\item contentType – content type returned by server;
\item data – response data.
\end{itemize}

Response data is handled differently depending on content type (if contentType on request 
is set, it takes priority over server content type):

\begin{itemize}
\item application/json and text/x-json are returned as JSON object;
\item application/xml and text/xml are returned as XML object;
\item application/octet-stream is returned as binary ArrayBuffer;
\item string is returned otherwise.
\end{itemize}
In case data cannot be parsed as valid JSON/XML, it is still returned as string, and 
additional parseError member is present.


\begin{listingverbatim}
http.request({
	url: "http://server.com" (string, required),
	method: "GET" (GET/POST/HEAD, optional, default "GET"),
	
	headers: (object, optional)
	{
		"name": "value",
		...
	},
	
	auth: (object, optional)
	{
		"login": "xxx" (string, required),
		"password": "***" (string, required)
	},
	
	data: (object, optional, for POST only)
	{
		"name": "value",
		...
	}
	-- OR --
	data: "name=value&..." (string, optional, for POST only),

	async: true (boolean, optional, default false),

	timeout: (number, optional, default 20000)
	
	success: function(rsp) {} (function, optional, for async only),
	error: function(rsp) {} (function, optional, for async only),
	complete: function(rsp) {} (function, optional, for async only)
});


response:
{
	status: 200 (integer, -1 for non-http errors),
	statusText: "OK" (string),
	url: "http://server.com" (string),
	contentType: "text/html" (string),
	headers: (object)
	{
		"name": "value"
	},
	data: result (object or string, depending on content type)
}
\end{listingverbatim}

\subsection{XML parser}

ZXmlDocument object allows converting any valid XML document into a JSON object and vice versa.

\subsubsection{var x = new ZXmlDocument()}
Create new empty XML document

\subsubsection{x = new ZXmlDocument("xml content")}
Create new XML document from a string

\subsubsection{x.root}
Get/set document root element. Elements are got/set in form of JS objects:

\begin{listingverbatim}
\{
    name: "node_name", - mandatory
    text: "value", - optional, for text nodes
    attributes: { - optional
    	name: "value",
    	...
    },
    children: [ - optional, should contain a valid object of same type
    	{ ... }
    ]
}
\end{listingverbatim}

For example:
\begin{listingverbatim}
(new ZXmlDocument('<weather><city id="1"><name>Zwickau</name>
	<temp>2.6</temp></city> 
	<city id="2"><name>Moscow</name><temp>-23.4</temp></city>
	</weather>')).root =
{  
   "children":[  
      {  
         "children":[  
            {  
               "text":"Zwickau",
               "name":"name"
            },
            {  
               "text":"2.6",
               "name":"temp"
            }
         ],
         "attributes":{  
            "id":"1"
         },
         "name":"city"
      },
      {  
         "children":[  
            {  
               "text":"Moscow",
               "name":"name"
            },
            {  
               "text":"-23.4",
               "name":"temp"
            }
         ],
         "attributes":{  
            "id":"2"
         },
         "name":"city"
      }
   ],
   "name":"weather"
}
\end{listingverbatim}

\subsubsection{x.isXML}
This hidden read-only property allows detecting if the object is an XML object or not (it is always true).

\subsubsection{x.toString()}
Converts XML object into a string with valid XML content.

\subsubsection{x.findOne(XPathString)}
Returns first matching to XPathString element or null if not found.

\begin{listingverbatim}
x.findOne('/weather/city[@id="2"]') // returns only city tag for Moscow
x.findOne('/weather/city[name="Moscow"]/temp/text()') // returns temperature in Moscow
\end{listingverbatim}

\subsubsection{x.findAll(XPathString)}
Returns array of all matching to XPathString elements or empty array if not found.

\begin{listingverbatim}
x.findAll('/weather/city') // returns all city tags
x.findAll('/weather/city/name/text()') // returns all city names
\end{listingverbatim}

\subsubsection{x.document}
A hidden property that refers to the document root.

\subsubsection{XML elements}
Each XML element (tag) in addition to properties described above (text, attributes, 
children) have hidden read-only property parent pointing to parent object and the following methods:
\begin{itemize}
\item insertChild(element) Insert new child eleemnt
\item removeChild(element) Remove child element
\item findOne(XPathString) Same as on root object, but relative (no leading / needed in XPathString
\item findAll(XPathString) Same as on root object, but relative (no leading / needed in XPathString
\end{itemize}

ZXmlDocument is returned from http.request() when content type is 'application/xml', 
'text/xml' or any other ending with '+xml'. Namespaces are not yet supported.

\subsection{Cryptographic functions}

\textbf{crypto} object provides access to some popular cryptographic functions such
as SHA1, SHA256, SHA512, MD5, HMAC, and provides good random numbers.

\subsubsection{var guid = crypto.guid()}
Provides standard GUID in string format.

\subsubsection{var rnd = crypto.random(n)}
Generates n random bytes.
Returned value is of type ArrayBuffer. To convert it into array, use this trick:

\begin{listingverbatim}
	rnd = (new Uint8Array(crypto.random(10)));
\end{listingverbatim}

\subsubsection{var dgst = crypto.digest(hash, data, ...)}
Returns digest calculated using selected hash algorithm. It supports virtually all the 
algorithms available in OpenSSL (md4, md5, mdc2, sha, sha1, sha224, sha256, sha384, sha512, ripemd160).
If no data parameters specified, it returns a digest of an empty value. If more than one 
data parameters are specified, they're all used to calculate the result. Data parameters 
may be of several types: strings, arrays, ArrayBuffers.
Return value is of type ArrayBuffer.

There are also a few shortcut functions for popular algorithms: 'md5', 'sha1', 'sha256', 
'sha512'. For example, these calls are equivalent:

\begin{listingverbatim}
	dgst = crypto.digest('sha256', data);
	dgst = crypto.sha256(data);
\end{listingverbatim}

\subsubsection{var hmac = crypto.hmac(cipher, key, data, ...)}
Returns hmac calculated using selected hash algorithm. Hash algorithms are the same as 
for digest() function.

Key parameter is required. 

If no data parameters specified, it returns a HMAC of an empty value. If more than one 
data parameter is specified, they're all used to calculate the result. Key and data 
parameters may be of different types (strings, arrays, ArrayBuffers).
Return value is of type ArrayBuffer.

There are also a few shortcut functions for popular algorithms: 'hmac256', 'hmac512'. For 
example, these calls are equivalent:

\begin{listingverbatim}
  dgst = crypto.hmac('sha256', key, data);
  dgst = crypto.hmac256(key, data);
\end{listingverbatim}

\subsection{Sockets functions}

Socket module allows easy access to TCP and UDP sockets from JavaScript.
Both connection to distant ports and listening on local are available. This API fully 
mirrors into JavaScript POSIX TCP/IP sockets.
This can be used to control third party devices like Global Cache or Sonos
as well as emulating third party services.

To start communication, one needs to create socket and either
\textbf{connect} it or \textbf{listen} it. \textbf{onrecv} method is called
on data receive from remote, while \textbf{send} is used to send data to remote side.

The example below dumps to log file response to http://ya.ru:80/ (raw HTTP
protocol is used as an example).

\begin{listingverbatim}
var sock = new sockets.tcp();

sock.onrecv = function(data) {
    debugPrint(data.byteLength);
};

sock.connect('ya.ru', 80);

sock.send("GET / HTTP/1.0\r\n\r\n");
\end{listingverbatim}

Here is an example of TCP echo server on port 8888:

\begin{listingverbatim}
var sock = new sockets.tcp();

sock.bind(8888);

sock.onrecv = function(data) {
    this.send(data);
};

sock.listen();
\end{listingverbatim}

And echo server for UDP:
\begin{listingverbatim}
var sock = new sockets.udp();

sock.bind(8888);

sock.onrecv = function(data, host, port) {
    this.sendto(data, host, port);
};

sock.listen();
\end{listingverbatim}

Important! Callbacks can only be specified before the connection is established.

``this`` inside callbacks refers to the socket object itself.

Detailed description of Socket API:
\begin{itemize}
\item bind(ip, port) or bind(port) binds socket to port (integer number). ip should be a 
string like "192.168.0.1". If omited "0.0.0.0" is used (bind on all IP addresses of all 
interfaces). Returns false on error.
\item connect(ip, port) connects to remote side ip:port. TCP sockets requires this call 
before sending data. For UDP sockets it is optional, but once used, it allows using send call  instead of sendto call. Returns false on error.
\item listen() starts listening port (this is required not only for TCP, but for UDP too). 
Returns false on error.
\item close() initiate close of socket.
\item send(data) sends data to connected or accepted socket.
\item sendto(data, host, port) sends data to a non-connected UDP socket.
\item onrecv(data, host, port) called on new data receiption from remote side. For UDP 
sockets and connected TCP sockets "this" object refers to the socket itself, while for 
accepted TCP sockets "this" reffers to the client's individual objects.
\item onconnect(remoteHost, remotePort, localHost, localPort) is called only for TCP sockets 
on new connection accept or for established connection on client socket.
"this" refers to the client individual socket object.
\item onclose(remoteHost, remotePort, localHost, localPort) called on socket close by 
remote or due to close() call. Note that for TCP sockets this callback is called for client 
sockets on connection close and for bindound listening socket if close() is called. "this" 
object will be defined like in onrecv.
\item reusable() sets SO\_REUSEADDR socket option to allow multiple bind() on the same port.
\item broadcast() sets SO\_BROADCAST socket option to allow sending broadcast UDP messages.
\item multicastAddMembership(multicastGroup) subscribe socket to multicast group
\item multicastDropMembership(multicastGroup) unsubscribe socket from multicast group
\end{itemize}

\subsection{WebSockets functions}

Socket module also implements WebSockets (RFC 6455). WebSocket API is made to be
compatible with browser implementations (some rarely used functions are not
implemented, see below).

The example below implements basic application using the WebSockets client:

\begin{listingverbatim}
var sock = new sockets.websocket("ws://echo.websocket.org");

sock.onopen = function () {
    debugPrint('connected, sending ping');
    sock.send('ping');
}

sock.onmessage = function(ev) {
    debugPrint('recv', ev.data);
}

sock.onclose = function() {
    debugPrint('closed');
}

sock.onerror = function(ev) {
    debugPrint('error', ev.data);
}
\end{listingverbatim}

Next example shows basic application using WebSockets server:

\begin{listingverbatim}
var sock = new sockets.websocket(9009);

sock.onconnect = function () {
    debugPrint('client connected, sending ping');
}

sock.onmessage = function(ev) {
    debugPrint('recv', ev.data);
    sock.send('pong');
}

sock.onclose = function() {
    if (this === sock) {
        debugPrint('server websocket closed');
    } else {
        debugPrint('client disconnected');
    }
}

sock.onerror = function(ev) {
    debugPrint('error', ev.data);
}
\end{listingverbatim}

Detailed description of WebSocket API:
\begin{itemize}
\item sockets.websocket(url, [protocol], [ssl\_ca\_filepath], [ssl\_cert\_filepath, 
ssl\_private\_key\_filepath]) creates new client WebSocket and connects to the 
specified URL (should be a string like "ws://host:port" or "wss://host:port" for 
SSL channel). Optional protocol parameter can be used to specify protocol from server 
capabilities (comma separated string), default is "default". To use a specific CA file 
instead of system default use ssl\_ca\_filepath. To send client certificate use  ssl\_cert\_filepath and ssl\_private\_key\_filepath.
\item sockets.websocket(port) creates new WebSocket server on port.
\item close() initiate close of WebSocket.
\item send(data) sends data to WebSocket. data can be array, ArrayBuffer (sent as binary) or string (sent as text).
\item onmessage(event) called on new data reception from remote side. Object event contains only data property. Other properties mentioned in RFC 6455 are not supported.
\item onopen() or onconnect() called on connection establish. Compared RFC 6455 event parameter is not passed.
\item onclose() called on WebSocket close by remote or due to close() call. For server side will be called on client instance and on listening instance (use this to differenciate). Note that if close() is called before connection was established, onclose() is not executed. Compared RFC 6455 event parameter is not passed.
\item onerror(event) called on error. For example, host or port unreachable. Note that new sockets.websocket(..) can throw an exception on DNS resolution error or on network unreachable. Other errors will be reported via onerror. Parameter event contains only propery data. Other properties from RFC 6455 are not implemented.
\end{itemize}

\subsection{MQTT functions}

MQTT module allows to connect to MQTT broker from JavaScript.
Both subscribtion to remote to topics and publishing own topics are possible.
TCP and WebSocket transports are supported as well as TLS (requires libmosquitto 2.1.0 or upper).

The example below connects to a server using and publishes a topic.

\begin{listingverbatim}
var m = new mqtt("broker.emqx.io", 1883);

m.onconnect = function() {
	debugPrint("Connected");
};

m.ondisconnect = function() {
	debugPrint("Disconnected");
};

m.onpublish = function() {
	debugPrint("Published");
};
it will work when the sent message is published;

m.onsubscribe = function () {
	debugPrint("Subscribed");
};

m.onmessage = function (topic, message) {
	debugPrint("New topic " + topic + ": " + message);
};

m.connect();
\end{listingverbatim}

Here is an example of WebSocket based connection:

\begin{listingverbatim}
var m = new mqtt("ws://broker.emqx.io", 8083);
\end{listingverbatim}

Important! Callbacks can only be specified before the connection is established.

``this`` inside callbacks refers to the mqtt object itself.

Detailed description of WebSocket API:
\begin{itemize}
\item new mqtt(host, port, [login], [password], [clientId]) host can be just a host name or ws://hostname or wss://hostname. login/password allows to use authentication if the broker requires it. Instead of login/password it is possible to use clientId.
\item onconnect() called when the connection is established
\item ondisconnect() called when the client is disconnected
\item onpublish() called when the message is successfully published
\item onsubscribe() called when the broker confirms the subscription
\item onmessage(topic, message) called on a new publication received
\end{itemize}

You can to set the TSL settings before the connection is established.
Configure the client for certificate based SSL/TLS support:
\begin{itemize}
\item tlsset(ca\_file, ca\_path, cert\_file, key\_file) set TLS settings.
ca\_file: path to a file containing the PEM encoded trusted CA certificate files. Either cafile or ca\_path must not be empty string.
ca\_path: path to a directory containing the PEM encoded trusted CA certificate files. Either cafile or ca\_path must not be empty string.
cert\_file: path to a file containing the PEM encoded certificate file for this client. If empty string, key\_file must also be empty string and no client certificate will be used.
key\_file: path to a file containing the PEM encoded private key for this client. If empty string, cert\_file must also be empty string and no client certificate will be used.
\item tlsinsecure() disables the verification of the server hostname in the server certificate. This is unsecure, use for debugging only!
\item tlsadvanced(cert\_reqs, tls\_version, ciphers) Set advanced SSL/TLS options.
cert\_reqs: 1 to enable the server certificate verification, or 0 to disable during debugging.
tls\_version: tlv1, tlsv1.1, tlv1.2 or empty for default.
ciphers: a string describing the ciphers available for use. If empty string, the default ciphers will be used.
\end{itemize}

The MQTT object provides the following methods:
\begin{itemize}
\item connect() to connect to the broker
\item publish(topic, message) to pblish a message on a given topic
\item subscribe(topic) to subscribe to a topic
\item unsubscribe(topic) to unsubscribe to a topic
\item disconnect() to disconnect from the broker, you can connect again later
\end{itemize}

\subsection{Other JavaScript Extensions}

\subsubsection{fs.list(folder)}

This returns the list of items in the folder or undefined if the folder does not exist.


\subsubsection{fs.stat(file)}

This returns one of the following values:

\begin{itemize}
\item 1) undefined if object does not exist or not readable
\item 2) object \{ type: 'file', size: \textless{}size\textgreater{}\} if it is a file
\item 3) object \{ type: 'dir' \} if it is a folder
\end{itemize} 


\subsubsection{fs.loadJSON(filename)}

This function reads a file from the file system and loads it into the memory. The file 
must contain a valid JSON object. The only argument is the name of the file including 
relative pathname to the automation folder. Returns the full JSON object or null in case of error.

\subsubsection{fs.load(filename)}

This function reads a file from the file system and returns its content as a string. The 
only argument is the name of the file including relative pathname to the automation folder.
Returns null in case of error.

\subsubsection{executeFile(filename) and executeJS(string)}

Loads and executes a particular JavaScript file from the local filesystem or executes 
JavaScript code represented in string (like eval in browsers).

The script is executed within the global namespace.

Remark: If an error occurrs during the execution, it won't stop from further execution, 
but erroneous scripts will not be executed completely. It will stop at the first error.
Exceptions in the executed code can be trapped in the caller using standard try-catch mechanism.

\subsubsection{system(command)}

The command system() allows executing any shell level command available on the operating 
system. It will return the shell output of the command.  By default the execution of 
system commands is forbidden. Each command executed need to be permitted by putting one 
line with the starting commands in the file automation/.syscommands or in an different 
automation folder as specified in config.xml.

\subsubsection{Timers}
Timers are implemented exactly as they are used in browsers. They are very helpful for 
periodical and delayed operations. Timeout/period is defined in milliseconds.
\begin{itemize}
\item timerId = setTimeout(function() { }, timeout)
\item timerId = setInterval(function() { }, period)
\item clearTimeout(timerId)
\item clearInterval(timerId)
\end{itemize}

\subsubsection{loadObject(object\_name) and saveObject(object\_name, object)}
Loads and saves JSON object from/to storage. These functions implement flat storage for 
application with access to the object by its name. No folders are available.

Data is saved in automation/storage folder. Filenames are made from object names by 
stripping characters but [a-ZA-Z0-9] and adding checksum from original name (to avoid name conflicts).

\subsubsection{exit()}
Stops JavaScript engine and shuts down \zway server


\subsubsection{allowExternalAccess(handlerName) and listExternalAccess()}
allowExternalAccess allows registering HTTP handler. handlerName can contain strings like 
aaa.bbb.ccc.ddd --- in that case any HTTP request starting by /aaa/bbb/ccc/ddd will be 
handled by a function aaa.bbb.ccc.ddd() if present, otherwise aaa.bbb.ccc(), ... up to aaa().
The handler should return object with at least property status and body (one can also specify 
headers like it was in http.request module).

listExternalAccess returns array with names of all registered HTTP handlers.

Here is an example how to attach handlers for /what/timeisit and /what:

\begin{listingverbatim}
what = function() {
  return { status: 500, body: 'What do you want to know' };
};

what.timeisit = function() {
  return { status: 200, body: (new Date()).toString() }
};

allowExternalAccess("what");
allowExternalAccess("what.timeisit");
\end{listingverbatim}

\subsubsection{debugPrint(object, object, ...)}

Prints arguments converted to string to \zway console. Very useful for debugging.
For convenience, one can map 'console.log()' to debugPrint().

This is how it was done in automation/main.js in \zway Home Automation engine:
\begin{listingverbatim}
var console = {
    log: debugPrint,
    warn: debugPrint,
    error: debugPrint,
    debug: debugPrint,
    logJS: function() {
        var arr = [];
        for (var key in arguments)
            arr.push(JSON.stringify(arguments[key]));
        debugPrint(arr);
    }
};
\end{listingverbatim}

\subsection{Debugging JavaScript code}
Change in config.xml debug-port to 8183 (or some other) turn on V8 debugger capability on \zway start.

\begin{listingverbatim}
<config>
    ...
    <debug-port>8183</debug-port>
    ....
</config>
\end{listingverbatim}

node-inspector debugger tool is required. It provides web-based UI for debugging similar 
to Google Chrome debug console.

You might want to run debugger tool on another machine (for example if it is not possible to 
install it on the same box as \zway is running on).

Use the following command to forward debugger port defined in config.xml to your local machine:
\begin{quote} 
\cmdline{ssh -N USER@IP\_OF\_Z-WAY\_MACHINE -L 8183:127.0.0.1:8183}
(for RaZberry USER is pi)
\end{quote} 


Install node-inspector debugger tool and run it:
\begin{quote} 
\cmdline{npm install -g node-inspector node-inspector --debug-port 8183}
\end{quote} 

Then you can connect to 

\murl{http://IP\_OF\_MACHINE\_WITH\_NODE\_INSPECTOR:8080/debug?port=8183}

If debugging is turned on, \zway gives you five seconds during startup to reconnect debugger to 
\zway (refresh the page of debugger Web UI within these five seconds).
This allows you to debug startup code of \zway JavaScript engine from the very first line of code.

\section{The virtual device concept (vDev)}
\label{cap:vdev}
\index{Virtual Device}
\index{vDev}

A virtual device is a data object within the JS engine. Virtual devices have 
properties and functions. Most virtual devices represent a physical device or a part of 
a physical device but virtual devices are not limited to this. Virtual devices can be 
pure dummy device doing nothing but pretenting to be a device (There is an app called
\app{Dummy Device} that works exactly like this). Virtual devices can also connect to
services via TCP/IP.

The purpose of virtual devices is to unify the appearance on a graphical user interface 
and to unify the communication between them. At the level of virtual devices 
and EnOcean controller can switch a Z-Wave switch and trigger a rule in a cloud 
service.
 
\subsection{Names and Ids}

Every virtual device is identified by a simple string type id. For all virtual devices 
that are related to physical Z-Wave devices the device name is auto-generated by the 
module (app) 'Z-Wave' following this logic:

\paragraph{ZWayVDev\_[Node ID]:[Instance ID]:[Command Class ID]:[Scale ID]}

The Node Id is the node id of the physical device, the Instance ID is the instance id 
of the device  or '0' if there is only one instance. The command class ID refers to the 
command class the function is embedded in. The scale id is usually '0' unless the virtual
device is generated from a Z-Wave device that supports multiple sensors with different 
scales in one single command class.

Virtual devices not generated by a Z-Wave device may have other Ids. They are either created 
by other physical device subsystems such as 433MHz or EnOcean or they are generated
by a module (app).

\subsection{Device Type}

Virtual devices can have a certain types. Table shows the different types plus the defines 
commands.  Table \ref{tab:devicetypes} shows the list of current device types with their 
metrics and defines commands.

\begin{table}
\scriptsize
\begin{tabular}{|p{0.15\textwidth}|p{0.15\textwidth}|p{0.2\textwidth}|p{0.40\textwidth}|}
\hline
deviceType	&Metrics	&Commands	&Examples\\
\hline
battery	&probeTitle,scaleTitle, level, icon, title	&-	&- \\
\hline
doorlock	&level, icon, title	&open or close	&apiURL/devices/:deviceId/command/open\\
\hline
thermostat	&scaleTitle, min, max, level, icon, title	&exact with get-param level	&apiURL/devices/:deviceId/command/exact?
level=22.5\\
\hline
switchBinary (Thermostat)	&level, icon, title	&on, off or update&	apiURL/devices/:deviceId/command/on\\
\hline
switchBinary&	level, icon, title	&on, off or update	&apiURL/devices/:deviceId/command/on\\
\hline
switchMultilevel	&level, icon, title	& on Set(255), off Set(0), min Set(10), max Set(99), increase Set(l+10), decrease Set(l-10), update, exact + get params level	&apiURL/devices/:deviceId/command/exact? level=40 \\
\hline
switchMultilevel (Blinds)&	level, icon, title	&up Set(255), down Set(0), upMax Set(99), increase Set(l+10), decrease Set(l-10), startUp StartLevelChange(0), startDown StartLevelChange(1), stop StopLevelChange(), update, excactSmooth + get params level	& apiURL/devices/:deviceId/command/stop  \\
\hline
sensorBinary	&probeTitle, level, icon, title	&update	&apiURL/devices/:deviceId/command/update\\
\hline
sensorMultilevel	&probeTitle, scaleTitle, level, icon, title	&update	&apiURL/devices/:deviceId/command/update \\
\hline
toggleButton	&level, icon, title&	on	&apiURL/devices/:deviceId/command/on \\
\hline
camera	&icon, title	&depends on installed camera - could be: zoomIn, zoomOut, up, down, left, right, close, open	&apiURL/devices/:deviceId/zoomIn \\
\hline
switchControl	&level, icon, title,& change	on, off, upstart, upstop, downstart, downstop, exact with get-param level	&apiURL/devices/:deviceId/command/on \\
\hline
text	&title, text, icon	&-	&- \\
\hline
sensorMultiline	&multilineType, title, icon, level, (scaleTitle, ...)	&depends on apps	&apiURL/devices/:deviceId/command/:cmd \\
\hline
switchRGB	&icon, title, color: {r:255,g:255,b:255}, &level	on, off, exact with get-params: red, green and blue	& apiURL/devices/:deviceId/command/exact?
red=20\&green=240\&blue=0 \\
\hline
\end{tabular}
\caption{vDev device types with metrics and commands} 
\label{tab:devicetypes}
\end{table}	

 
\subsection{Access to Virtual Devices}

Virtual devices can be access both on the server side using JS modules and on the client 
side using the JSON API. On the client they are encoded into a URL style for easier 
handling  in AJAX code. A typical client side command in the vDev API looks like

\murl{http://YOURIP:8083/ZAutomation/api/v1/devices/ZWayVDev\_6:0:37/command/off}

'api' points to the vDev API function, 'v1' is just a constant to allow future extensions. 
The devices are referred by a name that is automatically generated from the Z-Wave 
Device API. The vDev also unifies the commands 'command' and the parameters, here 'off'.

On the server side the very same command would be encoded in a JavaScript style.

\begin{listingverbatim}[Access vDevs]

vdevId = vdev.id;

vDev = this.controller.devices.get(vdevId);

vDevList = this.controller.devices.filter(function(x) { 
	return x.get("deviceType") === "switchBinary"; }); 

vDevTypes = this.controller.devices.map(function(x) { 
	return x.get("deviceType"); }); 
\end{listingverbatim}

\subsection{Virtual Device Usage / Commands}

In case the virtual device is an actor it will accept and execute a command using the 
syntax:

\paragraph{Vdev.performCommand(„name of the command“)}

The name of the accepted command should depend on the device type and can again be defined 
free of restrictions when implementing the virtual device. For auto-generated devices 
derived from Z-Wave the following commands are typically implemented.

\begin{enumerate}
\item 'update': updates a sensor value
\item 'on': turns a device on.  Only valid for binary commands
\item 'off': turns a device off. Only valid for binary commands
\item 'exact': sets the device to an exact value. This will be a temperature for 
thermostats or a percentage value of motor controls or dimmers
\end{enumerate}

\subsection{Virtual Device Usage / Values}

Virtual devices have inner values. They are called metrics. A metric can be set and get. 
Each virtual device can define its own metrics. Metrics can be level, title icon and 
other device specific values like scale (\%, kWh, ...)

\begin{listingverbatim}
vDev.set("metrics:...", ...);  
vDev.get("metrics:...");
\end{listingverbatim}

\subsection{How to create your own virtual devices}

A Virtual Device (Vdev) is an instance of a VirtualDevice class' descendant which exposes 
set of metrics and commands (according to it's type/subtype). Virtual devices are the only 
runtime instances which is controllable and observable through the JS API.

Technically, VDev is a VirtualDevice subclass which concretize, overrides or extends 
superclass' methods.

\subsubsection{Step 1. Define a VirtualDevice subclass}

\begin{listingverbatim}
// Important: constructor SHOULD always be successful
BatteryPollingDevice = function (id, controller) {
    // Always call superconstructor first
    BatteryPollingDevice.super_.call(this, id, controller);

    // Define VDevs properties
    this.deviceType = "virtual";
    this.deviceSubType = "batteryPolling";
    this.widgetClass = "BatteryStatusWidget";

    // Setup some additional metrics (many of them is setted up in a base class)
    this.setMetricValue("someMetric", "someValue");
}
inherits(BatteryPollingDevice, VirtualDevice);
\end{listingverbatim}

VDev class should always fill in the deviceType property and often fill in the 
deviceSubType property.

If the particular VDev class can be controller by the client-side widget, it should define 
widget's class name in the widgetClass property.

\subsubsection{Step 2. Override performCommand() method}

\begin{listingverbatim}
BatteryPollingDevice.prototype.performCommand = function (command) {
    var handled = true;

    if ("update" === command) {
        for (var id in zway.devices) {
            zway.devices[id].Battery && zway.devices[id].Battery.Get();
        }
    } else {
        handled = false;
    }

    return handled ? true : 
    BatteryPollingDevice.super_.prototype.performCommand.call(this, command);
}
\end{listingverbatim}


VDev itself mostly needed to handle commands, triggered by the events, system or the API.

In the example above you could see, that this VDev is capable of performing "update" 
command. But base class can be capable of performing some other commands, so the last l
ine calls superclass' performCommand() method if the particular command wasn't handled 
by the VDev itself.

This extensibility provides the possibility to create a VDev class tree. Take a look at 
ZWaveGate module as an example of such tree.

\subsubsection{Step 3. Instantiate your VDev by the module}

\begin{listingverbatim}
// ...part of the BatteryPolling.init() method
executeFile(this.moduleBasePath()+"/BatteryPollingDevice.js");
this.vdev = new BatteryPollingDevice("BatteryPolling", this.controller);
\end{listingverbatim}

First line of code is loads and executes apropriate .js-file which provides 
BatteryPollingDevice class.

Secnd line instantiates this class.

The last line calls controller's registerDevice method to register and VDev instance.

\subsubsection{Step 4. Register device}

\begin{listingverbatim}[Register Device]
        vDev = this.controller.devices.create(vDevId, {
            deviceType: "deviceType",
            metrics: {
                level: "level",
                icon: "icon from lib or url"
                title: "Default title"
            }
        }, function (command, ...) {
                // handles actions with the widget
        });  
\end{listingverbatim}

\subsubsection{Step 5: Unregister device}

Devices can be deleted or unregistered  using the following command:

\paragraph{this.controller.devices.remove(vDevId)}

\subsection{Binding to metric changes}

The metric - the inner variables of the vDev a changed by the system automatically.
In order to perform certain functions on these changes the function needs to be 
bound to the change to the vdev. The syntax for this is

\paragraph{vDev.on('change:metrics:...", function (vDev) { ... });}


Unbinding then works as one can expect:

\paragraph{vDev.off(’change:metrics:...”, function (vDev) ... )} 

\section{The event bus}
\label{cap:eventbus}

All communication from and to the automation modules is handled by events. An event 
is a structure containing certain information that is exchanged using a central 
distribution place, {\bf the event bus}. This means that all modules can send events 
to the event bus and can listen to event in order to execute commands on them. All 
modules can 'see' all events but need to filter out their events of relevance.  The 
core objects of the automation are written in JS and they are available as source 
code in the sub folder 'classes':

\begin{itemize}
\item AutomationController.js: This is the main engine of the automation function
\item AutomationModule.js: the basic object for the module
\end{itemize}

The file main.js is the startup file for the automation system and it is loading the three 
classes just mentioned. The subfolder /lib contains the key JS script for the Event 
handling: eventemitter.js.

\subsection{Emitting events}

The 'Event emitter' emits events into the central event bus. The event emitter can 
be called from all modules and scripts of the automation system. The syntax is:

\begin{quote}
\texttt{  
$controller.emit(eventName, args1,arg2,...argn)$
}
\end{quote}

The event name 'eventName' has to be noted in the form of 'XXX.YYY' where 'XXX' is the name 
of the event source (e.g. the name of the module issuing the event or the name of the 
module using the event) and 'YYY' is the name of the event itself.  To allow a scalable 
system it makes sense to name the events by the name of the module that is supposed to 
receive and to manage events. This simplifies the filtering of these events by the 
receiver module(s).

Certain event names are forbidden for general use because they are already used in the 
existing modules. One example are events with the name cron.XXXX that are used by the 
cron module handling all timer related events.

Every event can have a list of arguments developers can decide on. For the events used by 
preloaded modules (first and foremost the cron module) this argument structure is 
predefined. For all other modules the developer is free to decide on structure and content. 
It is also possible to have list fields and or any other structure as argument for the event

One example of an issued event can be 

\begin{quote}
\texttt{  
$emit(“mymodule.testevent”,”Test”,[“event1”,”event2”])$
}
\end{quote}

\subsection{Catching (binding to) events}

The controller object, part of every module, offers a function called 'on()' to catch events. 
The 'on(name, function())' function subscribes to events of a certain name type. If not 
all  events of a certain name tag shall be processed a further filtering needs to be 
implemented  processing  the further arguments of the event. The function argument contains a reference 
to the implementation using the event to perform certain actions. The argument list of the event is 
handed over to this function in its order but need to be declared in the function call statement.

\begin{quote}
\texttt{  
this.controller.on(“mymodule.testevent”, function (name,eventarray) {})
}
\end{quote}

The same way objects can unbind from events:

\begin{quote}
\texttt{  
this.controller.off(“mymodule.testevent”, function (name,eventarray));
}
\end{quote}

\subsection{Notification and Severity}

Notifications are a special  kind of event to inform the user on the graphical user interface or 
out-of-band.. This means that normal events are typically describes with numbers or ids 
while notifications contain a human readable message. 
 
The UI can be notified on the certain events.

\paragraph{this.controller.addNotification("....severity....", "....message....", "....origin....");} 

The parameters define
\begin{itemize}
\item severity is error, info, debug; 
\item origin describes which part of the system it is about: core, module, device, battery.
\end{itemize}

The controller can act on notifications or disable them.

\paragraph{this.controller.on('notifications.push', this.handler);}
\paragraph{this.controller.off('notifications.push', this.handler);}

\section{Modules (for users called 'Apps')}
\label{cap:modules}

Beside the core functions encoded into the JS core there are extensions to this code called
modules. Modules extend the JS core by providing internal or external ( visible to the user)
functions.

Each modules code is located in a sub directory of the sub folder module as described 
in chapter \ref{c11:zwayfolder}. The name of the subfolder equals the name of the module. The sub folder 
contains files to define the behavior of the module.
{\footnotesize
\begin{forest}
  for tree={
    font=\ttfamily,
    grow'=0,
    child anchor=west,
    parent anchor=south,
    anchor=west,
    calign=first,
    edge path={
      \noexpand\path [draw, \forestoption{edge}]
      (!u.south west) +(7.5pt,0) |- node[fill,inner sep=1.25pt] {} (.child anchor)\forestoption{edge label};
    },
    before typesetting nodes={
      if n=1
        {insert before={[,phantom]}}
        {}
    },
    fit=band,
    before computing xy={l=15pt},
  }
[ModuleName
[Module.json: The Manifest file of the Module]
[index.js: The main JS file]
[htdocs: ressources accessible by the web server]
[lang: translation into local languages] 
]
\end{forest}
}
\subsection{Module.json}

This file contains the module meta-definition used by the AutomationController. It must 
be a valid JSON object with the following fields (all of them are required):
\begin{itemize}
\item \textbf{autoload} — Boolean, defines will this module automatically instantiated during Home Automation startup.
\item \textbf{singleton} — Boolean, defines this module can be instantiated more than one time or not.
\item \textbf{defaults} — Object, default module instance settings. This object will be patched with the particular 
config object from the controller's configuration and resulting object will be passed to the initializer.
\item \textbf{actions} — Object, defines exported module instance actions. Object keys 
are the names of actions and  values are meta-definitions of exported actions used by 
AutomationController and API webserver.
\item \textbf{metrics} — Object, defines exported module metrics.
\end{itemize}
All configuration fields are required. Types of the object must be equal in every 
definition in every case. For instance, if module doesn't export any metric corresponding 
key value should be and empty object “{}”.

\subsection{index.js}
 
This script defines an automation module class which is descendant of AutomationModule base class.
During initialization the module script must define the variable '\_module' containing the particular module class.


Example of a minimal automation module:

\begin{listingverbatim}[Minimal Module]

function SampleModule (id, controller) {
SampleModule.super_.call.init(this, id, controller);

this.greeting = "Hello,World!";
}

inherits(SampleModule, AutomationModule);
_module = SampleModule;

SampleModule.prototype.init = function () {
    this.sayHello();
}

SampleModule.prototype.sayHello = function () {
    debugPrint(this.greeting);
} 

SampleModule.prototype.stop = function () {
    this.sayByeBye();
}    

\end{listingverbatim}
 
The first part of the code illustrates how to define a class function named SampleModule 
that calls the superclass' constructor. Its highly recommended not to do further 
instantiations in the constructur. Initializations should be implemented within the 
'init' function.
 
The second part of the code is almost immutable for any module. It calls prototypal 
inheritance support routine and it fills in \_module variable.

The third part of the sample code defines module's init() method which is an 
instance initializer. This initializer must call the superclass's initializer prior to 
all other tasks. In the initializer module can setup it's private environment, subscribe 
to the events and do any other stuff.
Sometimes, whole module code can be placed withing the initializer without creation of 
any other class's methods. As the reference of such approach you can examine AutoOff 
module source code.

After the init function a module may contain other functions. The 'sayHello' function 
of the Sample Module shows this as example.

\subsection{Available Core Modules}

All modules in \zway are designed the same way using the same file structure but they 
serve different purposes and they are of different importance:
\begin{itemize}
\item Core Module are modules that provide essential parts of the \zway system. They run
from the beginning and should not be terminated without good reason. Normal users will
not even see them in the list of active apps. Users with management privilege can set a 
checkbox in their \keystroke{My Settings} to unhide them.
\item Standard Modules can be started and stopped by the user. they are already in the 
subfolder 'modules' and can not be deleted.
\item Modules from the Online Service must be downloaded first before they can be used. 
They can be started, stopped and even removed. They are stored in the folder 'userModules'.
\end{itemize}
The two core modules are worth to be explained in detail:

\subsubsection{Cron, the timer module}

All time driven actions need a timer. The \zway automation engine implement a cron-type timer 
system as a module as well. The basic function of the cron module is

\begin{itemize}
\item It accepts registration of events that are triggered periodically
\item It allows to de-register such events. 
\end{itemize}


The registration and deregistration of events is also handled using the event mechanism. 
The cron module is listening for events with the tags 'cron.addTask' and  'cron.removeTask'. 
The first argument of these events are the name of the event fired by the cron module. The 
second argument of the 'addTask' event is an array desricing the times when this event 
shall be issued. It has the format: 
\begin{itemize}
\item Minute [start,stop, step] or 0-59 or null
\item Hour [start,stop, step] or 0-23 or null
\item weekDay [start,stop, step] or 0-6 or null
\item dayOfMonth [start,stop, step] or 1-31 or null
\item Month [start,stop, step] or 1-12 or null
\end{itemize}
The argument for the different time parameters has one of three formats
\begin{itemize}
\item null: the event will be fired on every minute or hour etc.
\item single value: the event will be fired when the value reaches the given value
\item array [start, stop, step]: The event will be fired between start and stop in steps.
\end{itemize}
 
 The object  
\begin{quote}
{\tt  $\{minute: null,hour: null,weekDay: null, day: null, month: null\} $   }
\end{quote} 

will fire every minute within every hour within every weekday on every day of the month 
every month. Another example of an event emitted towards the cron 
module for registering an timer event can be found in the Battery Polling Module:

\begin{listingverbatim}[Registering a Battery Polling Command]
    this.controller.emit("cron.addTask", "batteryPolling.poll", {
        minute: 0,
        hour: 0,
        weekDay: this.config.launchWeekDay,
        day: null,
        month: null        
    });
\end{listingverbatim}

This call will cause the cron module to emit an event at night (00:00) on a day 
that is defined in the configuration variable this.config.launchWeekDay, e.g. 0 = Sunday.

The 'cron.removeTask' only needs the name of the registered event to deregister.

\subsubsection{Z-Wave}

The whole mapping of Z-Wave devices into virtual devices is handled by a module called 
\app{ZWave}. This module is quite powerful. It does not only manage the mapping but handles
various Z-Wave specific functions such as timing recording, etc.

