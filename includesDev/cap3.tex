\chapter{JavaScript API}
\label{jsapi}

The JavaScript API mirrors all functions of the Z-Wave device API and 
combines them with the ability to run JavaScript code on top of the 
Z-Wave Device functions and variables.

There are two ways to run JavaScript functions in Z-Way.
\begin{itemize}
\item They can be executed in the web browser URL string (using /JS/Run/ prefix)
\item They can be implemented as module running in the backend or be stored in a file on the server side.
\end{itemize}
Both options have their pros and cons. Running JS code in the browser is a very nice
and convenient way to test things but the function is not persistent.

Writing a module requires some more knowledge and debugging is more complicated. 
On the other hand the possibilities of JavaScript in the module are almost infinite 
and goes far beyond just accessing Z-Wave device. JavaScript as any other language
can make use of services available in the Internet and combine them in any possible 
way with information from the Z-Wave network and can execute functions within 
the Z-Wave network but the same time on any place accessible via Internet based 
services. In theory there is not even a need to have Z-Wave device in order to make 
use of the powerful JavaScript engine. As an example you can write a JavaScript 
module polling weather data from the internet and depending on certain well defined 
conditions the same module can send you a short message on your mobile.
Both functions are by the way already implemented as open source modules and can be 
accessed and studied for further modification or use.

\section {The JavaScript Engine}

Z-Way uses the JavaScript engine provided by Google referred to as V8. You find more 
information about this JavaScript implementation on https://code.google.com/p/v8/.
V8 implements JavaScript according to the specification ECMA 5
\footnote {http://www.ecma-international.org/publications/standards/Ecma-262.htm}.

Z-Way extends the basic functionality provided by V8 with plenty of application 
specific functions.

\section{Accessing the JS API}

The JS API can be accessed from any web browser with the URL

\paragraph{http://YOURIP:8083/JS/Run/*}

All functions of the Z-Wave Device API can be used by JavaScript. They are encapsulated
in the 'zway' object.  This object has the same structure as defined in chapter 
\ref{c2-data model}. 
The client side access to the device data is done like

\paragraph{http://YOURIP:8083/JS/Run/zway.devices[x].*}

Due to the scripting nature of JavaScript its possible to 'inject' code at run time
using the interface. Here a nice example how to use the Java Script 
setInterval function:

\begin{lstlisting}[caption=Polling device \#2]{Name}
/JS/Run/setInterval(function() { 
	zway.devices[2].Basic.Get();
}, 300*1000);
\end{lstlisting}

This code will, once 'executed' as URL within a web browser, call the Get() command
of the command class Basic of Node ID 2 every 300 seconds.  

A very powerful function of the JS API is the ability to bind functions to certain
values of the device tree. they get then executed when the value changes. Here an 
example for this binding. The device No. 3 has a command class SensorMultilevel that offers
the variable level. The following call - both available on the client side 
and on the server side - will bind a simple alert function to the change of 
the variable.

\begin{lstlisting}[caption=Bind a function]{Name}
zway.devices[3].SensorMultilevel.data[1].val.bind( function() { 
	debugPrint('CHANGED TO: ' + this.value + '\n'); 
});  
\end{lstlisting}

\section{HTTP Access}

The JavaScript implementation of Z-Way allows to directly accessing HTTP objects.

The http request is much like jQuery.ajax(): r = http.request(options);

Here's the list of options:
\begin{itemize}
\item url - required. Url you want to request (might be http, https, or maybe even ftp);
\item method – optional. HTTP method to use (currently one of GET, POST, HEAD). If not 
specified, GET is used;
\item headers – optional. Object containing additional headers to pass to server:

\begin{lstlisting}
headers: {
    "Content-Type": "text/xml",
    "X-Requested-With": "RaZberry/1.5.0"
}
\end{lstlisting}

\item data – used only for POST requests. Data to post to the server. May be either a
string (to post raw data) or an object with keys and values (will be serialized as 
'key1=value1\&key2=value2\&…');
\item auth – optional. Provides credentials for basic authentication. It is an object 
containing login and password:
\begin{lstlisting}
auth: {
    login: 'username',
    password: 'secret'
}
\end{lstlisting}
\item contentType – optional. Allows to override content type returned by server for 
parsing data (see below);
\item async – optional. Specifies whether request should be sent asynchronously. Default 
is false. In case of synchronous request result is returned immediately (as function 
return value), otherwise function exits immediately, and response is delivered later 
thru callbacks.
\item success, error and complete – optional, valid only for async requests. Success 
callback is called after successful request, error is called on failure, complete is 
called nevertheless (even if success/error callback produces exception, so it is like 
'finally' statement);
\end{itemize}

Response (as stated above) is delivered either as function return value, or as callback 
parameter. Is is always an object containing following members:

\begin{itemize}
\item status – HTTP status code (or -1 if some non-HTTP error occurred). Status codes 
from 200 to 299 are considered success;
\item statusText – status string;
\item URL – response URL (might differ from url requested in case of server redirects);
\item headers – object containing all the headers returned by server;
\item contentType – content type returned by server;
\item data – response data.
\end{itemize}


Response data is handled differently depending on content type (if contentType on request is set, it takes priority over server content type):
\begin{itemize}
\item application/json and text/x-json are returned as JSON object;
\item application/xml and text/xml are returned as XML object;
\item application/octet-stream is returned as binary ArrayBuffer;
\item string is returned otherwise.
\end{itemize}
In case data cannot be parsed as valid JSON/XML, it is still returned as string, and additional parseError member is present.


\begin{lstlisting}
http.request({
	url: "http://server.com" (string, required),
	method: "GET" (GET/POST/HEAD, optional, default "GET"),
	
	headers: (object, optional)
	{
		"name": "value",
		...
	},
	
	auth: (object, optional)
	{
		"login": "xxx" (string, required),
		"password": "***" (string, required)
	},
	
	data: (object, optional, for POST only)
	{
		"name": "value",
		...
	}
	-- OR --
	data: "name=value&..." (string, optional, for POST only),

	async: true (boolean, optional, default false),
	
	success: function(rsp) {} (function, optional, for async only),
	error: function(rsp) {} (function, optional, for async only),
	complete: function(rsp) {} (function, optional, for async only)
});


response:
{
	status: 200 (integer, -1 for non-http errors),
	statusText: "OK" (string),
	url: "http://server.com" (string),
	contentType: "text/html" (string),
	headers: (object)
	{
		"name": "value"
	},
	data: result (object or string, depending on content type)
}
\end{lstlisting}

\section{XML parser}

ZXmlDocument object allows to convert any valid XML document into a JSON object and vice versa.

\subsection{var x = new ZXmlDocument()}
Create new empty XML document

\subsection{x = new ZXmlDocument("xml content")}
Create new XML document from a string

\subsection{x.root}
Get/set document root element. Elements are got/set in form of JS objects:

\begin{lstlisting}
{
    name: "node_name", – mandatory
    text: "value", – optional, for text nodes
    attributes: { – optional
    	name: "value",
    	...
    },
    children: [ – optional, should contain a valid object of same type
    	{ ... }
    ]
}
\end{lstlisting}

For example:
\begin{lstlisting}
(new ZXmlDocument('<weather><city id="1"><name>Zwickau</name><temp>2.6</temp></city><city id="2"><name>Moscow</name><temp>-23.4</temp></city></weather>')).root =
{  
   "children":[  
      {  
         "children":[  
            {  
               "text":"Zwickau",
               "name":"name"
            },
            {  
               "text":"2.6",
               "name":"temp"
            }
         ],
         "attributes":{  
            "id":"1"
         },
         "name":"city"
      },
      {  
         "children":[  
            {  
               "text":"Moscow",
               "name":"name"
            },
            {  
               "text":"-23.4",
               "name":"temp"
            }
         ],
         "attributes":{  
            "id":"2"
         },
         "name":"city"
      }
   ],
   "name":"weather"
}
\end{lstlisting}

\subsection{x.isXML}
This hidden readonly property allows to detect if object is XML object or not (it is always true).

\subsection{x.toString()}
Converts XML object into a string with valid XML content.

\subsection{x.findOne(XPathString)}
Returns first matching to XPathString element or null if not found.
\begin{lstlisting}
x.findOne('/weather/city[@id="2"]') // returns only city tag for Moscow
x.findOne('/weather/city[name="Moscow"]/temp/text()') // returns temperature in Moscow
\end{lstlisting}

\subsection{x.findAll(XPathString)}
Returns array of all matching to XPathString elements or empty array if not found.
\begin{lstlisting}
x.findAll('/weather/city') // returns all city tags
x.findAll('/weather/city/name/text()') // returns all city names
\end{lstlisting}

\subsection{XML elements}
Each XML element (tag) in addition to properties described above (text, attributes, children) have hidden readonly property parent pointing to parent object and the following methiods:
\begin{itemize}
\item insertChild(element) Insert new child eleemnt
\item removeChild(element) Remove child element
\item findOne(XPathString) Same as on root object, but relative (no leading / needed in XPathString
\item findAll(XPathString) Same as on root object, but relative (no leading / needed in XPathString
\end{itemize}

ZXmlDocument is returned from http.request() when content type is "application/xml", "text/xml" or any other ending with "+xml". Namespaces are not yet supported.

\section{Cryptographic functions}

crypto object provides access to some popular cryptographic functions such
as SHA1, SHA256, SHA512, MD5, HMAC, and provides good random numbers.

\subsection{var guid = crypto.guid()}
Provides standard GUID in string format.

\subsection{var rnd = crypto.random(n)}
Generates n random bytes.
Returned values is of type ArrayBuffer. To convert it into array use this trick:
\begin{lstlisting}
	rnd = (new Uint8Array(crypto.random(10)));
\end{lstlisting}

\subsection{var dgst = crypto.digest(hash, data, ...)}
Returns digest calculated using selected hash algorithm. It supports virtually all the algorithms available in OpenSSL (md4, md5, mdc2, sha, sha1, sha224, sha256, sha384, sha512, ripemd160).
If no data parameters specified, it returns a digest of an empty value. If more than one data parameters are specified, they're all used to calculate the result. Data parameters may be of different types (strings, arrays, ArrayBuffers).
Return value is of type ArrayBuffer.

There are also a few shortcut functions for popular algorithms: "md5", "sha1", "sha256", "sha512". For example, these calls are equivalent:

\begin{lstlisting}
	dgst = crypto.digest("sha256", data);
	dgst = crypto.sha256(data);
\end{lstlisting}

\subsection{var hmac = crypto.hmac(cipher, key, data, ...)}
Returns hmac calculated using selected hash algorithm. Hash algorithms are the same as for digest() function.
Key parameter is required. 
If no data parameters specified, it returns a HMAC of an empty value. If more than one data parameters are specified, they're all used to calculate the result. Key and data parameters may be of different types (strings, arrays, ArrayBuffers).
Return value is of type ArrayBuffer.

There are also a few shortcut functions for popular algorithms: "hmac256", "hmac512". For example, these calls are equivalent:

\begin{lstlisting}
  dgst = crypto.hmac("sha256", key, data);
  dgst = crypto.hmac256(key, data);
\end{lstlisting}

\section{Sockets functions}

Socket module allows easy access to TCP and UDP sockets from JavaScript.
Both connection to distant ports and listening on local are available. This API fully mirrors into JavaScript POSIX TCP/IP sockets.
This can be used to control third party devices like Global Cache or Sonos
as well as emulating third party services.

To start communications one need to create socket and either
\textbf{connect} it or \textbf{listen} it. \textbf{onrecv} method is called
on data receive from remote, while \textbf{send} is used to send data to remote side.

The example below dumps to log file response to http://ya.ru:80/ (raw HTTP
protocol is used as an example).

\begin{lstlisting}
var sock = new sockets.tcp();

sock.onrecv = function(data) {
    debugPrint(data.byteLength);
};

sock.connect("ya.ru", 80);

sock.send("GET / HTTP/1.0\r\n\r\n");
\end{lstlisting}

Here is an example of TCP echo server on port 8888:

\begin{lstlisting}
var sock = new sockets.tcp();

sock.bind(8888);

sock.onrecv = function(data) {
    this.send(data);
};

sock.listen();
\end{lstlisting}

And echo server for UDP:
\begin{lstlisting}
var sock = new sockets.udp();

sock.bind(8888);

sock.onrecv = function(data, host, port) {
    this.sendto(data, host, port);
};

sock.listen();
\end{lstlisting}

Detailed description of Socket API:
\begin{itemize}
\item bind(ip, port) or bind(port) binds socket to port (integer number). ip should be a string like "192.168.0.1". If omited "0.0.0.0" is used (bind on all IP addresses of all interfaces). Returns false on error.
\item connect(ip, port) connects to remote side ip:port. TCP sockets requires this call before sending data. For UDP sockets it is optional, but once used allows to use send call instead of sendto call. Returns false on error.
\item listen() starts listening port (this is required not only for TCP, but for UDP too). Returns false on error.
\item close() initiate close of socket.
\item send(data) sends data to connected or accepted socket.
\item sendto(data, host, port) sends data to a non-connected UDP socket.
\item onrecv(data, host, port) called on new data receiption from remote side. For UDP sockets and connected TCP sockets "this" object reffers to the socket itself, while for accepted TCP sockets "this" reffers to the client's individual objects.
\item onconnect(remoteHost, remotePort, localHost, localPort) called only for TCP sockets on new connection accept. "this" reffers to the client individual socket object.
\item onclose(remoteHost, remotePort, localHost, localPort) called on socket close by remote or due to close() call. Note that for TCP sockets this callback is called for client sockets on connection close and for binded listening socket if close() is called. "this" object will be defined like in onrecv.
\item reusable() sets SO\_REUSEADDR socket option to allow multiple bind() on the same port.
\item broadcast() sets SO\_BROADCAST socket option to allow sending broadcast UDP messages.
\item multicastAddMembership(multicastGroup) subscribe socket to multicast group
\item multicastDropMembership(multicastGroup) unsubscribe socket from multicast group
\end{itemize}

\subsection{WebSockets functions}

Socket module also implements WebSockets (RFC 6455). WebSocket API is made to be
compatible with browser implementations (some rarely used functions are not
implemented - see below).

The example below implemnts basic application using WebSockets client:

\begin{lstlisting}
var sock = new sockets.websocket("ws://echo.websocket.org");

sock.onopen = function () {
    debugPrint('connected, sending ping');
    sock.send('ping');
}

sock.onmessage = function(ev) {
    debugPrint('recv', ev.data);
}

sock.onclose = function() {
    debugPrint('closed');
}

sock.onerror = function(ev) {
    debugPrint('error', ev.data);
}
\end{lstlisting}

Next example shows basic application using WebSockets server:

\begin{lstlisting}
var sock = new sockets.websocket(9009);

sock.onconnect = function () {
    debugPrint('client connected, sending ping');
}

sock.onmessage = function(ev) {
    debugPrint('recv', ev.data);
    sock.send('pong');
}

sock.onclose = function() {
    if (this === sock) {
        debugPrint('server websocket closed');
    } else {
        debugPrint('client disconnected');
    }
}

sock.onerror = function(ev) {
    debugPrint('error', ev.data);
}
\end{lstlisting}

Detailed description of WebSocket API:
\begin{itemize}
\item socket.websocket(url, [protocol]) creates new client WebSocket and connects to the specified URL (should be a string like "ws://host:port" or "wss://host:port" for SSL channel). Optional protocol parameter can be used to specify protocol from server capabilities (comma separated string), default is "default".
\item socket.websocket(port) creates new WebSocket server on port.
\item close() initiate close of WebSocket.
\item send(data) sends data to WebSocket. data can be array, ArrayBuffer (sent as binary) or string (sent as text).
\item onmessage(event) called on new data receiption from remote side. Object event contains only data property. Other properties mentioned in RFC 6455 are not supported.
\item onopen() called on connection establish. Compared RFC 6455 event parameter is not passed.
\item onclose() called on WebSocket close by remote or due to close() call. For server side will be called on client instance and on listening instance (use this to differenciate). Compared RFC 6455 event parameter is not passed.
\item onerror(event) called on error. Parameter event contains only propery data. Other properties from RFC 6455 are not implemented.
\end{itemize}

\section{Other JavaScript Extensions}

\subsubsection{fs.list(folder)}

This returns list of items in the folder or undefined if not folder is not existing.


\subsubsection{fs.stat(file)}

This returns one of the following values:

\begin{itemize}
\item 1) undefined if object does not exist or not readable
\item 2) object \{ type: 'file', size: \textless{}size\textgreater{}\} if it is a file
\item 3) object \{ type: 'dir' \} if it is a folder
\end{itemize} 


\subsubsection{fs.loadJSON(filename)}

This function reads a file from the file system and loads it into the memory. The file must contain a valid JSON object. The only argument is the name of the file including relative pathname to the automation folder. The functions returns the full JSON object or null in case of error.

\subsubsection{fs.load(filename)}

This function reads a file from the file system and returns it's content as a string. The only argument is the name of the file including relative pathname to the automation folder. The functions returns null in case of error.

\subsubsection{executeFile(filename) and executeJS(string)}

Loads and executes a particular JavaScript file from the local filesystem or executes JavaScript code represented in string (like eval in browsers).

The script is executed withig the global namespace.

Remark: If an error occurred during the execution it won't stop from further execution, but erroneous script will not be executed completely. It will stop on the first error.
Exceptions in the callee can be trapped in the caller using standard try-catch mechanism.

\subsubsection{system(command)}

The command system() allows to execute any shell level command available on the operating 
system. It will return the shell output of the command.  On default the execution of 
system commands is forbidden. Each command executed need to be permitted by putting one 
line with the starting commands in the file automation/.syscommands or in an different 
automation folder as specified in config.xml.

\subsubsection{Timers}
Timers are implemented exactly as they are used in browsers. They are very helpfull for periodical and delayed operations. Timeout/period is defined in milliseconds.
\begin{itemize}
\item timerId = setTimeout(function() { }, timeout)
\item timerId = setInterval(function() { }, period)
\item clearTimeout(timerId)
\item clearInterval(timerId)
\end{itemize}

\subsubsection{loadObject(object\_name) and saveObject(object\_name, object)}
Loads and saves JSON object from/to storage. These functions implements flat storage for application with access to the object by it's name. No folders are available.

Data is saved in automation/storage folder. Filenames are made from object names by stripping characters but [a-ZA-Z0-9] and adding checksum from original name (to avoid name conflicts).

\subsubsection{exit()}
Stops JavaScript engine and shuts down Z-Way server


\subsubsection{allowExternalAccess(handlerName) and listExternalAccess()}
allowExternalAccess allows to register HTTP handler. handlerName can contain strings like aaa.bbb.ccc.ddd - in that case any HTTP request starting by /aaa/bbb/ccc/ddd will be handled by a function aaa.bbb.ccc.ddd() if present, otherwise aaa.bbb.ccc(), ... up to aaa().
Handler should return object with at least properties status and body (one can also specify headers like it was in http.request module).

listExternalAccess returns array with names of all registered HTTP handlers.

Here is an example how to attach handlers for /what/timeisit and /what:

\begin{lstlisting}
what = function() {
  return { status: 500, body: 'What do you want to know' };
};

what.timeisit = function() {
  return { status: 200, body: (new Date()).toString() }
};

allowExternalAccess("what");
allowExternalAccess("what.timeisit");
\end{lstlisting}

\subsubsection{debugPrint(object, object, ...)}

Prints arguments converted to string to Z-Way console. Very usefull for debuggin.
For convenience one can map 'console.log()' to debugPrint().

This is how it was done in automation/main.js in Z-Way Home Automation engine:
\begin{lstlisting}
var console = {
    log: debugPrint,
    warn: debugPrint,
    error: debugPrint,
    debug: debugPrint,
    logJS: function() {
        var arr = [];
        for (var key in arguments)
            arr.push(JSON.stringify(arguments[key]));
        debugPrint(arr);
    }
};
\end{lstlisting}

\subsection{Debugging JavaScript code}
Change in config.xml debug-port to 8183 (or some other) turn on V8 debugger capability on Z-Way start.

\begin{lstlisting}
<config>
    ...
    <debug-port>8183</debug-port>
    ....
</config>
\end{lstlisting}

node-inspector debugger tool is required. It provides web-based UI for debugging similar to Google Chrome debug console.

You might want to run debugger tool on another machine (for example if it is not possible to install it on the same box as Z-Way is running on).

Use the following command to forward debugger port defined in config.xml to your local machine:
\begin{lstlisting}
ssh -N USER@IP_OF_Z-WAY_MACHINE -L 8183:127.0.0.1:8183
\end{lstlisting}
(for RaZberry USER is pi)

Install node-inspector debugger tool and run it:
\begin{lstlisting}
npm install -g node-inspector
node-inspector --debug-port 8183
\end{lstlisting}

Then you can connect to http://IP\_OF\_MACHINE\_WITH\_NODE\_INSPECTOR:8080/debug?port=8183

If debugging is turned on, Z-Way gives you 5 seconds during startup to reconnect debugger to Z-Way (refresh the page of debugger Web UI withing these 5 seconds).
This allows you to debug startup code of Z-Way JavaScript engine from the very first line of code.
