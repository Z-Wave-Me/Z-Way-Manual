\section{File System}

All Z-Way files of RaZberry are located in the folder /data and its subfolders

\subsection{config.xml}

This file contains the global settings to get the code running


\begin{itemize}
\item device: The serial device of the Transceiver chip. For RaZberry this is /dev/AMA0
\item config-dir: Optioh to change folder of config files
\item translations-dir: Optioh to change folder of translations files
\item zddx-dir: Option to change folder of zddx files
\item port: port of the web server
\item http-root-dir: root file system of web server
\item automation-file: name of automation main faile
\item shell-script-on-save-xml: points to shell script for backup
\end{itemize}

\subsection{Subfolder /ZDDX}

This folder contains the Device Description Records (DDR). These DDRs are human readable XML files providing device information not accessible from the device itself. Z-Way determines the correct Device Type identified by the Vendor ID and vendor specific Product IDs.  The XML files contain among others verbal descriptions of the association groups and configuration parameters.

\subsection{Subfolder config}

This subfolder contains installation specific settings:

\begin{itemize}
\item File Configuration.xml: The default configuration values of different device types. This is autogenerated.
\item File Defaults.xml: Some basic definitions of Z-Way. This may need manual changes.
\item File Rules.xml: user defined device names, maps, etc. This is autogenerated.
\item Folder Maps: The pictures of the floor plan
\item Folder zddx: the device data of included devices plus the whole device data tree . 
\end{itemize}

\subsubsection{Options of File Defaults.xml}

\begin{itemize}
\item AutoConfig: Flag if Z-Way shall interview the device right after inclusion (default = 1)
\item DeepInterview: Flag that Interview is only completed after all values are received back. This includes
asking the device for all initial values of sensor or status data (default = 1)
\item SaveDataAfterInterviewSteps: Flag whether or not all device data shall be saved after each interview step (default = 1)
\item TryToBecomeSIS: Shall Z-Way try to become Networks SIS if transceiver hardware allows to (default = 1)
\item Controller: Description how Z-Way shall behave as device in the network. This entry has the following subentries
\begin{itemize}
\item NodeInformationFrame: The command classes Z-Way is announcing as "supported" in the network
\item SecureNodeInformationFrame: Command Classes available in secure environment
\item InstanceNodeInformationFrame: Command Classes in Instances if Multi Channel is emulated
\item Version id =iD: Versions to be reported in Command Class Version Get Command for all Command Classes announced in NIF
\item Name: Default Node Name reported by NodeNaming Report
\item Location Default Node Location of Controller reported by NodeNaming Report
\item AppVersion: Application Version reported by ManufacturerSpecific Report 

\item ManufacturerSpecific: Values report by ManufacturerSpecific Report 
\item SpecificDeviceClass: Specific Device Class reported
\item GenericDeviceClass: Generic Device Class reported
\end{itemize} 
\item Command class 0x73 (Powerlevel) Timeout: Seconds until PowerLevel test will time out
\item Command class 0x73 (Powerlevel) MaxFrames: Number of Test-Frames sent out per power level 
\item Command class 0x84 (Wakeup) WakeupInterval: Default Wakeup Interval
\item Command class 0x2C (Scene Activation) Max Scenes: Maximum number of Scenes supported
\item Command class 0x2D (Scene Activation) Max Scenes: Maximum number of Scenes supported
\item Command class 0x75 (Protection) Mode: Default Protection Mode
\item Command class 0x31 (SensorMultilevel) Fahrenheit: Flag what temperature scale is used
\item Command class 0x27 (SwitchAll) Mode: default Switch All mode
\item Command class 0x60 (Multichannel) ChannelsNumber: max number of channels emulated
\item Command class 0x60 (Multichannel) GenericDeviceClass: Generic Device Class of simulated Channel
\item Command class 0x60 (Multichannel) SpecificDeviceClass: Specific Device Class of simulated Channel
\end{itemize} 


\subsection{Subfolder /htdocs}

This folder is the root file system of the built-in webserver. It contains the JavaScripts and HTMP pages of 
the Demo User Interface.

\subsection{Subfolder  /libs}

This folder contains the binary libraries of the z-way server.

\subsection{Subfolder /libzway-dev}

This folder contains the C Header Files for the C API. The logic behind the C API is similar to the JSON API.

\subsection{Subfolder /translations}

This subfolder contains several XML files to translate numerical of the Z-Wave protocol into human readable Strings.
\begin{itemize}
\item AEC.xml: Advanced Energy Frame Work descriptions
\item Alarms.xml: Alarm types
\item DeviceClasses.xml: Device Classes of the Z-Wave
\item RulesEvents.xml: different Event types
\item SDKIds.xml: the System Development Kit revision numbers.
\item Scales.xml. scales of Sensor Multilevel and Meter Command Class
\item ThermostatModes.xml: different thermostat modes
\item VendorIds.xml: The manufacturers of the Z-Wave devices.

\end{itemize}
\subsection{File  /z-way-server}

This is the executable of Z-Way

\section{Localization of Z-Way}

Z-Way has three language specific data storages that need to be updated in order to add new language support.
\begin{itemize}
\item The Web UI: In htdocs/js there is a one language translation file per language, e.g. language.en.js. A new file needs to be generated with the two character language code (e.g. ru for Russian, ro for Romanian) and all strings on the right part of semicolon (:) on each lines need to be translated
\item The Pepper-One Device Database: Z-Way takes all association group descriptions, device background info, configuration parameter and value descriptions from this database
\item The folder /translations of Z-Way contains language specific IDs for sensor value scales, vendor information.

\end{itemize}