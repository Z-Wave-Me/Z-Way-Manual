\subsection{JSON-API}

JSON API allows to execute commands on server side using HTTP POST requests (currently GET requests are also allowed, 
but this might be deprecated in future). The command to execute is taken from the URL.

All functions are executed in form 

\paragraph{http://IPOFYOURRASPBERRYPI:8083/$<$URL$>$}


\subsubsection{/ZWaveAPI/Run/$<$command$>$}

This executes zway.$<$command$>$ in JavaScript engine of the server. As an example to switch ON a device no 2 using the
command class BASIC (0x20) its possible to write:

\begin{quote}/ZWaveAPI/Run/devices[2].instances[0].commandClasses[0x20].Set(255)\end{quote}

or

\begin{quote}/ZWaveAPI/Run/devices[2].instances[0].Basic.Set(255)\end{quote}



The Z-Way demo GUI has a JS command runCmd($<$command$>$) to simplify such operations. This function 
is accessable in the Javascript console of your web browser (in Chrome you find the JavaScript 
console unter View-$>$Debug-$>$JS Console). Using this feature the command in JS console would looke like

\begin{quote}runCmd('devices[2].instances[0].Basic.Set(255)')\end{quote}




The usual way to access a command class is using the format 'devices[nodeId].instances[instanceId].
commandClasses[commandclassId]'.
There are ways to simplify the syntax:
\begin{itemize}
\item 'devices[nodeId].instances[instanceId].commandClasses.Basic' is equivalent to 'devices[nodeId].
instances[instanceId].commandClasses[0x20]'
\item The word 'commandClass' can be obmitted if the command class name is used: 'devices[nodeId].
instances[instanceId].Basic' is equivalent to 'devices[nodeId]. instances[instanceId]. commandClasses.Basic'
\item the instances[0] can be obmitted:  'devices[nodeId].instances[instanceId].Basic' then turns into 'devices[nodeId].
Basic'
\end{itemize} 

Each Instance object has a device property that refers to the parent device it belongs to. Each Command Class Object has a device and an instance property that
refers to the instance and the device this command class belongs to.
 
 
Data holder object have properties value, updateTime, invalidateTime, name, but for compatibility with JS and previous versions we have valueOf() method (allows to omit .value in JS code, hence write "data.level == 255"), updated (alias to updateTime), invalidated (alias toinvalidateTime). These aliases are not enumerated if the dataholder is requested (data.level returns {value: 255, name: "level", updatedTime: 12345678, invalidatedTime: 12345678}).



\subsubsection{/JS/Run/$<$command$>$}

Executes $<$command$>$ in JavaScript engine of the server and returns JSON object returned by JavaScript function.

For example to execute a request for binary sensor value every 5 minutes use:

\begin{quote}/JS/Run/setInterval(function() \{ zway.devices[2].instances[0].commandClasses[0x30].Get(); \}, 300*1000);\end{quote}

Note: /ZWaveAPI/Run/$<$command$>$ is an alias to /JS/Run/zway.$<$command$>$

The Z-Way demo GUI has a JS command runJS($<$command$>$) to simplify such operations. This function 
is accessable in the Javascript console of your web browser (in Chrome you find the JavaScript 
console unter View-$>$Debug-$>$JS Console). Using this feature the command in JS console would look like

\begin{quote}runJS('setInterval(function() \{zway.devices[2].instances[0].commandClasses[0x30].Get(); \}, 300*1000);')\end{quote}

\subsubsection{/ZWaveAPI/InspectQueue}

This function is used to visualize the Z-Way job queue. This is for debugging only but very useful to understand the 
current state of Z-Way engine.


\subsubsection{/ZWaveAPI/Data/$<$timestamp$>$}
Returns an associative array of changes in Z-Way data tree since $<$timestamp$>$. The array consists 
of ($<$path$>$: $<$JSON object$>$) object pairs. The client is supposed to assign the new $<$JSON object$>$ to 
the subtree with the <path> discarding previous content of that subtree. Zero (0) can be used instead 
of $<$timestamp$>$ to obtain the full Z-Way data tree.

The tree have same structure as the backend tree (Figure \ref{zwaystructure}) with one additional root element "updateTime" which 
contains the time of latest update. This "updateTime" value should be used in the next request for changes. 
All timestamps (including updateTime) corresponds to server local time.

Each node in the tree contains the following elements:
\begin{itemize}
\item value — the value itself
\item updateTime — timestamp of the last update of this particular value
\item invalidateTime — timestamp when the value was invalidated by issuing a Get command to a device and expecting a Report command from the device
\end{itemize}

The object looks like:
\begin{lstlisting}[caption=JSON Data Structure]{Name}
{
"[path from the root]": [updated subtree],
"[path from the root]": [updated subtree],
...
updateTime: [current timestamp]
}
\end{lstlisting}

Examples for Commands to update the data tree look like:

\begin{quote}Get all data: /ZWaveAPI/Data/0\end{quote}

\begin{quote} Get updates since 134500000 (Unix timestamp): /ZWaveAPI/Data/134500000\end{quote}

Please note that during data updates some values are updated by big subtrees. For example, in Meter Command Class value 
of a scale is always updated as a scale subtree by [scale].val object (containing scale and type descriptions).


\subsubsection{Handling of updates coming from Z-Way}

A good design of a UI is linking UI objects (label, textbox, slider, ...) to a certain path in the 
tree object. Any update of a subtree linked to UI will then update the UI too. This is called bindings.

For web applications Z-Way Web UI contains a library called jQuery.triggerPath (extention of jQuery written by 
Z-Wave.Me), that allows to make such links between objects in the tree and HTML DOM objects. Use

\begin{quote}var tree;\end{quote}
\begin{quote}jQuery.triggerPath.init(tree);\end{quote}

during web application initialization to attach the library to a tree object. Then run

\begin{quote}jQuery([objects selector]).bindPath([path with regexp], [updater function], [additional arguments]);\end{quote}

to make binding between path changes and updater function. The updater function would be called upon changes in 
the desired object with this pointing to the DOM object itself, first argument pointing to the updated object 
in the tree, second argument is the exact path of this object (fulfilling the regexp) and all other arguments 
copies additional arguments. RegExp allows only few control characters: * is a wildcard, (1|2|3) - is 1 or 
2 or 3. For example:

\begin{lstlisting}[caption=Subscription of Data Values]{Name}
$('span.light_level').bindPath('(devices[*].instances[0].commandClasses[38].
data.level|devices.*.instances.*.commandClasses[39].data.level)', 
function (obj, path, arg1, arg2) \{
	$(this).html(path + ': ' + obj.value);
	\}, "this will be passed to update parser", "this too");
\end{lstlisting}
And finally here is an example of function handling updates (it is supposed to be called each 1-10 seconds):

\begin{lstlisting}[caption=Data Update]{Name}

// Call this function periodically or manually to get updates.
// Parameter sync allows to make request synchronous or asynchronous

function getDataUpdate(sync) {
	$.postJSON('/ZWaveAPI/Data/' + tree.updateTime, handlerDataUpdate, sync);
	}

// Wrapper that makes AJAX request
$.postJSON = function(url, data, callback, sync) {
	// shift arguments if data argument was omited
	if (jQuery.isFunction(data)) {
		sync = sync || callback;
		callback = data;
		data = {};
	};
	$.ajax({
		type: 'POST',
		url: url,
		data: data,
		dataType: 'json',
		success: callback,
		error: callback,
		async: (sync!=true)
		});
	};

// Handle updates
function handlerDataUpdate(data) {
try {
	$.each(data, function (path, obj) {
		var pobj = tree;
		var pe_arr = path.split('.');
		for (var pe in pe_arr.slice(0, -1)) {
			pobj = pobj[pe_arr[pe]];
			};
		pobj[pe_arr.slice(-1)] = obj;

// Restrict UI updates only to some paths. This is optional to make parsing faster. 
// Remove this "if" to parse all updates
	if (
	(new RegExp('^devices\\..*\\.instances\\..*\\.commandClasses\\.
	(37|38|48|49|50|67|98|128)\\.data.*$')).test(path) || (new RegExp(
	'^areas\\.data\\..*\\..*$')).test(path))
	$.triggerPath.update(path); // this updates objects bound to paths
	});
} catch(err) {
	console.log("Error in handlerDataUpdate: " + err.stack);
};
};
\end{lstlisting}