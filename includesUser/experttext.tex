\subsection{Expert Commands} 

The expert command tabs allows accessing all command class related data of the Z-Way controller as 
well as executing all command class related commands in a generic way. This interface overlay the
 more function-oriented tabs for switches, sensor etc. and can be used for debugging and testing purposes.

Just select the device on the left hand side and access the command interface on the right hand side. 
Clicking on the names of the Command Classes allows accessing the full data structure of this particular 
command class of the device chosen.

\subsection{Controller Information}

The controller information tab shows all conttoller information. The buttons “Show Controller Data” 
shows the internal Z-Way data structure related to the specific controller function of the controller 
device. The button “Show controller device data” show the generic device related data of the controller device.

The information given on this page is only relevant for advanced Z-Wave developers and for debugging.


\subsection{Queue Inspection}

In expert mode the Network management tab shows a button “Inspect queue” allows monitoring the work of the 
job execution of the Z-Way backend. Every communication with the Z-Wave transceiver is scheduled into a 
job and queued that it can transmitted over the serial hardware interface.

The table shows the active jobs with their respective status and additional information.
\begin{itemize}

\item n: This column shows the number of sending attempts for a specific job.  Z-Way tries three times to dispatch a job to the transceiver.

\item W,S,D: This shows the status of the job. If no indicator is shown the job is in active state. This means that the controller just tries to execute the job. W- states indicated that the controller believes that the target device of this job is in deep sleep state. Jobs in “W” state will remain in the queue to the moment when the target devices announces its wakeup state by sending a wakeup notification to the controller. Jobs in “S” state remain in the waiting queue to the moment the security token for this secured information exchanged was validated. 
“D” marks a job as done. The job will remain in the queue for information purposes until a job garbage collection removed it from the queue.

\item ACK: … shows if the Z-Wave transceiver has issued an ACK message to confirm that the message was successfully received by the transceiver. This ACK however does not confirm that the message was delivered successfully. A successful delivery of a message will result in a “D” state of this particular job. 

If the ACK field is blank, then no ACK is expected. A “.” indicates that the controller expects an ACK but the ACK was not received yet. A “+” indicates that an ACK was expected and was received.

\item RESP: ... shows if a certain command was confirmed with a valid response. Commands are either answered by a response or a callback.

If the RESP field is blank, then no Response is expected. A “.” indicates that the controller expects a Response but the Response was not received yet. A “+” indicates that a Response was expected and was received.

\item Cbk: … shows that the z-wave transceiver has finally reported the status of the delivery. This may either be a success  - resulting in a “D” state right away or a failure, which would either trigger a retransmission or an abort of the job.

If the Cbk field is blank, then no callback is expected. A “.” indicates that the controller expects a Callback but the Callback was not received yet. A “+” indicates that a Callback was expected and was received.

\item Timeout: …shows the time left until the job is de queued.

\item Node Id … shows the id of the target node. Communication concerning the network – like inclusion of new nodes – will have the controller node id as target node ID. For command classes command the node ID of the destination Node is shown. For commands directed to control the network layer of the protocol, the node id is zero.
 
\item Description: shows a verbal description of the job

\item Progress: shows a success or error message depending on the delivery status of the message. Since Z-Way tries three times to deliver a job up to 3 failure messages may appear.

\item Buffer: … shows the hex values of the command sent within this job.
\end{itemize}



