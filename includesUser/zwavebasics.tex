\section {Z-Wave Network Quick Basics}

\subsection{Network and controllers}

A Z-Wave network consists of various devices interconnected by a wireless communication protocol. Thanks to the 
Z-Wave standard products from different vendors can work together seamlessly. 

Another advantage of Z-Wave is their ability to act as repeaters and forward data packets between nodes 
not able to communicate directly over the air. This extends the range of a Z-Wave network and improves 
stability. In order to perform this packet routing and forwarding the particular node needs to be mains 
powered. Battery operated nodes can’t act as repeaters.

Z-Wave differentiates between portable and static controllers to control other devices. Portable 
controllers change their location and they are battery powered. To allow long battery life-time 
they are inactive most of the time and will only communicate with other devices during manual 
interaction (pressing a button). 

Static controllers are installed on a fixed location. They are mains powered and therefore able to 
stay alive all the time to communicate with other devices. 

Z-way implements a static controller and will usually be the primary controller handling all inclusions 
and exclusions of devices in and from the network.  However Z-Way can also be included as a secondary 
controller in other networks controlled by other Z-Wave controllers. Z-way will then follow the network 
setup of this other network.


\subsection{Inclusion and Configuration of various device types}

Z-Wave devices can be mains powered or battery operated.  The inclusion process for both devices types is 
similar, however battery operated devices need special handling.

\subsubsection{Mains Powered Devices}

A mains powered device is easy to configure after inclusion since the device will receive all configuration 
commands and execute them immediately. Mains powered devices are always listening to other commands and can 
repeat commands to other nodes. 

\subsubsection{Battery Operated Devices}

The main objective of a battery-operated device is to preserve the battery power and only use as much 
battery power as needed.  Battery powered devices are therefore in a deep-sleep state most of the time.
 In deep-sleep state they are not able to communicate with other devices. 

In order to communicate with an other device the battery-operated device needs to be woken up and send to 
sleep mode right after the communication took place. To maintain a minimal level of “responsiveness” and 
to allow to configure and to use battery-operated devices Z-Wave offers three basic solutions:
\begin{itemize}
\item Devices with wakeup intervals
\item Frequently listening battery devices 
\item Devices with manual wakeup
\end{itemize}

\subsubsection{Wakeup Interval}

Devices with wakeup interval will wakeup after a defined interval and send out a wakeup notification. 
Other devices such as the Z-Way controller are able to communicate with this device and send out messages 
to this device (The controller know about the status of the battery operated device and will queue messages 
in a waiting queue.) After all communication is done the controller is supposed to send the device back into 
deep sleep.
If no communication happens the battery-operated device will go back into deep sleep mode after a defined 
time (typically some seconds – up to one minute).

The best practice of Z-Wave suggests that battery operated devices stay awake for a defined time right 
after inclusion and go into deep sleep mode afterwards. This first awake-time is device dependent and 
varies from 1 minute to one hour.

Hence it is recommended to configure the device right after inclusion to make sure the device is still 
alive. If the battery operated device is already in sleep state all configuration commands will be 
queued and executed after the next scheduled wakeup if and only if a valid wakeup interval was configured 
during the first part of the configuration process. If this was not the case the device needs to be woken 
up manually.

If the device went already into deep sleep before the configuration was finished it is recommended to 
wake up the device manually to speed up the configuration process. Otherwise the configuration will happen 
after the next scheduled wakeup.

Some battery-operated devices may not go into sleep mode at all after inclusion but need to be sent into 
deep sleep after configuration is finished. This is done by Z-Way automatically.

The configuration of the wakeup interval is a tradeoff between maximum battery life-time  (suggest a 
very long wakeup interval with few wakeup cycles) and some responsiveness of the device in case of a 
network-reorganization. Typical values are between 5 minutes and 5 hours.

\subsubsection{Frequently Listening Devices}

Z-Wave has introduced frequently listening devices (FLIRS). These devices will wakeup at least once in 
a second and try to receive a message. The trick is that the wakeup is so short, that on average the power 
consumption of FLIRS devices are low enough to allow battery life times greater than one year. FLIRS 
devices can be configured without any problems like mains powered devices since every command will 
always be received latest after one second.
However FLIRS device will not route other devices messages to preserve battery power.

\subsubsection{Devices with manual wakeup}

Remote controls are battery operated as well but they are only awake if a button is pressed. Remote 
controls will not wakeup regularly to check for queued messages. Hence whenever a remote control is 
configured from the Z-Way controller, the device needs to be woken up manually. Please refer to the 
manual of the remote control for further instructions how to wakeup the device.  


 